%%
%% Author: mesenev
%% 6/18/18
%%

% Preamble
\documentclass[11pt]{article}

% Packages
\usepackage{a4wide}
\usepackage[T1]{fontenc}
\usepackage[utf8]{inputenc}
\usepackage{amsmath}

% Document
\begin{document}
\section[Вариационные формулировки одномерных задач]{Вариационные формулировки одномерных задач}\label{sec:variationalFormataion}
    Рассмотрим простейшую одномерную краевую задачу, заключающуюся в нахождении функции $u$
    из условий
    \begin{equation}
        \label{3.1}
        [Lu](x) \equiv - u''(x) = f(x) \, \text{в} \, (0, l), \\
    \end{equation}
    \begin{equation}
        \label{3.2}
        u(0) = 0, u(l) = 0.
    \end{equation}
    Обозначим через $C^{m} = C^m [0,l]$ пространство функций, определённых и непрерывных на замкнутом интервале $[0, l]$ вместе со всеми производными до порядка $m>0$ включительно.
    Определив в данном пространстве скалярное произведение и норму
    \begin{equation}
        \label{3.3}
        (u, v)_m = \int_0^l [uv + u'v'+ \ldots + u^{(m)}v^{(m)}]dx, \,  \| u \|_m = (u,u^{1/2}_m),
    \end{equation}
    получим бесконечномерное евклидово (или \textit{предгильбертово}) пространство, т.е. не полное по введённой норме функциональное пространствосо скалярным произведением в~\ref{3.3}.
    Через $P$ обозначим ммножество фунсведения независимо от формы их представлениякций на $[0, l]$ удовлетворяющих краевым условиям~\ref{3.2}.
    Предположим, что
    \begin{equation}
        f \in C^0
    \end{equation}
    т.е. что $f$ непрерывна на $[0, l]$ и обозначим через $F$ линейную форму на пространстве $C^0$, порождённую функцией f:
    \begin{equation}
        F(v)=\int_0^l f(x)v(x)dx.
    \end{equation}
    Поставим в соответствие задаче
    \begin{equation}
        \label{1.1}
        Lu \equiv -u'' = f \, \text{в} \, (0, l), u(0) =u(l) = 0
    \end{equation}
    следующие четыре задачи.
    Задача 0. Найти функцию $u \in  C^2 \cap P$, удовлетворяющую уравнению~\ref{3.1} в каждой точке $x \in (0,l)$.
    Задача 1. Пусть $U = C^2  \cap P, V=C^0$. Найти такую функцию $u \in U$, что
    \[ A_1(u, v) \equiv \int_0^l Luvdx=F(v) \, \forall v \in V \]
    Задача 2. Пусть $U=V=C^1 \cap P$. Найти такую функцию $u \in U$, что
    \[ A_2(u, v) \equiv \int_0^l u'v'dx = F(v) \, \forall v \in V. \]
    Задача 3. Пусть $U= C^0, V = C^2 \cap P$. Найти такую функцию $u \in U$, что
    \[ A_3(u,v) \equiv \int_0^l uLvdx = F(v) \, \forall v \in V. \]

    Данные задачи эквивалентны между собой и, более того, каждая из них имеет единственное решение, причем указанные решения совпадают.

    Сформулируем теорему
\begin{theorem}
    При выполнении условия~\ref{3.4} задачи 0, 1, 2 и 3 эквивалентны и, более того, имеют единственное решение $u \in C^2 [0,l] \cap P$.
\end{theorem}
    Отметим, что задачи 1, 2 и 3 являются \textit{сильной, полусильной} (либо \textit{полуслабой}) и \textit{слабой вариационными формулировками} задачи~\ref{1.1}.
    Основой слабой формулировки является процесс умножения на \textit{тестовую} или \textit{пробную} функцию, из некоторого пространства $V$, в результате чего получается соотношение
    \begin{equation}
        \label{3.14}
        \int_0^l Luvdx =F(v) \, \forall v \in V.
    \end{equation}
    Тип полученной формулировки определяется выбором пространства $V$.
    Сильная вариационная формулировка получается при выборе пространства $V = C^0$.
    В таком случае мы ищем функцию $u$ в классе $U = C^2 \cap P$ функций имеющих достаточно сильные дифференциальные свойства, фактически совпающие со свойствами классического решения задачи~\eqref{1.1}
    Полагая далее $V = C^1 \cap P$, выражение в левой части~\eqref{3.14} можно проинтегрировать один раз по частям.
    Это позволит искать решение задачи~\eqref{1.1} не в классе $C^2$, а в более широком классе функций $C^1$.
    В результате мы получаем Задачу 2.
    Наконец, выбирая $V = C^2 \cap P$, можно дважды проитегрировать левую часть~\eqref{3.14} и получить Задачу 3.
    При этом двукратное интегрирование по частям позволит искать $u$ в классе $C^0$, обладающиъ совсем слабыми дифференциальными свойствами -- лишь непрерывностью.

    Полуслабая формулировка обладает двумя важными особенностями:
    1) для полуслабой формулировки $U = V$, т.е. решение $u$ и пробные функции $v$ берутся из одного пространства V;
    2) билинейная форма $A_2$ симметрична на $V$, т.е.
    \[ A_2(u,v) = \int_0^l \equiv u' v' dx = A_2(v,u), \forallu,v \in V \]
    и положительна в том смысле, что
    \[ A_2(v,v) = \int_0^l (v')^2 dx > 0 \forall v \in V, v \neq 0. \]
    Таким образом наша задача эквивалентна следующей задаче минимизации

    Задача 4. Пусть $V = C^1 \cap P$. Найти такую функцию $u$, что
    \[ u \in V : J (u) \leq J(v) \, \forall v \in V. \]
    Здесь J -- квадратичный функционал, определяемый формулой
    \[ J(v) = \frac{1}{2} A_2(v,v)-F(v)=\frac{1}{2} \int_0^l (v')^2 dx-\int_0^l f v dx. \]
    




\end{document}