\section{Модели сложного теплообмена}\label{sec:--}

\subsection{Уравнение переноса теплового излучения}\label{subsec:---}

Уравнение переноса излучения описывает поле интенсивности излучения
при взаимодействии теплового излучения с поглощающей, излучающей и рассеивающей средой
(radiatively participating medium).
Будем предполагать, что среда имеет постоянный показатель преломления 𝑛, является неполяризующей,
находится в состоянии покоя (по сравнению со скоростью света) и в локальном
термодинамическом равновесии [67, с. 280].


Спектральной интенсивностью излучения 𝐼ν(𝑥, ω, 𝑡) [Вт/(м 2 · стер · Гц)]
называется количество энергии излучения, проходящего через единичную
площадку, перпендикулярную направлению распространения ω, внутри единичного телесного угла,
осью которого является направление ω, в единичном
интервале частот, включающем частоту ν, и в единицу времени.
Считаем, что направления излучения ω связаны с точками единичной сферы 𝑆 = {ω ∈ R 3 : |ω| = 1}.


