\documentclass[10pt]{article}
\usepackage[utf8]{inputenc}
\usepackage[T1]{fontenc}
\usepackage{amsmath}
\usepackage{amsfonts}
\usepackage{amssymb}
\usepackage[version=4]{mhchem}
\usepackage{stmaryrd}
\usepackage{graphicx}
\usepackage[export]{adjustbox}
\graphicspath{ {./images/} }
\usepackage{bbold}

\title{Обратные задачи сложного теплообмена }


\author{Месенев Павел Ростиславович}
\date{}


\begin{document}
\maketitle
Федеральное государственное автономное образовательное учреждение высшего образования «Дальневосточный федеральный университет»

На правах рукописи



Специальность 05.13.18 -

«Математическое моделирование, численные методы и комплексы программ»

Диссертация на соискание учёной степени

кандидата физико-математических наук

Научный руководитель:

доктор физико-математических наук, профессор

Чеботарев Александр Юрьевич

Владивосток - 2020

\section{Оглавление}
Стр.

Введение .............................. 3

Глава 1. Модели сложного теплообмена . . . . . . . . . . . . . . . . 10

$1.1$ Уравнение переноса теплового излучения . . . . . . . . . . . . . 10

$1.2$ Диффузионное $P_{1}$ приближение .................. 16

Глава 2. Теоретический анализ обратных задач сложного теплообмена .......................... 24

$2.1$ Граничная обратная задача для уравнений сложного теплообмена 24 2.1.1 Постановка обратной задачи . . . . . . . . . . . . . . . . . 24

2.1.2 Формализация задачи нахождения квазирешения . . . . . 25

$2.1 .3$ Анализ экстремальной задачи $\ldots$. . . . . . . . . . . . . . . 26

$2.2$ Оптимизационный метод для задачи радиационного теплообмена 28

$2.2 .1$ Формализация задачи управления . . . . . . . . . . . . . . 29

$2.2 .2$ Разрешимость задачи $(C P)$. . . . . . . . . . . . . 31

$2.2 .3$ Условия оптимальности $\ldots \ldots$. . . . . . . . . . 32

$2.2 .4$ Аппрокимация задачи с условиями типа Коши . . . . . . . 34

Глава 3. Численные эксперименты . . . . . . . . . . . . . . . . . . . . 36

$3.1$ Численный алгоритм нахождения квазирешения . . . . . . . . . 36

$3.2$ Численная реализация решения оптимизационной задачи . . . . . 38

\begin{center}
\includegraphics[max width=\textwidth]{2022_10_02_2aab7a2952871da15492g-02(1)}
\end{center}

Приложение А. Приложение . . . . . . . . . . . . . . . . . . . 43

A.1 Расчёт квазирешения для обратной граничной задачи в пакете

\begin{center}
\includegraphics[max width=\textwidth]{2022_10_02_2aab7a2952871da15492g-02}
\end{center}

A.2 Код программы генерации изображений на языке Python . . . . . 45

\section{Введение}
Под сложным теплообменом понимают процесс распространения тепла, в котором участвуют несколько видов переноса тепла - радиационный, кондуктивный, конвективный. При чём в данном процессе радиационный перенос тепла занимает существенную роль при высоких температурах. С математической точки зрения процесс сложного теплообмена моделируется системой из дифференциального уравнения теплопроводности, а также интегро-дифференциального уравнения переноса излучения.

Решение уравнения переноса излучения является трудновычислимой задачей из-за того, что помимо временной и пространственной переменной также задействовано векторное поле, задающее направление излучения. В связи с этим для уравнения переноса излучения применяют ряд аппроксимаций, в том числе диффузионное $P_{1}$ приближение, которое использует усреднённую по всем направлениям интенсивность излучения. Широко используемое $P_{1}$ приближение является частным случаем метода сферических гармоник $\left(P_{N}-\right.$ приближения $)$ и упрощенного метода сферических гармоник $\left(S P_{N}\right.$-приближения, $S P_{1}$ эквивалентно $\left.P_{1}\right)$.

В классических прямых задачах сложного теплообмена задаются параметры системы и по ним расчитывается состояние системы - температурное поле и интенсивность теплового излучения. Обратные задачи сложного теплообмена состоят в разыскании исходных параметров системы по некоторым известным сведениям о температурном поле или интенсивности излучения. Например, обратные задачи, связанные с теплопроводностью, обычно связаны с оценкой неизвестного граничного теплового потока при известной температуре.

Отметим трудности, возникающие при решении обратных задач сложного теплообмена.

Эти задачи математически классифицируются как некорректные в общем смысле, из-за высокой нестабильности решений. Как следствие, обратные задачи теплообмена долгое время не представляли физического интереса. Появление в 50 -х годах эвристических методов и в 60 -70е годы методов оптимизации позволило исправить проблемы некорректности исследуемых задач. В основе таких методов лежит идея замены исходной задачи на задачу оптимизации с использованием регуляризации, которая и позволяет преодолеть проблемы нестабильности решений. Приведём некоторые пионерские работы А.Н. Тихонова, О.М. Олифанова, J.V.Beck и других, нашедших различные способы преодоления неустойчивостей обратных задач.

Диссертация посвящена теоретическому анализу обратных стационарных задач сложного теплообмена в трёхмерной области в рамках $P_{1}$-приближения уравнения переноса излучения. Теоретические результаты проиллюстрированы численными примерами.

Исследование математических моделей радиационного теплопереноса учитывающих одновременно вклад эффектов теплопроводности и излучения даёт теоретическую основу для инженерных решений в различных областях, таких как производство стекла [1], лазерная интерстициальная термотерапия [2], и другие. Главной особенностью данных процессов является существенное влияние излучения на теплообмен при высоких температурах.

Степень разработанности темы исследования. Приведём работы [3, $4,5,6,7,8,9,10,11,12,13]$, которые посвящены анализу разрешимости моделей сложного теплообмена между телами, разделенными прозрачной средой. Эти модели включают уравнение теплопроводности с нелинейным нелокальным краевым условием, моделирующим тепловое излучение границы области и теплообмен излучением между частями границы, в $[4,6,7,9,11,12,13]$ pacсмотрены стационарные модели, в $[3,5,8,9,10,11,12]$ - нестационарные.

В следующих работах исследована разрешимость моделей сложного теплообмена в полупрозрачной среде, в которых для моделирования радиационного теплообмена используется полное уравнение переноса излучения. В [14, 15] доказана однозначная разрешимость одномерных стационарных задач радиационно-кондуктивного теплообмена, в [16, 17, 18] доказана однозначная разрешимость трехмерных задач: в [18] исследована стационарная модель, в [17] - нестационарная, в [16] - квазистационарная. Под квазистационарными моделями сложного теплообмена понимаются модели, включающие нестационарное уравнение теплопроводности и стационарное уравнение переноса излучения.

Отметим работы $[19,20,21,22,23,24]$, посвященные разработке численных методов для указанных моделей. В работах А.А.Амосова [25, 26, 27, 28] доказана однозначная разрешимость стационарных и квазистационарных моделей сложного теплообмена в системе полупрозрачных тел, где для описания распространения излучения используется уравнение переноса излучения с краевыми условиями, моделирующимиотражение и преломление излучения на границах тел, также учитывается зависимость интенсивности излучения и оптических свойств тел от частоты излучения: в $[25,26]$ ставились условия диффузного отражения и преломления излучения, в $[27,28]$ - условия отражения и преломления излучения по законам Френеля.

Отметим следующие работы, посвященные решению обратных задач в рамках моделей сложного теплообмена с полным уравнением переноса излучения. В [29] проведен теоретический анализ задачи оптимального управления источниками тепла в рамках квазистационарной модели сложного теплообмена, включающей полное уравнение переноса излучения: доказана однозначная разрешимость прямой задачи, доказана разрешимость задачи управления, и получены условия оптимальности. В [30] разработан численный алгоритм решения задачи оптимального управления источниками тепла и излучения в рамках стационарной модели сложного теплообмена с полным уравнением переноса излучения. Работа [31] посвящена теоретическому и численному анализу обратной задачи восстановления начального распределения температуры по известной зависимости температуры на границе области от времени в рамках квазистационарной модели сложного теплообмена.

Отметим также работы $[32,33,34,11]$, посвященные анализу задач оптимального управления для стационарных моделейсложного теплообмена в прозрачной среде, включающих уравнение теплопроводности с нелинейным нелокальным краевым условием, моделирующим тепловое излучение границы области и теплообмен излучением между частями границы, и работу [35], в которой построен численный алгоритм решения задачи оптимального управления граничными коэффициентами в одномерной нестационарной модели, включающей уравнение теплопроводности с нелинейным краевым условием, которое описывает тепловое излучение границ.Работы $[36,37,38,39]$ посвящены численному моделированию в рамкахдиффузионных моделей сложного теплообмена, в [40] исследована схема методаконечных объемов для решения квазистационарной системы уравнений сложного теплообмена на основе $P_{1}$-приближения уравнения переноса излучения.

Сравнение $P_{1}$-приближения с другими методами аппроксимации уравнения переноса излучения проводилось в [41, 42, 24, 37, 43, 44]. Вывод и численный анализ нестационарного $P_{1}$-приближения выполнен в [45, 46 , 47, 42, 44]. В работах R. Pinnau, О. Tse [48, 49] проведен теоретический анализ квазистационарных моделей сложного теплообмена на основе $S P_{1}$ и $S P_{3}$-приближений. Эти модели включают уравнение теплопроводности, стационарное $S P_{N}$-приближение, а также в [49] уравнения Навье - Стокса приближении Буссинеска.

В [48] доказаны существование, единственность и ограниченность решения задачи сложного теплообмена на основе $P_{1}$-приближения без источников тепла и излучения, в $[49]$ доказана однозначная разрешимость задачи свободной конвекции с радиационным теплообменом на основе $S P_{3}$-приближения в двумерной области, в этой модели присутствуют источники тепла с ограниченной плотностью. В работах А.Е.Ковтанюка, А.Ю.Чеботарева [50, 51, 52] доказана однозначная разрешимость краевых задач для стационарных моделей сложноготеплообмена на основе $P_{1}$-приближения, доказана сходимость метода простой итерации нахождения решения.

Отметим, что численная реализация данного метода затруднена, поскольку на каждой итерации необходимо решить нелинейное эллиптическое уравнение. В [53] доказана однозначная разрешимостьсходной субдифференциальной краевой задачи с многозначной зависимостьюкоэффициента излучения границы от интенсивности излучения. В [54, 55] получены результаты о существовании и единственности решений обратных задач для стационарной диффузионной модели сложного теплообмена, которые заключаются в нахождении неизвестной плотности источников тепла в виделинейной комбинации заданных функционалов при известных значениях этихфункционалов на решении краевой задачи.Работы R. Pinnau, O. Tse $[48,49]$ посвящены теоретическому анализу задач оптимального управления температурой на границе области в рамках квазистационарных моделей сложного теплообмена на основе $S P_{N}$-приближений: доказана разрешимость задач управления, найдены необходимые условия оптимальности В $[56,57,47,58,59,60]$ разработаны численные методы решения задач оптимального управления граничной температурой для квазистационарноймодели сложного теплообмена на основе $P_{1}$-приближения, при этом в $[56,57,58]$ использовалась модель с учетом зависимости коэффициента поглощения от частоты излучения: в $[47,59]$ минимизировалось отклонение поля температурыот заданного и для решения задачи оптимизации применялся метод Ньютона, в [58] для решения задачи минимизации отклонения поля температуры от заданного применялся метод проекции градиента, в [56, 60] минимизироваласьнорма градиента температуры и для решения задачи оптимизации применялся метод проекции градиента, в [57] решалась задача минимизации отклонения поля температуры от заданного на основе серии из трех моделей, аппроксимирующих уравнение переноса излучения с разной точностью, использовался оптимизационный метод второго порядка.

В [61] проведен теоретический анализзадачи оптимального управления температурой на границе области в рамкахстационарной диффузионной модели сложного теплообмена, для численногорешения задачи управления применен метод проекции градиента. В работах А.Е.Ковтанюка, А.ЮО. Чеботарева и др. $[50,62,63,64]$ исследованы задачи оптимального управления коэффициентом излучения границы области в рамках стационарной модели сложного теплообмена на основе $P_{1}$-приближения. В $[50,64]$ выведены необходимые условия оптимальности для задачи максимизации выходящей из среды энергии, доказана разрешимость задачи управления и получены достаточные условия регулярности системы оптимальности, которые выполняются при достаточно большой скорости движения среды и малых размерах области. В [62, 63] получены достаточные условия оптимальности для задач максимизации и минимизации полей температуры и излучения во всей областитеплообмена, доказана сходимость метода простой итерации нахождения оптимального управления, эти исследования были выполнены позже аналогичной работы автора для нестационарной модели.

Таким образом, ряд важных задач, относящихся к моделированию и оптимизации сложного теплообмена на основе диффузионного приближения, оставался нерешенным: исследование разрешимости нестационарной задачи сложного теплообмена с источниками тепла и излучения и нестационарной задачисвободной конвекции с радиационным теплообменом в трехмерной области, исследование устойчивости по Ляпунову стационарных решений, вывод диффузионной модели сложного теплообмена в многослойной среде, анализ сходимостиметода Ньютона для уравнений сложного теплообмена, разработка численныхметодов решения задач оптимального управления коэффициентом излученияграницы области в рамках нестационарных моделей сложного теплообмена, доказательство регулярности условий оптимальности для задачи оптимального управления коэффициентом излучения границы в рамках стационарной модели.

Цели и задачи диссертационной работы. Цели работы - теоретическое исследование разрешимости обратных стационарных задач сложной теплопроводности, разработка численных методов решения исследуемых краевых задач, а также задач оптимального управления. Разработка вычислительных программ для постановки численных экспериментов и демонстрации результатов расчётов.

Перед началом работы были поставлены следующие задачи:

\begin{itemize}
  \item Исследовать разрешимость задачи по нахождению коэффициента отражения участка границы для стационарной модели, по дополнительной информации о температурном поле.

  \item Разработать численный метод по нахождению решения для соответствующей экстремальной задачи

  \item Исследовать стационарную задачу оптимального управления для уравнений радиационно-кондуктивного теплообмена в трехмерной области в рамках $P_{1}$-приближения уравнения переноса излучения.

  \item Результаты теоретического анализа проиллюстрировать численными примерами.

\end{itemize}

Научная новизна. Результатом работы является теоретический анализ разрешимости обратных задач сложного теплообмена. доказано существование квазирешения для первой рассматриваемой задачи. Реализован алгоритм градиентного спуска для решения экстремальной задачи и представлены результаты численных экспериментов. Далее показано, что последовательность решений экстремальных задач сходится к решению краевой задачи с условиями типа Коши для температуры. Результаты теоретического анализа также проиллюстрированы численными примерами.

Теоретическая и практическая значимость. Исследование однозначной разрешимости экстремальных задач, а также задач оптимального управления крайне важно при реализации численных алгоритмов и позволяет судить об адекватности полученных решений.

Задачи оптимизации имеют крайне важное практическое применение при выборе параметров системы для получения желаемой температуры или теплового излучения. Необходимость выбора параметров системы возникает при проектировании инженерных установок в которых присутствуют процессы сложного теплообмена.

Разработанные комплексы программ служат практическим подтверждением теоретических результатов, а также могут быть использованы в качестве примеров для решения подобных задач. Научная значимость данной работы состоит в теоретическом вкладе в исследования корректности и разрешимости задач сложного теплообмена. Реализация конкретных методов решения проблем оптимального управления, в свою очередь, имеет высокую значимость для решения прикладных инженерных задач по проектированию тепловых установок с заданными температурными свойствами.

Методология и методы исследования. В работе широко использовались методы математического и функционального анализа, теории дифференциальных уравнений в частных производных, теории экстремальных задач. Для разработки численных алгоритмов решения применялись методы вычислительной математики, объектно-ориентированное и функциональное программирование, методы оптимизации и другие. Положения, выносимые на защиту. В области математического моделирования:

\begin{enumerate}
  \item Разрешимость экстремальной задачи для стандартной модели радиационно-диффузионного теплообмена.
\end{enumerate}

Публикации. Результаты диссертационного исследования опубликованы в двух статьях $[65,66]$ в изданиях, рекомендованных ВАК.

Объем и структура работы. Диссертация состоит из введения, 3 глав, заключения и 1 приложения. Полный объём диссертации составляет 46 страниц, включая 5 рисунков и 0 таблиц.

\section{Глава 1. Модели сложного теплообмена}
\section{$1.1$ Уравнение переноса теплового излучения}
Уравнение переноса излучения описывает поле интенсивности излучения при взаимодействии теплового излучения с поглощающей, излучающей и рассеивающей средой (radiatively participating medium). Будем предполагать, что среда имеет постоянный показатель преломления $n$, является неполяризующей, находится в состоянии покоя (по сравнению со скоростью света) и в локальном термодинамическом равновесии [67, с. 280].

Спектральной интенсивностью излучения $I_{v}(x, \omega, t)\left[\mathrm{Bт} /\left(\mathrm{M}^{2} \cdot\right.\right.$ стер $\cdot$ Гц $\left.)\right]$ называется количество энергии излучения, проходящего через единичную площадку, перпендикулярную направлению распространения $\boldsymbol{\omega}$, внутри единичного телесного угла, осью которого является направление $\boldsymbol{\omega}$, в единичном интервале частот, включающем частоту $v$, и в единицу времени. Считаем, что направления излучения $\omega$ связаны с точками единичной сферы $S=\{\omega \in$ $\left.\mathbb{R}^{3}:|\omega|=1\right\}$

Рассмотрим пучок излучения интенсивностью $I_{v}(x, \omega, t)$, распространяющегося в поглощающей, излучающей и рассеивающей среде в заданном направлении. Энергия излучения будет уменьшаться вследствие поглощения излучения веществом и отклонения части его от первоначальной траектории в результате рассеяния во всех направлениях, но одновременно она будет возрастать вследствие испускания излучения веществом.

Обозначим через $\kappa_{a v}\left[\mathrm{M}^{-1}\right]$ спектральный коэффициент поглощения, равный доле падающего излучения, поглощенной веществом на единице длины пути распространения излучения. Приращение интенсивности излучения за счет поглощения равно $\left(d I_{v}\right)_{\text {погл }}=-$ К $_{a v} I_{v} d s$, где $d s$ - элемент пути. Отметим, что $1 / \kappa_{a v}$ есть средняя длина свободного пробега фотона до его поглощения веществом [67, с. 281].

Для получения выражения для испускания излучения элементом объема часто используется предположение о локальном термодинамическом равновесии. Оно означает, что любой малый элемент объема среды находится в локальном термодинамическом равновесии, вследствие чего состояние любой точки может быть охарактеризовано локальной температурой $T(x)$. Это предположение законно, когда столкновения атомов в веществе происходят столь часто, что это приводит к локальному термодинамическому равновесию в каждой точке $x$ среды. В этом случае испускание излучения элементом объема можно описать с помощью функции Планка [68, с. 36]. Приращение интенсивности излучения за счет испускания равно $\left(d I_{v}\right)_{\text {исп }}=j_{v} d s$, где $j_{v}-$ коэффициент испускания. В локальном термодинамическом равновесии справедлива формула [68, с. 36], [67, с. 282] $j_{v}=\kappa_{a v} I_{b v}$, где $I_{b v}$ - интенсивность излучения абсолютно черного тела.

Абсолютно черным называется тело, которое поглощает все падающее со всех направлений излучение любой частоты без отражения, пропускания и рассеяния. Из закона Кирхгофа следует, что абсолютно черное тело также излучает максимальное количество энергии при данной температуре [68, с. 25], [67, с. 5]. Интенсивность излучения абсолютно черного тела при температуре $T$ равна

$$
I_{b v}(T)=\frac{2 h v^{3} n^{2}}{c_{0}^{2}\left(e^{h v / k T}-1\right)},
$$

где $h$ - постоянная Планка, $k$ - постоянная Больцмана, $c_{0}-$ скорость света в вакууме, $T$ - абсолютная температура, $n$ - показатель преломления. Интегральная интенсивность излучения абсолютно черного тела $I_{b}(T)$ вычисляется по формуле [68, с. 28], [67, с. 10]

$$
I_{b}(T)=\int_{0}^{\infty} I_{b v}(T) d v=\frac{n^{2} \sigma T^{4}}{\pi},
$$

где $\sigma$ - постоянная Стефана - Больцмана.

Рассеяние излучения учитывается так же, как поглощение, с той разницей, что рассеянная энергия просто перенаправляется и возникает в приращении интенсивности излучения в другом направлении. Различают когерентное и некогерентное рассеяние. Рассеяние называется когерентным, если рассеянное излучение имеет ту же самую частоту, что и падающее излучение, и некогерентным, если частота рассеянного излучения отличается от частоты падающего излучения. В дальнейшем мы будем рассматривать только когерентное рассеяние. Обозначим через $\kappa_{s v}\left[\mathrm{M}^{-1}\right]$ спектральный коэффициент рассеяния, равный доле падающего излучения, рассеянной веществом во всех направлениях на единице длины пути распространения излучения. Тогда приращение интенсивности излучения за счет «рассеяния вне» равно $\left(d I_{v}\right)_{\text {расс.вне }}=-$ К $_{s v} I_{\gamma} d s$. Для описания «рассеяния в» вводится неотрицательная фазовая функция рассеяния $P_{v}\left(\boldsymbol{\omega}, \boldsymbol{\omega}^{\prime}\right)$ такая, что $\frac{1}{4 \pi} \int_{S} P_{v}\left(\boldsymbol{\omega}, \boldsymbol{\omega}^{\prime}\right) d \boldsymbol{\omega}=1$. Величина $\frac{1}{4 \pi} P_{v}\left(\boldsymbol{\omega}, \boldsymbol{\omega}^{\prime}\right) d \boldsymbol{\omega}$ определяет вероятность того, что излучение частоты $v$, падающее в направлении $\omega^{\prime}$, будет рассеяно в пределах элементарного телесного угла $d \omega$ в направлении $\omega$. Случай $P_{v} \equiv 1$ соответствует изотропному рассеянию. Тогда для того, чтобы получить приращение интенсивности излучения за счет «рассеяния в», нужно проинтегрировать $I_{v}\left(\omega^{\prime}\right) P_{v}\left(\omega, \omega^{\prime}\right) / 4 \pi$ по всем входящим направлениям $\boldsymbol{\omega}^{\prime}[67$, с. 283]: $\left(d I_{v}\right)_{\text {pacc.в }}=d s \frac{\mathrm{K}_{s v}}{4 \pi} \int_{S} I_{v}\left(\omega^{\prime}\right) P_{v}\left(\omega, \omega^{\prime}\right) d \omega^{\prime}$.

Учитывая приращения интенсивности излучения с учетом поглощения, испускания и рассеяния, получим искомое уравнение переноса излучения $[68$, c. 272$],[67$, c. 284$]$ :

$$
\begin{aligned}
\frac{1}{c} \frac{\partial I_{v}(x, \omega, t)}{\partial t}+\omega \cdot \nabla_{x} I_{v}(x, \boldsymbol{\omega}, t)+\mathrm{k}_{v} I_{v}(x, \boldsymbol{\omega}, t)=\\
&=\mathrm{\kappa}_{a v} I_{b v}(T(x, t))+\frac{\mathrm{\kappa}_{s v}}{4 \pi} \int_{S} I_{v}\left(x, \boldsymbol{\omega}^{\prime}, t\right) P_{v}\left(\boldsymbol{\omega}, \omega^{\prime}\right) d \boldsymbol{\omega}^{\prime} .
\end{aligned}
$$

Здесь $\kappa_{v}=\kappa_{a v}+\kappa_{s v}-$ полный спектральный коэффициент взаимодействия, c - скорость света в среде.

Далее получим граничные условия для уравнения переноса излучения. Будем считать, что граница области непрозрачна, испускает излучение диффузно и отражает излучение диффузно и зеркально. Степенью черноты поверхности $\varepsilon_{v}(x)$ называется отношение количества энергии, испускаемого данной поверхностью, к количеству энергии, испускаемому абсолютно черным телом при той же температуре. При диффузном испускании излучения степень черноты не зависит от направления и определяется формулой $\varepsilon_{v}(x)=\frac{I_{v, \text { исп }}(x)}{I_{b v}(T(x))}$, где $I_{v, \text { псп }}(x)$ - интенсивность излучения, испускаемого поверхностью при температуре $T(x)[68$, с. 53]. При диффузном поглощении степень черноты равняется поглотительной способности, которая равна доле поглощенного излучения $[67$, c. 66]. Также введем коэффициенты зеркального и диффузного отражения $\rho_{v}^{s}(x), \rho_{v}^{d}(x)$ как доли зеркально и диффузно отраженного излучения соответственно. Отметим, что в случае непрозрачной поверхности $\varepsilon_{v}+\rho_{v}^{s}+\rho_{v}^{d}=1$. Граничное условие имеет вид [69], [67, с. 289]

$$
\begin{aligned}
&I_{v}(x, \boldsymbol{\omega}, t)=\varepsilon_{v}(x) I_{b v}(T(x, t))+\rho_{v}^{s}(x) I_{v}\left(x, \boldsymbol{\omega}_{R}, t\right)+ \\
&+\frac{\rho_{v}^{d}(x)}{\pi} \int_{\omega^{\prime} \cdot \mathbf{n}>0} I_{v}\left(x, \omega^{\prime}, t\right) \omega^{\prime} \cdot \mathbf{n} d \boldsymbol{\omega}^{\prime}, \boldsymbol{\omega} \cdot \mathbf{n}<0,
\end{aligned}
$$

где $\mathbf{n}$ - вектор внешней нормали к границе области, $\boldsymbol{\omega}$ - входящее направление, $\omega_{R}-$ направление отражения, определяемое из соотношения $\boldsymbol{\omega}+\left(-\omega_{R}\right)=$ $2 \cos \theta \mathbf{n}$, где $\cos \theta=\boldsymbol{\omega} \cdot \mathbf{n}-$ косинус угла между вектором нормали и направлением падающего излучения. Таким образом, $\boldsymbol{\omega}_{R}=\boldsymbol{\omega}-2(\boldsymbol{\omega} \cdot \mathbf{n}) \mathbf{n}$.

Поле температуры описывается уравнением теплопроводности [67, с. 297]:

$$
\rho c_{p} \frac{\partial T(x, t)}{\partial t}-k \Delta T(x, t)+\rho c_{p} \mathbf{v}(x, t) \cdot \nabla T(x, t)=-\div \mathbf{q}_{r}(x, t),
$$

где $T[\mathrm{~K}]$ - температура, $\mathbf{v}[\mathrm{m} / \mathrm{c}]$ - поле скоростей, $k[\mathrm{BT} /($ м $\cdot \mathrm{K})]$ - коэффициент теплопроводности, $c_{p}[$ Дж $/($ кг $\cdot \mathrm{K})]$ - удельная теплоемкость при постоянном давлении, $\rho\left[\kappa г /\right.$ м $\left.^{3}\right]$ - плотность, $\mathbf{q}_{r}$ - вектор плотности потока излучения, определяемый формулой [67, с. 292] $\mathbf{q}_{r}(x, t)=\int_{0}^{\infty} \int_{S} I_{v}(x, \omega, t) \omega d \omega d v$. Дивергенция вектора плотности потока излучения $\div \mathbf{q}_{r}$ характеризует изменение в единицу времени энергии излучения, заключенной в единице объема среды, по всему спектру частот вследствие испускания излучения во всё сферическое пространство и поглощения падающего из него излучения [68, с. 274]. Для нахождения $\div \mathbf{q}_{r}$ проинтегрируем уравнение (1.1) по $\boldsymbol{\omega} \in S$, получим

$$
\begin{gathered}
\frac{1}{c} \frac{\partial}{\partial t} \int_{S} I_{v}(x, \boldsymbol{\omega}, t) d \boldsymbol{\omega}+\div \int_{S} I_{v}(x, \boldsymbol{\omega}, t) \boldsymbol{\omega} d \boldsymbol{\omega}+\kappa_{v} \int_{S} I_{v}(x, \boldsymbol{\omega}, t) d \boldsymbol{\omega}= \\
=4 \pi \kappa_{a v} I_{b v}(T(x, t))+\frac{\kappa_{s v}}{4 \pi} \int_{S} \int_{S} I_{v}\left(x, \boldsymbol{\omega}^{\prime}, t\right) P_{v}\left(\boldsymbol{\omega}, \boldsymbol{\omega}^{\prime}\right) d \boldsymbol{\omega}^{\prime} d \boldsymbol{\omega} .
\end{gathered}
$$

Поменяем порядок интегрирования во втором слагаемом в правой части:

$$
\begin{aligned}
\int_{S} \int_{S} I_{v}\left(x, \omega^{\prime}, t\right) P_{v} &\left(\boldsymbol{\omega}, \boldsymbol{\omega}^{\prime}\right) d \boldsymbol{\omega}^{\prime} d \boldsymbol{\omega}=\\
&=\int_{S} I_{v}\left(x, \boldsymbol{\omega}^{\prime}, t\right) \int_{S} P_{v}\left(\boldsymbol{\omega}, \omega^{\prime}\right) d \boldsymbol{\omega} d \boldsymbol{\omega}^{\prime}=4 \pi \int_{S} I_{v}\left(x, \boldsymbol{\omega}^{\prime}, t\right) d \boldsymbol{\omega}^{\prime} .
\end{aligned}
$$

Обозначим через $G_{v}(x, t)=\int_{S} I_{v}(x, \omega, t) d \omega$ пространственную плотность падающего излучения. Тогда

$$
\frac{1}{c} \frac{\partial G_{v}(x, t)}{\partial t}+\div \int_{S} I_{v}(x, \boldsymbol{\omega}, t) \boldsymbol{\omega} d \omega+\mathrm{\kappa}_{v} G_{v}(x, t)=4 \pi \mathrm{\kappa}_{a v} I_{b v}(T(x, t))+\mathrm{\kappa}_{s v} G_{v}(x, t),
$$

отсюда

$$
\begin{gathered}
\div \int_{S} I_{v}(x, \omega, t) \omega \boldsymbol{\omega} d \boldsymbol{\omega}=4 \pi \mathrm{\kappa}_{a v} I_{b v}(T(x, t))-\mathrm{\kappa}_{a v} G_{v}(x, t)-\frac{1}{c} \frac{\partial G_{v}(x, t)}{\partial t}, \\
\div \mathbf{q}_{r}(x, t)=\int_{0}^{\infty} \mathrm{\kappa}_{a v}\left(4 \pi I_{b v}(T(x, t))-G_{v}(x, t)\right) d v-\frac{1}{c} \frac{\partial}{\partial t} \int_{0}^{\infty} G_{v}(x, t) d v .
\end{gathered}
$$

Таким образом, уравнение теплопроводности принимает вид

$$
\begin{aligned}
&\rho c_{p} \frac{\partial T(x, t)}{\partial t}-k \Delta T(x, t)+\rho c_{p} \mathbf{v}(x, t) \cdot \nabla T(x, t)= \\
&=-\int_{0}^{\infty} \int_{S} \kappa_{a v}\left(I_{b v}(T(x, t))-I_{v}(x, \omega, t)\right) d \omega d v+\frac{1}{c} \frac{\partial}{\partial t} \int_{0}^{\infty} \int_{S} I_{v}(x, \omega, t) d \omega d v
\end{aligned}
$$

Получим граничные условия для уравнения теплопроводности из закона Ньютона-Рихмана. Согласно этому закону, плотность теплового потока пропорциональна разности температур поверхности тела $T$ и окружающей среды $T_{b}$ : $q=h\left(T-T_{b}\right)$. Здесь $h\left[\right.$ Вт $\left./\left(\mathrm{м}^{2} \cdot \mathrm{K}\right)\right]$ - коэффициент теплоотдачи, характеризующий интенсивность теплообмена между поверхностью тела и окружающей средой. Численно он равен количеству тепла, отдаваемому (воспринимаемому) единицей поверхности в единицу времени при разности температур между поверхностью и средой в $1 \mathrm{~K}[70]$. Отметим, что непосредственно на поверхности контакта тела с окружающей средой $T=T_{b}$, однако мы считаем, что температура $T$ на границе поверхности - это температура за пределами пограничного слоя [71]. Рассматривая граничное условие для уравнения переноса излучения (1.2), будем считать, что поверхностное излучение происходит из пограничного слоя, поэтому в качестве аргумента функции $I_{b v}(T)$ будем использовать $T_{b}$. По закону сохранения энергии количество тепла, отводимое с единицы поверхности вследствие теплоотдачи, должно равняться теплу, подводимому к единице поверхности вследствие теплопроводности из внутренних объемов тела, тогда $h\left(T-T_{b}\right)=\mathbf{q} \cdot \mathbf{n}=-k \nabla T \cdot \mathbf{n}=-k \frac{\partial T}{\partial n}$. Таким образом, граничное условие имеет вид:

$$
k \frac{\partial T(x, t)}{\partial n}+h(x)\left(T(x, t)-T_{b}(x, t)\right)=0 .
$$

Следует отметить, что условия третьего рода для температуры обычно ставятся на твердой стенке, где $\mathbf{v} \cdot \mathbf{n}=0$. В данном случае постановка условий третьего рода на всей границе и, в частности, на участке втекания моделирует процесс теплообмена при малых значениях нормальной компоненты скорости.

В дальнейшем мы будем рассматривать случай «серой» среды, когда $\kappa_{a v}$ и $\mathrm{K}_{s v}$ не зависят от частоты $v$, так что $\mathrm{K}_{a v}=\mathrm{K}_{a}, \mathrm{~K}_{s v}=\mathrm{K}_{s}$. Граница области также предполагается «серой». В этом случае уравнения и граничные условия принимают вид (ср. [69]):

$$
\begin{aligned}
& \frac{1}{c} \frac{\partial I(x, \boldsymbol{\omega}, t)}{\partial t}+\boldsymbol{\omega} \cdot \nabla_{x} I(x, \boldsymbol{\omega}, t)+\kappa I(x, \boldsymbol{\omega}, t)= \\
& =\frac{\kappa_{s}}{4 \pi} \int_{S} P\left(\omega, \omega^{\prime}\right) I\left(x, \omega^{\prime}, t\right) d \omega^{\prime}+\kappa_{a} \frac{\sigma n^{2} T^{4}(x, t)}{\pi}, \\
& \rho c_{p} \frac{\partial T(x, t)}{\partial t}-k \Delta T(x, t)+\rho c_{p} \mathbf{v}(x, t) \cdot \nabla T(x, t)= \\
& =-\mathrm{\kappa}_{a}\left(4 \sigma n^{2} T^{4}(x, t)-\int_{S} I(x, \boldsymbol{\omega}, t) d \boldsymbol{\omega}\right)+\frac{1}{c} \frac{\partial}{\partial t} \int_{S} I(x, \boldsymbol{\omega}, t) d \boldsymbol{\omega}, \\
& \text { гр.у.: } I(x, \boldsymbol{\omega}, t)=\varepsilon(x) \frac{\sigma n^{2}}{\pi} T_{b}^{4}(x, t)+\rho^{s}(x) I\left(x, \boldsymbol{\omega}_{R}, t\right)+ \\
& +\frac{\rho^{d}(x)}{\pi} \int_{\omega^{\prime} \cdot \mathbf{n}>0} I\left(x, \omega^{\prime}, t\right) \omega^{\prime} \cdot \mathbf{n} d \omega^{\prime}, \omega \cdot \mathbf{n}<0, \\
& \text { гр.у.: } k \frac{\partial T(x, t)}{\partial n}+h(x)\left(T(x, t)-T_{b}(x, t)\right)=0 \text {. } \\
& \text { Здесь } I=\int_{0}^{\infty} I_{v} d v \text {. }
\end{aligned}
$$

Поставим также начальные условия:

$$
I(x, \boldsymbol{\omega}, 0)=I_{0}(x, \boldsymbol{\omega}), \quad T(x, 0)=T_{0}(x) .
$$

Соотношения (1.3)-(1.7) представляют собой модель сложного теплообмена с полным уравнением переноса излучения.

Перейдем к безразмерным величинам. Обозначим

$$
I(x, \omega, t)=\left(\frac{\sigma n^{2}}{\pi} T_{\max }^{4}\right) I^{*}(x, \boldsymbol{\omega}, t), \quad T(x, t)=T_{\max } \theta(x, t),
$$

Здесь $I^{*}-$ нормализованная интенсивность излучения, $\theta-$ нормализованная температура, $T_{\max }$ - максимальная температура в ненормализованной модели. Подставив (1.8) в уравнения $(1.3),(1.4)$, получим

$$
\begin{aligned}
\frac{1}{c} \frac{\partial I^{*}(x, \omega, t)}{\partial t}+\omega \cdot \nabla_{x} I^{*}(x, \omega, t) &+\kappa I^{*}(x, \omega, t)=\\
=& \frac{\kappa_{s}}{4 \pi} \int_{S} P\left(\omega, \omega^{\prime}\right) I^{*}\left(x, \omega^{\prime}, t\right) d \omega^{\prime}+\kappa_{a} \theta^{4}(x, t),
\end{aligned}
$$

$$
\begin{aligned}
\frac{\partial \theta(x, t)}{\partial t} &-a \Delta \theta(x, t)+\mathbf{v}(x, t) \cdot \nabla \theta(x, t)=\\
&=-b \kappa_{a}\left(\theta^{4}(x, t)-\frac{1}{4 \pi} \int_{S} I^{*}(x, \omega, t) d \omega\right)+\frac{b}{4 \pi c} \frac{\partial}{\partial t} \int_{S} I^{*}(x, \omega, t) d \boldsymbol{\omega},
\end{aligned}
$$

где $a=\frac{k}{\rho c_{p}}, b=\frac{4 \sigma n^{2} T_{\max }^{3}}{\rho c_{p}}$. Подставляя (1.8) в граничные условия (1.5),

и полагая $T_{b}=T_{\max } \theta_{b}$, получим

$$
\begin{aligned}
& \text { гр.у.: } I^{*}(x, \boldsymbol{\omega}, t)=\varepsilon(x) \theta_{b}^{4}(x, t)+\rho^{s}(x) I^{*}\left(x, \omega_{R}, t\right)+ \\
& +\frac{\rho^{d}(x)}{\pi} \int_{\omega^{\prime} \cdot \mathbf{n}>0} I^{*}\left(x, \omega^{\prime}, t\right) \omega^{\prime} \cdot \mathbf{n} d \omega^{\prime}, \omega \cdot \mathbf{n}<0, \\
& \text { гр.у.: } a \frac{\partial \theta(x, t)}{\partial n}+\beta(x)\left(\theta(x, t)-\theta_{b}(x, t)\right)=0 \text {, }
\end{aligned}
$$

где $\beta=\frac{h}{\rho c_{p}}$. Аналогично получаем начальные условия:

$$
I^{*}(x, \boldsymbol{\omega}, 0)=I_{0}^{*}(x, \boldsymbol{\omega}), \quad \theta(x, 0)=\theta_{0}(x),
$$

где $I_{0}^{*}(x, \boldsymbol{\omega})=\left(\frac{\sigma n^{2}}{\pi} T_{\max }^{4}\right)^{-1} I_{0}(x, \boldsymbol{\omega}), \quad \theta_{0}(x)=\frac{T_{0}(x)}{T_{\max }}$.

\section{$1.2$ Диффузионное $P_{1}$ приближение}
$P_{1}$ приближение уравнения переноса излучения является частным случаем метода сферических гармоник $\left(P_{N}\right)$. Идея $P_{N}$ приближений состоит в том, что функцию интенсивности излучения $I(x, \omega)$ раскладывают в ряд Фурье по сферическим гармоникам $\mathcal{Y}_{l}^{m}(\boldsymbol{\omega})$ [67, с. 496]:

$$
I(x, \omega)=\sum_{l=0}^{\infty} \sum_{m=-l}^{l} I_{l}^{m}(x) \mathcal{Y}_{l}^{m}(\omega),
$$

где $I_{l}^{m}(x)$ - коэффициенты, зависящие от $x$. Также в ряд раскладывают фазовую функцию $P\left(\omega, \omega^{\prime}\right)$. Тогда решение уравнения переноса излучения ищется в виде отрезка ряда Фурье для $l \leqslant N$. При подстановке указанной конечной суммы в исходное уравнение интегро-дифференциальное уравнение переноса излучения относительно $I(x, \omega)$ сводится к $(N+1)^{2}$ дифференциальным уравнениям относительно $I_{l}^{m}(x)$.

В $P_{1}$ приближении используется линейное приближение для интенсивности излучения и фазовой функции:

$$
\begin{gathered}
I^{*}(x, \omega, t)=\varphi(x, t)+\boldsymbol{\Phi}(x, t) \cdot \omega, \\
P\left(\omega, \omega^{\prime}\right)=1+A \omega \cdot \omega^{\prime} .
\end{gathered}
$$

Для фазовой функции выполняется условие нормировки:

$$
\frac{1}{4 \pi} \int_{S} P\left(\omega, \omega^{\prime}\right) d \omega=1+\frac{A}{4 \pi} \int_{S} \omega \cdot \omega^{\prime} d \omega=1,
$$

вычисление интеграла см. ниже. Коэффициент $A \in[-1,1]$ описывает анизотропию рассеяния, а величина $A / 3$ имеет смысл среднего косинуса угла рассеяния, поскольку

$$
\frac{1}{4 \pi} \int_{S}\left(\omega \cdot \omega^{\prime}\right) P\left(\omega, \omega^{\prime}\right) d \omega=\frac{1}{4 \pi} \int_{S} \omega \cdot \omega^{\prime} d \omega+\frac{A}{4 \pi} \int_{S}\left(\omega \cdot \omega^{\prime}\right)\left(\omega \cdot \omega^{\prime}\right) d \omega=\frac{A}{3},
$$

вычисление интегралов см. ниже. Случай $A=0$ соответствует изотропному рассеянию. Диапазон допустимых значений величины $A \in[-1,1]$ обусловлен тем, что при $|A|>1$ фазовая функция может принимать отрицательные значения.

Отметим, что если функция $I^{*}$ ищется в виде (1.14), то [67, с. 502]

$$
G(x, t)=\int_{S} I^{*}(x, \omega, t) d \omega=4 \pi \varphi(x, t), \quad \mathbf{q}_{r}(x, t)=\int_{S} I^{*}(x, \boldsymbol{\omega}, t) \boldsymbol{\omega} d \boldsymbol{\omega}=\frac{4 \pi}{3} \boldsymbol{\Phi}(x, t),
$$

поэтому

$$
\varphi(x, t)=\frac{1}{4 \pi} G(x, t), \quad \Phi(x, t)=\frac{3}{4 \pi} \mathbf{q}_{r}(x, t),
$$

где $G$ - аппроксимация пространственной плотности падающего излучения, $\mathbf{q}_{r}-$ аппроксимация плотности потока излучения. Следовательно, функция $\varphi(x, t)$ имеет физический смысл нормализованной интенсивности излучения в точке $x$ в момент времени $t$, усредненной по всем направлениям.

Лемма 1. Справедливы равенства:

$$
\begin{gathered}
\int_{S} 1 \cdot d \omega=4 \pi, \quad \int_{S} \mathbf{a} \cdot \boldsymbol{\omega} d \omega=0, \quad \int_{S}(\mathbf{a} \cdot \boldsymbol{\omega})(\mathbf{b} \cdot \boldsymbol{\omega}) d \omega=\frac{4 \pi}{3} \mathbf{a} \cdot \mathbf{b}, \\
\int_{S} \omega d \omega=0, \quad \int_{S}(\mathbf{a} \cdot \omega) \omega d \omega=\frac{4 \pi}{3} \mathbf{a}, \\
\int_{\boldsymbol{\omega} \cdot \mathbf{a}>0} \mathbf{a} \cdot \boldsymbol{\omega} d \boldsymbol{\omega}=\pi, \quad \int_{\omega \cdot \mathbf{a}>0}(\mathbf{a} \cdot \boldsymbol{\omega})(\mathbf{b} \cdot \boldsymbol{\omega}) d \omega=\frac{2 \pi}{3} \mathbf{a} \cdot \mathbf{b},
\end{gathered}
$$

\begin{center}
\includegraphics[max width=\textwidth]{2022_10_02_2aab7a2952871da15492g-17}
\end{center}

Доказательство. Первое равенство вытекает из определения поверхностного интеграла и представляет собой выражение для площади поверхности единичной сферы.

Для вычисления остальных интегралов воспользуемся формулой перехода от поверхностного интеграла к двойному [72, с. 143, теорема 5.3]:

$$
\int_{S} f(\omega) d \omega=\int_{D} f\left(\omega_{1}(u, v), \omega_{2}(u, v), \omega_{3}(u, v)\right)\left|\boldsymbol{\omega}_{u} \times \boldsymbol{\omega}_{v}\right| d u d v
$$

где $D=\left\{(u, v): 0 \leqslant u \leqslant 2 \pi,-\frac{\pi}{2} \leqslant v \leqslant \frac{\pi}{2}\right\}, \omega_{1}(u, v)=\cos u \cos v, \omega_{2}(u, v)=$ $\sin u \cos v, \omega_{3}(u, v)=\sin v,\left|\omega_{u} \times \omega_{v}\right| d u d v=\cos v d u d v-$ элемент площади поверхности единичной сферы. тогда

Для вычисления второго интеграла положим $f(\boldsymbol{\omega})=\mathbf{a} \cdot \boldsymbol{\omega}=\sum_{i=1}^{3} a_{i} \boldsymbol{\omega}_{i}$,

$$
\begin{aligned}
& \int_{S} \mathbf{a} \cdot \boldsymbol{\omega} d \boldsymbol{\omega}=\int_{-\pi / 2}^{\pi / 2} \int_{0}^{2 \pi}\left(a_{1} \cos u \cos v+a_{2} \sin u \cos v+a_{3} \sin v\right) \cos v d u d v=0 . \\
& \text { получим } \\
& \text { В третьем интеграле положим } f(\boldsymbol{\omega})=(\mathbf{a} \cdot \boldsymbol{\omega})(\mathbf{b} \cdot \boldsymbol{\omega})=\sum_{i, j=1}^{3} a_{i} b_{j} \boldsymbol{\omega}_{i} \boldsymbol{\omega}_{j} \text {, } \\
& \int_{S}(\mathbf{a} \cdot \boldsymbol{\omega})(\mathbf{b} \cdot \boldsymbol{w}) d \boldsymbol{w}= \\
& =\int_{-\pi / 2}^{\pi / 2} \int_{0}^{2 \pi} \mathbf{a}^{T}\left(\begin{array}{ccc}\cos ^{2} u \cos ^{2} v & \sin u \cos u \cos ^{2} v & \cos u \sin v \cos v \\\sin u \cos u \cos ^{2} v & \sin ^{2} u \cos ^{2} v & \sin u \sin v \cos v \\\cos u \sin v \cos v & \sin u \sin v \cos v & \sin ^{2} v\end{array}\right) \mathbf{b} \cos v d u d v= \\
& =\int_{-\pi / 2}^{\pi / 2} \mathbf{a}^{T}\left(\begin{array}{ccc}\pi \cos ^{2} v & 0 & 0 \\0 & \pi \cos ^{2} v & 0 \\0 & 0 & 2 \pi \sin ^{2} v\end{array}\right) \mathbf{b} \cos v d v=\frac{4 \pi}{3} \mathbf{a} \cdot \mathbf{b}
\end{aligned}
$$

здесь $\mathbf{a}, \mathbf{b}$ - векторы-столбцы.

Равенства во второй строке получаются из доказанных равенств:

$$
\begin{gathered}
\int_{S} \omega d \boldsymbol{\omega}=\sum_{i=1}^{3} \mathbf{e}_{i} \int_{S}\left(\boldsymbol{\omega} \cdot \mathbf{e}_{i}\right) d \boldsymbol{\omega}=0, \\
\int_{S}(\mathbf{a} \cdot \boldsymbol{\omega}) \omega d \boldsymbol{\omega}=\sum_{i=1}^{3} \mathbf{e}_{i} \int_{S}(\mathbf{a} \cdot \boldsymbol{\omega})\left(\boldsymbol{\omega} \cdot \mathbf{e}_{i}\right) d \boldsymbol{\omega}=\frac{4 \pi}{3} \sum_{i=1}^{3}\left(\mathbf{a} \cdot \mathbf{e}_{i}\right) \mathbf{e}_{i}=\frac{4 \pi}{3} \mathbf{a} .
\end{gathered}
$$

Для доказательства первого равенства в третьей строке введем систему координат так, чтобы ось $O z$ была сонаправлена с вектором а. Воспользуемся формулой (1.16), в которой вместо $S$ следует взять верхнюю полусферу, $D=$ $\left\{(u, v): 0 \leqslant u \leqslant 2 \pi, 0 \leqslant v \leqslant \frac{\pi}{2}\right\}$. Заметим, что $f(\omega)=\mathbf{a} \cdot \boldsymbol{\omega}=|\mathbf{a}| \sin v$. Таким образом,

$$
\int_{\omega \cdot \mathbf{a}>0} \mathbf{a} \cdot \omega d \omega=|\mathbf{a}| \int_{0}^{\pi / 2} \int_{0}^{2 \pi} \sin v \cos v d u d v=2 \pi \int_{0}^{\pi / 2} \sin v \cos v d v=\pi
$$

Для доказательства второго равенства в третьей строке заметим, что

$$
\begin{gathered}
\int_{S}(\mathbf{a} \cdot \boldsymbol{\omega})(\mathbf{b} \cdot \boldsymbol{\omega}) d \boldsymbol{\omega}=\int_{\boldsymbol{\omega} \cdot \mathbf{a}>0}(\mathbf{a} \cdot \boldsymbol{\omega})(\mathbf{b} \cdot \mathbf{\omega}) d \boldsymbol{\omega}+\int_{\boldsymbol{\omega} \cdot \mathbf{a}<0}(\mathbf{a} \cdot \boldsymbol{\omega})(\mathbf{b} \cdot \boldsymbol{\omega}) d \boldsymbol{\omega}, \\
\int_{\omega \cdot \mathbf{a}>0}(\mathbf{a} \cdot \boldsymbol{\omega})(\mathbf{b} \cdot \boldsymbol{\omega}) d \boldsymbol{\omega}=\int_{\boldsymbol{\omega} \cdot \mathbf{a}<0}(\mathbf{a} \cdot \boldsymbol{\omega})(\mathbf{b} \cdot \boldsymbol{\omega}) d \boldsymbol{\omega},
\end{gathered}
$$

следовательно,

$$
\int_{\boldsymbol{\omega} \cdot \mathbf{a}>0}(\mathbf{a} \cdot \boldsymbol{\omega})(\mathbf{b} \cdot \boldsymbol{\omega}) d \boldsymbol{\omega}=\frac{1}{2} \int_{S}(\mathbf{a} \cdot \boldsymbol{\omega})(\mathbf{b} \cdot \boldsymbol{\omega}) d \boldsymbol{\omega}=\frac{2 \pi}{3} \mathbf{a} \cdot \mathbf{b}
$$

Подставляя $(1.14),(1.15)$ в (1.9), получаем

$$
\begin{gathered}
\frac{1}{c}\left(\frac{\partial \varphi(x, t)}{\partial t}+\boldsymbol{\omega} \cdot \frac{\partial \Phi(x, t)}{\partial t}\right)+\boldsymbol{\omega} \cdot \nabla \varphi(x, t)+\boldsymbol{\omega} \cdot \nabla_{x}(\Phi(x, t) \cdot \boldsymbol{\omega})+\kappa \varphi(x, t)+\mathrm{\kappa} \Phi(x, t) \cdot \boldsymbol{\omega}= \\
=\frac{\kappa_{s}}{4 \pi} \int_{S}\left(1+A \omega \cdot \boldsymbol{\omega}^{\prime}\right)\left(\varphi(x, t)+\boldsymbol{\Phi}(x, t) \cdot \boldsymbol{\omega}^{\prime}\right) d \boldsymbol{\omega}^{\prime}+\mathrm{\kappa}_{a} \theta^{4}(x, t) .
\end{gathered}
$$

С учетом равенств

$$
\int_{S} \Phi(x, t) \cdot \omega^{\prime} d \omega^{\prime}=0, \quad \int_{S} \omega \cdot \omega^{\prime} d \omega^{\prime}=0, \quad \int_{S}\left(\Phi(x, t) \cdot \omega^{\prime}\right)\left(\omega \cdot \omega^{\prime}\right) d \omega^{\prime}=\frac{4 \pi}{3} \Phi(x, t) \cdot \omega
$$

имеем

$$
\begin{gathered}
\frac{1}{c}\left(\frac{\partial \varphi(x, t)}{\partial t}+\omega \cdot \frac{\partial \Phi(x, t)}{\partial t}\right)+\boldsymbol{\omega} \cdot \nabla \varphi(x, t)+\boldsymbol{\omega} \cdot \nabla_{x}(\boldsymbol{\Phi}(x, t) \cdot \boldsymbol{\omega})+\boldsymbol{\kappa} \varphi(x, t)+\mathrm{\kappa} \boldsymbol{\Phi}(x, t) \cdot \boldsymbol{\omega}= \\
=\mathrm{k}_{s}\left(\varphi(x, t)+\frac{A}{3} \boldsymbol{\Phi}(x, t) \cdot \boldsymbol{\omega}\right)+\mathrm{\kappa}_{a} \theta^{4}(x, t),
\end{gathered}
$$

или

$$
\frac{1}{c}\left(\frac{\partial \varphi(x, t)}{\partial t}+\omega \cdot \frac{\partial \Phi(x, t)}{\partial t}\right)+\omega \cdot \nabla \varphi(x, t)+\omega \cdot \nabla_{x}(\Phi(x, t) \cdot \omega)+
$$

$$
+\mathrm{\kappa}_{a} \varphi(x, t)+\left(\mathrm{\kappa}_{a}+\mathrm{\kappa}_{s}^{\prime}\right) \Phi(x, t) \cdot \omega=\mathrm{\kappa}_{a} \theta^{4}(x, t),
$$

где $\kappa_{s}^{\prime}=\kappa_{s}(1-A / 3)$ - приведенный коэффициент рассеяния.

Проинтегрируем уравнение (1.17) по $\omega \in S$. Получим

$$
\frac{1}{c} \frac{\partial \varphi(x, t)}{\partial t}+\frac{1}{3} \div \Phi(x, t)+\mathrm{\kappa}_{a} \varphi(x, t)=\mathrm{\kappa}_{a} \theta^{4}(x, t),
$$

так как

$$
\begin{gathered}
\int_{S} \omega \cdot \nabla_{x}(\Phi(x, t) \cdot \boldsymbol{\omega}) d \boldsymbol{\omega}=\sum_{i=1}^{3} \int_{S}\left(\boldsymbol{\omega} \cdot \mathbf{e}_{i}\right)\left(\omega \cdot \frac{\partial \Phi(x, t)}{\partial x_{i}}\right) d \boldsymbol{\omega}= \\
=\frac{4 \pi}{3} \sum_{i=1}^{3} \frac{\partial \Phi(x, t)}{\partial x_{i}} \cdot \mathbf{e}_{i}=\frac{4 \pi}{3} \sum_{i=1}^{3} \frac{\partial \Phi_{i}(x, t)}{\partial x_{i}}=\frac{4 \pi}{3} \div \boldsymbol{\Phi}(x, t) .
\end{gathered}
$$

$\mathrm{Y}_{\text {множим уравнение }}(1.17)$ на $\omega:$

$$
\begin{aligned}
\frac{1}{c} \frac{\partial \varphi(x, t)}{\partial t} \omega+& \frac{1}{c}\left(\omega \cdot \frac{\partial \Phi(x, t)}{\partial t}\right) \omega+(\omega \cdot \nabla \varphi(x, t)) \omega+\left(\omega \cdot \nabla_{x}(\Phi(x, t) \cdot \omega)\right) \omega+\\
&+\kappa_{a} \varphi(x, t) \omega+\left(\kappa_{a}+\kappa_{s}^{\prime}\right)(\Phi(x, t) \cdot \omega) \omega=\kappa_{a} \theta^{4}(x, t) \omega
\end{aligned}
$$

и проинтегрируем полученное равенство по $\omega \in S$. Для вычисления четвертого слагаемого представим интеграл по единичной сфере $S$ как сумму интегралов по верхней $S_{1}$ и нижней $S_{2}$ полусферам и воспользуемся тем, что

$$
\int_{S_{2}}\left(\omega \cdot \nabla_{x}(\Phi(x, t) \cdot \omega)\right) \omega d \omega=-\int_{S_{1}}\left(\omega \cdot \nabla_{x}(\Phi(x, t) \cdot \omega)\right) \omega d \omega,
$$

следовательно, интеграл равен 0. Таким образом,

$$
\frac{1}{c} \frac{\partial \Phi(x, t)}{\partial t}+\left(\kappa_{a}+\kappa_{s}^{\prime}\right) \Phi(x, t)+\nabla \varphi(x, t)=0 .
$$

Итак, уравнения (1.18), (1.19) представляют собой $P_{1}$ приближение для уравнения переноса излучения. Дальнейшие преобразования основываются на предположении, что выполняется закон Фика:

$$
\Phi(x, t)=-3 \alpha \nabla \varphi(x, t),
$$

где $\alpha=\frac{1}{3\left(\kappa_{a}+\kappa_{s}^{\prime}\right)}=\frac{1}{3 \kappa-A \kappa_{s}}$. Фактически мы пренебрегаем производной $\frac{\partial \Phi}{\partial t}$ в уравнении (1.19). Подставив (1.20) в (1.18), получим

$$
\frac{1}{c} \frac{\partial \varphi(x, t)}{\partial t}-\alpha \Delta \varphi(x, t)+\kappa_{a}\left(\varphi(x, t)-\theta^{4}(x, t)\right)=0 .
$$

Чтобы получить уравнение для температуры, подставим (1.14) в (1.10). Получим

$$
\frac{\partial \theta(x, t)}{\partial t}-a \Delta \theta(x, t)+\mathbf{v}(x, t) \cdot \nabla \theta(x, t)+b \kappa_{a}\left(\theta^{4}(x, t)-\varphi(x, t)\right)=\frac{b}{c} \frac{\partial \varphi(x, t)}{\partial t} .
$$

Учитывая (1.21), уравнение (1.22) можно записать в виде с кросс-диффузией:

$$
\frac{\partial \theta(x, t)}{\partial t}-a \Delta \theta(x, t)+\mathbf{v}(x, t) \cdot \nabla \theta(x, t)=b \alpha \Delta \varphi(x, t) .
$$

В дальнейшем вместо уравнения (1.22) будем использовать уравнение с нулевой правой частью (см., например, [73])

$$
\frac{\partial \theta(x, t)}{\partial t}-a \Delta \theta(x, t)+\mathbf{v}(x, t) \cdot \nabla \theta(x, t)+b \kappa_{a}\left(\theta^{4}(x, t)-\varphi(x, t)\right)=0 .
$$

Далее выведем граничные условия типа Маршака для $P_{1}$ приближения (см. [74]). Для этого подставим (1.14) в граничное условие (1.11):

$$
\begin{gathered}
\varphi(x, t)+\boldsymbol{\Phi}(x, t) \cdot \boldsymbol{\omega}=\varepsilon(x) \theta_{b}^{4}(x, t)+\rho^{s}(x)\left(\varphi(x, t)+\boldsymbol{\Phi}(x, t) \cdot \boldsymbol{\omega}_{R}\right)+ \\
+\frac{\rho^{d}(x)}{\pi} \int_{\omega^{\prime} \cdot \mathbf{n}>0}\left(\varphi(x, t)+\boldsymbol{\Phi}(x, t) \cdot \boldsymbol{\omega}^{\prime}\right) \omega^{\prime} \cdot \mathbf{n} d \omega^{\prime}, \quad \boldsymbol{\omega} \cdot \mathbf{n}<0, \quad \boldsymbol{\omega}_{R}=\boldsymbol{\omega}-2(\boldsymbol{\omega} \cdot \mathbf{n}) \mathbf{n} .
\end{gathered}
$$

Для вычисления интеграла применим лемму 1 :

$$
\begin{aligned}
\varphi(x, t)+\boldsymbol{\Phi}(x, t) \cdot \boldsymbol{\omega} &=\varepsilon(x) \theta_{b}^{4}(x, t)+\rho^{s}(x)[\varphi(x, t)+\boldsymbol{\Phi}(x, t) \cdot \boldsymbol{\omega}-2(\boldsymbol{\omega} \cdot \mathbf{n})(\boldsymbol{\Phi}(x, t) \cdot \mathbf{n})]+\\
&+\rho^{d}(x)\left(\varphi(x, t)+\frac{2}{3} \boldsymbol{\Phi}(x, t) \cdot \mathbf{n}\right), \quad \boldsymbol{\omega} \cdot \mathbf{n}<0 .
\end{aligned}
$$

Умножим данное равенство на $\boldsymbol{\omega} \cdot \mathbf{n}$ и проинтегрируем по множеству входящих направлений, для которых $\boldsymbol{\omega} \cdot \mathbf{n}<0$. Получим

$$
\begin{gathered}
-\pi \varphi(x, t)+\frac{2 \pi}{3} \boldsymbol{\Phi}(x, t) \cdot \mathbf{n}=-\pi \varepsilon(x) \theta_{b}^{4}(x, t)-\pi \rho^{s}(x) \varphi(x, t)+\frac{2 \pi \rho^{s}(x)}{3} \boldsymbol{\Phi}(x, t) \cdot \mathbf{n}- \\
-\frac{4 \pi \rho^{s}(x)}{3} \boldsymbol{\Phi}(x, t) \cdot \mathbf{n}-\pi \rho^{d}(x)\left(\varphi(x, t)+\frac{2}{3} \boldsymbol{\Phi}(x, t) \cdot \mathbf{n}\right),
\end{gathered}
$$

ИЛИ

$$
\varepsilon(x) \varphi(x, t)=\varepsilon(x) \theta_{b}^{4}(x, t)+\frac{2(2-\varepsilon(x))}{3} \boldsymbol{\Phi}(x, t) \cdot \mathbf{n} .
$$

Воспользуемся равенством (1.20), будем иметь

$$
\alpha \frac{\partial \varphi(x, t)}{\partial n}+\gamma(x)\left(\varphi(x, t)-\theta_{b}^{4}(x, t)\right)=0,
$$

где $\gamma=\frac{\varepsilon}{2(2-\varepsilon)}$. Отметим, что на участках втекания и вытекания среды можно принять $\gamma=1 / 2[75]$.

Дополним полученные соотношения граничным условием для температуры (1.12):

$$
a \frac{\partial \theta(x, t)}{\partial n}+\beta(x)\left(\theta(x, t)-\theta_{b}(x, t)\right)=0
$$

и начальными условиями

$$
\theta(x, 0)=\theta_{0}(x), \quad \varphi(x, 0)=\varphi_{0}(x) .
$$

Соотношения (1.21), (1.23), (1.24)-(1.26) образуют диффузионную модель сложного теплообмена.

Укажем возможные пути обоснования закона Фика (1.20). В [76, с. 136], [77, с. 222$]$, [78, с. 96] указано, что в уравнении (1.19) можно пренебречь слагаемым $\frac{1}{c} \frac{\partial \Phi}{\partial t}$, если

$$
\frac{1}{|\Phi|} \frac{\partial|\Phi|}{\partial t} \ll c\left(\kappa_{a}+\kappa_{s}^{\prime}\right) .
$$

Это предположение означает, что относительное изменение плотности потока излучения во времени много меньше частоты столкновений фотонов, так как величина $\frac{1}{\mathrm{~K}_{a}+\mathrm{K}_{s}^{\prime}}$ есть средняя длина свободного пробега (transport mean free path) $[78]$

В диффузионном приближении предполагается, что среда имеет большое альбедо $\left(\kappa_{a} \ll \kappa_{s}\right)$ и излучение почти изотропно [78, с. 88]. В [78, с. 97] указано, что предположения о почти изотропности излучения (направленное расширение) и о малом относительном изменении плотности потока излучения (временное расширение потока фотонов по отношению к среднему времени свободного пробега) выполняются при большом числе рассеяний фотонов, так что оба приближения можно свести к предположению $\kappa_{s}^{\prime} \gg \kappa_{a}$. Кроме того, необходимо, чтобы точка наблюдения находилась достаточно далеко от источников и от границ.

В [79, 80] делается предположение $3 \omega_{0} \alpha \ll c$, где $\omega_{0}$ - частота синусоидально модулированного источника. Авторы $[79,81]$ сначала выводят из (1.18), (1.19) уравнение второго порядка по времени, а затем отбрасывают некоторые слагаемые, которые можно считать малыми в силу указанного предположения. Применив к уравнению (1.18) операцию дифференцирования по $t$, а к уравнению (1.19) операцию дивергенции, получим

$$
\begin{gathered}
\frac{1}{c} \frac{\partial^{2} \varphi}{\partial t^{2}}+\frac{1}{3} \frac{\partial}{\partial t} \div \Phi+\kappa_{a} \frac{\partial \varphi}{\partial t}=\kappa_{a} \frac{\partial\left(\theta^{4}\right)}{\partial t} \\
\frac{1}{c} \frac{\partial}{\partial t} \div \Phi+\left(\kappa_{a}+\kappa_{s}^{\prime}\right) \div \Phi+\Delta \varphi=0
\end{gathered}
$$

$У_{\text {множим второе уравнение на } c / 3}$ и вычтем из первого уравнения. Учитывая (1.18), будем иметь

$$
\frac{1}{c} \frac{\partial \varphi}{\partial t}-\alpha \Delta \varphi+\kappa_{a}\left(\varphi-\theta^{4}\right)+\frac{3 \alpha \kappa_{a}}{c} \frac{\partial \varphi}{\partial t}+\frac{3 \alpha}{c^{2}} \frac{\partial^{2} \varphi}{\partial t^{2}}=\frac{3 \alpha \kappa_{a}}{c} \frac{\partial\left(\theta^{4}\right)}{\partial t} .
$$

Подчеркнутые слагаемые отбрасываем, принимая во внимание, что $3 \alpha \kappa_{a}=\frac{\kappa_{a}}{\kappa_{a}+\kappa_{s}^{\prime}} \ll 1$. Более точные оценки с переходом в частотную область указаны в $[79,80]$. Однако их применение для нашей задачи требует дополнительных оценок правой части (1.27), содержащей $\theta$.

Также автор [67, с. 509] отмечает, что $P_{1}$ приближение может давать ошибочный результат в оптически тонкой среде со слишком анизотропным распределением интенсивности, в частности, в многомерных областях с длинными узкими конфигурациями и/или когда излучение поверхности преобладает над излучением среды. Среда называется оптически толстой, если средняя длина свободного пробега фотона мала по сравнению с ее характерным размером [68, с. 343]. Авторы $[82$, с. 228] также указывают, что для применения диффузионного $P_{1}$ приближения альбедо $\mathrm{k}_{s} / \mathrm{k}$ должно быть близко к единице и среда должна быть оптически толстой. В [79, с. 8] указано, что фазовая функция не должна быть слишком анизотропной $(|A / 3|$ не слишком близко к 1$)$.

Таким образом, благоприятными условиями для применения $P_{1}$ приближения являются: 1) $\kappa_{a} \ll \kappa_{s}^{\prime} ;$ ) оптически толстая среда; 3) удаление от границ области.

\section{Глава 2. Теоретический анализ обратных задач сложного теплообмена}
\section{$2.1$ Граничная обратная задача для уравнений сложного теплообмена}
\subsection{1 Постановка обратной задачи}
Нормализованная стационарная модель, описывающая процесс радиационного теплопереноса в области $\Omega \subset \mathbb{R}^{3}$ с липшицевой границей $\Gamma$ (см [83].), имеет следующий вид:

$$
\begin{aligned}
&-a \Delta \theta+b \kappa_{a}\left(\theta^{3}|\theta|-\varphi\right)=0, \\
&-\alpha \Delta \varphi+\kappa_{a}\left(\varphi-\theta^{3}|\theta|\right)=0 .
\end{aligned}
$$

Здесь $\theta$ - нормализованная температура, $\varphi$ - нормализованная интенсивность излучения, усреднённая по всем направлениям, к $a-$ коэффициент поглощения. Константы $a, b, \alpha, \gamma, \beta$ описываются следующим образом:

$$
a=\frac{k}{\rho c_{v}}, b=\frac{4 \sigma n^{2} T_{\max }^{3}}{\rho c_{v}}, \alpha=\frac{1}{3 \kappa-A \kappa_{s}}
$$

где $k$ - теплопроводность, $c_{v}-$ удельная теплоёмкость, $\rho-$ плотность, $\sigma-$ постоянная Стефана- Больцмана, $n$ - индекс рефракции, $T_{\max }-$ максимальная температура, $\mathrm{k}:=\mathrm{k}_{s}+\mathrm{k}_{a}-$ коэффициент полного взаимодействия, $\mathrm{k}_{s}-$ коэффициент рассеяния. Коэффициент $A \in[-1,1]$ описывает анизотропию рассеивания; случай $A=0$ отвечает изотропному рассеиванию.

Уравнения (2.1) дополняются граничными условиями на $\Gamma:=\partial \Omega=\bar{\Gamma}_{0} \cup$ $\bar{\Gamma}_{1} \cup \bar{\Gamma}_{2}$, где части границы $\Gamma_{0}, \Gamma_{1}, \Gamma_{2}$ не имеют пересечений.

$$
\begin{aligned}
\Gamma: & a \partial_{n} \theta+\beta\left(\theta-\theta_{b}\right)=0, \\
\Gamma_{0} \cup \Gamma_{2}: & \alpha \partial_{n} \varphi+\gamma\left(\varphi-\theta_{b}^{4}\right)=0, \\
\Gamma_{1}: & \alpha \partial_{n} \varphi+u\left(\varphi-\theta_{b}^{4}\right)=0 .
\end{aligned}
$$

Функции $\gamma, \theta_{b}, \beta$ - являются известными. Функция $u$ характеризует отражающие свойства участка границы $\Gamma_{1}$. Предполагается, что

$$
0<u_{1} \leqslant u \leqslant u_{2},
$$

где $u_{1}$ и $u_{2}$ - заданные ограниченные функции.

Обратная задача состоит в нахождении функций $u(x), x \in \Gamma_{1}, \theta(x), \varphi(x), x \in$ $\Omega$ удовлетворяющих условиям (2.1)-(2.3), а также дополнительному условию на участке границы $\Gamma_{2}$ :

$$
\left.\theta\right|_{\Gamma_{2}}=\theta_{0}
$$

где $\theta_{0}$ известная функция. Сформулированная обратная задача (2.1)-(2.4) сводится к экстремальной задаче, состоящей в минимизации функционала

$$
J(\theta)=\frac{1}{2} \int_{\Gamma_{2}}\left(\theta-\theta_{0}\right)^{2} d \Gamma
$$

на решениях краевой задачи (2.1)-(2.3). Решение задачи (2.1)-(2.3), (2.5) называется квазирешением задачи $(2.1)-(2.4)$

\subsection{2 Формализация задачи нахождения квазирешения}
Будем предполагать что исходные данные удовлетворяют следующему условию:

(i) $\beta \in L^{\infty}(\Gamma) ; \gamma \in L^{\infty}\left(\Gamma_{0} \cup \Gamma_{2}\right) ; u_{1}, u_{2} \in L^{\infty}\left(\Gamma_{1}\right) ; 0<\beta_{0} \leqslant \beta ; 0<\gamma_{0} \leqslant$ $\gamma ; \beta_{0}, \gamma_{0}=$ Const $0 \leqslant u_{1} \leqslant u_{2}$

Пусть $H=L^{2}(\Omega), V=W_{2}^{1}(\Omega), Y=V \times V$. Пространство $H$ отождествляем с сопряжённым пространством $H^{\prime}$ так, что $V \subset H=H^{\prime} \subset V^{\prime}$. Определим $(f, v)$ как значение функционала $f \in V^{\prime}$ на элементе $v \in V$, совпадающее со скалярным произведением в $H$, если $f \in H,\|f\|^{2}=(f, f)$. Пространство $U=L^{2}\left(\Gamma_{1}\right)$ является пространством управлений; $U_{a d}=\left\{u \in U, u_{1} \leqslant u \leqslant u_{2}\right\}$ - множество допустимых управлений.

Пусть $v$ произвольный элемент множества $H^{1}(\Omega)$. Определим операторы:

$$
A_{1,2}: V \rightarrow V^{\prime}, \quad F: V \times U \rightarrow V^{\prime}, f \in V^{\prime}, g \in V^{\prime} .
$$

$$
\begin{gathered}
\left(A_{1} \theta, v\right)=a(\nabla \theta, \nabla v)+\int_{\Gamma} \beta \theta v d \Gamma,\left(A_{2} \varphi, v\right)=\alpha(\nabla \varphi, \nabla v)+\int_{\Gamma_{0} \cup \Gamma_{2}} \gamma \varphi v d \Gamma, \\
(f, v)=\int_{\Gamma} \beta \theta_{b} v d \Gamma, \quad(g, v)=\int_{\Gamma_{0} \cup \Gamma_{2}} \gamma \theta_{b}^{4} v d \Gamma, \\
(F(\varphi, u), v)=\int_{\Gamma_{1}} u\left(\varphi-\theta_{b}^{4}\right) v d \Gamma .
\end{gathered}
$$

Пару $\{\theta, \varphi\} \in Y$ будем называть слабым решением задачи $(2.1),(2.2)$, если

$$
A_{1} \theta+b \kappa_{a}\left(|\theta| \theta^{3}-\varphi\right)=f, A_{2} \varphi+\kappa_{a}\left(\varphi-|\theta| \theta^{3}\right)+F(\varphi, u)=g .
$$

Задача нахождения квазирешения состоит в минимизации функционала $J(\theta)$, определённом на компоненте $\theta$ решения системы (2.6). Таким образом

$J(\theta) \rightarrow \inf ,\{\theta, \varphi\}$ решение (2.6), соответствующее функции $u \in U_{a d}$.

Пара $\{\hat{\theta}, \hat{\varphi}\}$ соответствующая минимуму $J$, отвечающая функции $\hat{u}$ называется оптимальным состоянием. В таком случае $\hat{u}$ называется квазирешением обратной задачи $(2.1)-(2.4)$.

\subsection{3 Анализ экстремальной задачи}
Для доказательства разрешимости задачи (2.7) нам необходимо также установить некоторые свойства решения задачи $(2.1),(2.2)$.

Лемма 2 ([84]). Пусть выполняется условие (i). Тогда для каждого $u \in U_{a d}$ существует единственное слабое решение $\{\theta, \varphi\}$ для задачи (2.1),(2.2) и справедливы оценки:

$$
\begin{gathered}
M_{1} \leqslant \theta \leqslant M_{2}, M_{1}^{4} \leqslant \varphi \leqslant M_{2}^{4}, \\
\|\nabla \varphi\|^{2} \leqslant C .
\end{gathered}
$$

Здесь $M_{1}=$ ess inf $\theta_{b}, M_{2}=$ ess sup $\theta_{b}$, и константа $C>0$ зависит только om $a, b, \alpha, \kappa_{a}, \beta, \gamma,\|u\|_{L^{\infty}(\Gamma)}$ и области $\Omega$.

На основе оценок (2.8) и (2.9) аналогично [85] доказывается разрешимость экстремальной задачи $(2.7)$. Теорема 1. Пусть выполняется условие (i). Тогда существует хотя бы одно решение задачи (2.7).

Для вывода системы оптимальности, покажем дифференцируемость функционала $J$.

Лемма 3. Функиионал $J: V \rightarrow \mathbb{R}$ дифферениируем по Фреше.

Доказательство. Покажем, что для произвольной функции $\theta \in V$ выполняется следующее равенство:

$$
J(\theta+h)=J(\theta)+J^{\prime}(\theta)\langle h\rangle+r(\theta, h) \forall h \in V, \quad \text { где } J^{\prime}(\theta)\langle h\rangle=\int_{\Gamma_{2}}\left(\theta-\theta_{0}\right) h d \Gamma,
$$

где для остаточного члена $r(\theta, h)$ справедливо соотношение:

$$
\frac{|r(\theta, h)|}{\|h\|_{V}} \rightarrow 0 \quad \text { при } \quad\|h\|_{V} \rightarrow 0 .
$$

Перепишем (2.10) в виде

$$
\frac{1}{2}\left\|\theta+h-\theta_{0}\right\|_{L^{2}\left(\Gamma_{2}\right)}^{2}=\frac{1}{2}\left\|\theta-\theta_{0}\right\|_{L^{2}\left(\Gamma_{2}\right)}^{2}+\left(\theta-\theta_{0}, h\right)_{L^{2}\left(\Gamma_{2}\right)}+\frac{1}{2}\|h\|_{L^{2}\left(\Gamma_{2}\right)}^{2} .
$$

Согласно теореме о следах $\|h\|_{L^{2}\left(\Gamma_{2}\right)} \leqslant C\|h\|_{V}$, где $C$ не зависит от $h$. Поэтому

$$
\frac{r(\theta, h)}{\|h\|_{V}} \leqslant \frac{1}{2} C^{2}\|h\|_{V} \rightarrow 0 \quad \text { при }\|h\|_{V} \rightarrow 0 .
$$

Вывод условий оптимальности основан на принципе множителей Лагранжа для гладко-выпуклых задач минимизации.

Теорема 2. Пусть $\hat{y}=\{\hat{\theta}, \hat{\varphi}\} \in Y, \hat{u} \in U_{a d}-$ решение экстремальной задачи (2.7). Тогда существует пара $p=\left(p_{1}, p_{2}\right), p \in Y$ такая, что тройка $(\hat{y}, \hat{u}, p)$, удовлетворяет следующим условиям:

$$
\begin{gathered}
A_{1} p_{1}+4|\hat{\theta}|{ }^{3} \kappa_{a}\left(b p_{1}-p_{2}\right)=f_{c}, \quad\left(f_{c}, v\right)=-\int_{\Gamma_{2}}\left(\hat{\theta}-\theta_{0}\right) v d \Gamma, \\
A_{2} p_{2}+\kappa_{a}\left(p_{2}-b p_{1}\right)=g_{c}\left(\left(p_{2}, \hat{u}\right), v\right), g_{c}\left(\left(p_{2}, \hat{u}\right), v\right)=-\int_{\Gamma_{1}} \hat{u} p_{2} v \Gamma, \\
\int_{\Gamma_{1}} p_{2}\left(\hat{\varphi}-\theta_{b}^{4}\right)(u-w) \leqslant 0 \quad \forall w \in U_{a d} .
\end{gathered}
$$

Доказательство. Перепишем уравнения (2.6) следующим образом:

$$
H(y, u)=0, y=\{\theta, \varphi\} \in Y,
$$

где

$$
\begin{gathered}
H: Y \times U \rightarrow Y^{\prime}, \\
H(y, u)=\left\{A_{1} \theta+b \kappa_{a}\left(|\theta| \theta^{3}-\varphi\right)-f, A_{2} \varphi+\kappa_{a}\left(\varphi-|\theta| \theta^{3}\right)+F(\varphi, u)-g\right\} .
\end{gathered}
$$

Заметим, что для всех $u \in U_{a d}$, отображение $y \rightarrow J(\theta)$ и $y \rightarrow H(y, u)$ непрерывно дифференцируемо в окрестности $\mathcal{O}(\hat{y})$ точки $\hat{y}$. Непрерывная дифференцируемость членов в $H$ следует из непрерывной дифференцируемости функции $t \in \mathbb{R} \rightarrow|t| t^{3}$, а также из непрерывности вложения $V \subset L^{6}(\Omega)$. В дополнение, отображение $u \rightarrow H(y, u)$ непрерывно из $U \rightarrow Y^{\prime}$ и афинно. B [85] показано, что $\operatorname{Im} H_{y}^{\prime}(\hat{y}, \hat{u})=Y$, что влечёт невырожденность условий оптимальности.

Рассмотрим функцию Лагранжа $L(y, u, p)=J(\theta)+(H(y, u), p)$, где $y, p \in$ $Y, u \in U_{a d}$. Согласно принципу Лагранжа [86, Гл.2, Теорема 1.5] существует пара $p=\left\{p_{1}, p_{2}\right\} \in Y$ такая, что

$$
\begin{gathered}
\left(L_{\theta}, \zeta\right)=\int_{\Gamma_{2}}\left(\hat{\theta}-\theta_{0}\right) \zeta d \Gamma+\left(A_{1} \zeta+4 b \kappa_{a}|\hat{\theta}|^{3} \zeta, p_{1}\right)-4 \kappa_{a}\left(|\hat{\theta}|^{3} \zeta, p_{2}\right)=0 \forall \zeta \in V, \\
\left(L_{\varphi}, \zeta\right)=\left(A_{2} \zeta+\kappa_{a} \zeta, p_{2}\right)-b \kappa_{a}\left(\zeta, p_{1}\right)+\int_{\Gamma_{1}} \hat{u} \zeta p_{2}=0 \forall \zeta \in V, \\
\left(L_{u}, \tau\right)=\int_{\Gamma_{1}} \tau\left(\varphi-\theta_{b}^{4}\right) p_{2} d \Gamma \leqslant 0, \tau:=\hat{u}-w \forall w \in U_{a d} .
\end{gathered}
$$

Сопряжённые уравнения $(2.12),(2.13)$ являются прямым следствием вариационных равенств $(2.15)$ и $(2.16)$.

\section{$2.2$ Оптимизационный метод для задачи радиационного теплообмена}
Стационарный радиационный и диффузионный теплообмен в ограниченной области $\Omega \subset \mathbb{R}^{3}$ с границей $\Gamma=\partial \Omega$ моделируется в рамках $P_{1}-$ приближения для уравнения переноса излучения следующей краевой задачей $[87,88]$ :

$$
-a \Delta \theta+b \kappa_{a}\left(|\theta| \theta^{3}-\varphi\right)=0, \quad-\alpha \Delta \varphi+\kappa_{a}\left(\varphi-|\theta| \theta^{3}\right)=0, x \in \Omega .
$$

$$
a\left(\partial_{n} \theta+\theta\right)=r, \quad \alpha\left(\partial_{n} \varphi+\varphi\right)=u \text { на } \Gamma .
$$

Здесь $\theta$ - нормализованная температура, $\varphi$ - нормализованная интенсивность излучения, усредненная по всем направлениям. Положительные физические параметры $a, b, \kappa_{a}$ и $\alpha$, описывающие свойства среды, определяются стандартным образом [88]. Функция $r(x), x \in \Gamma$ является заданной, а неизвестная функция $u(x), x \in \Gamma$ играет роль управления. Через $\partial_{n}$ обозначаем производную в направлении внешней нормали $\mathbf{n}$.

Экстремальная задача заключается в отыскании тройки $\left\{\theta_{\lambda}, \varphi_{\lambda}, u_{\lambda}\right\}$ такой, что

$$
J_{\lambda}(\theta, u)=\frac{1}{2} \int_{\Gamma}\left(\theta-\theta_{b}\right)^{2} d \Gamma+\frac{\lambda}{2} \int_{\Gamma} u^{2} d \Gamma \rightarrow \inf
$$

на решениях краевой задачи $(2.18),(2.19)$. Функция $\theta_{b}(x), x \in \Gamma$ и параметр регуляризации $\lambda>0$ заданы.

Задача оптимального управления (2.18)-(2.20), если $r:=a\left(\theta_{b}+q_{b}\right)$, где $q_{b}$ - заданная на $Г$ функция, является при малых $\lambda$ аппроксимацией краевой задачи для уравнений $(2.18)$, в которой неизвестны краевые условия для интенсивности излучения $\varphi$, а ставятся граничные условия для температуры и тепловых потоков на границе,

$$
\left.\theta\right|_{\Gamma}=\theta_{b},\left.\quad \partial_{n} \theta\right|_{\Gamma}=q_{b} .
$$

\subsection{1 Формализация задачи управления}
В дальнейшем считаем, что $\Omega \subset \mathbb{R}^{3}$ - ограниченная строго липшицева область, граница $\Gamma$ которой состоит из конечного числа гладких кусков. Через $L^{p}, 1 \leqslant p \leqslant \infty$ обозначаем пространство Лебега, а через $H^{s}$ - пространство Соболева $W_{2}^{s}$. Пусть $H=L^{2}(\Omega), V=H^{1}(\Omega)$, через $V^{\prime}$ обозначаем пространство, сопряженное с пространством $V$. Пространство $H$ отождествляем с пространством $H^{\prime}$, так что $V \subset H=H^{\prime} \subset V^{\prime}$. Обозначим через $\|\cdot\|$ стандартную норму в $H$, а через $(f, v)-$ значение функционала $f \in V^{\prime}$ на элементе $v \in V$, совпадающее со скалярным произведением в $H$, если $f \in H$. Через $U$ обозначаем пространство $L^{2}(\Gamma)$ с нормой $\|u\|_{\Gamma}=\left(\int_{\Gamma} u^{2} d \Gamma\right)^{1 / 2}$. Будем предполагать, что

(i) $a, b, \alpha, \mathrm{k}_{a}, \lambda=$ Const $>0$,

(ii) $\theta_{b}, q_{b} \in U, \quad r=a\left(\theta_{b}+q_{b}\right)$.

Определим операторы $A: V \rightarrow V^{\prime}, B: U \rightarrow V^{\prime}$, используя следующие равенства, справедливые для любых $y, z \in V, w \in U$ :

$$
(A y, z)=(\nabla y, \nabla z)+\int_{\Gamma} y z d \Gamma, \quad(B w, z)=\int_{\Gamma} w z d \Gamma .
$$

Билинейная форма $(A y, z)$ определяет скалярное произведение в пространстве $V$, а соответствующая норма $\|z\|_{V}=\sqrt{(A z, z)}$ эквивалентна стандартной норме $V$. Поэтому определен непрерывный обратный оператор $A^{-1}: V^{\prime} \mapsto V$. Отметим справедливость, для любых $v \in V, w \in U, g \in V^{\prime}$, следующих неравенств

$$
\|v\|^{2} \leqslant C_{0}\|v\|_{V}^{2},\|v\|_{V^{\prime}} \leqslant C_{0}\|v\|_{V},\|B w\|_{V^{\prime}} \leqslant\|w\|_{\Gamma},\left\|A^{-1} g\right\|_{V} \leqslant\|g\|_{V^{\prime}} .
$$

Здесь постоянная $C_{0}>0$ зависит только от области $\Omega$.

Далее используем следующее обозначение $[h]^{s}:=|h|^{s} \operatorname{sign} h, s>0, h \in \mathbb{R}$ для монотонной степенной функции.

Определение. Пара $\theta, \varphi \in V$ называется слабым решением задачи $(2.18),(2.19)$, если

$$
a A \theta+b \kappa_{a}\left([\theta]^{4}-\varphi\right)=B r, \quad \alpha A \varphi+\kappa_{a}\left(\varphi-[\theta]^{4}\right)=B u .
$$

Для формулировки задачи оптимального управления определим оператор ограничений $F(\theta, \varphi, u): V \times V \times U \rightarrow V^{\prime} \times V^{\prime}$,

$$
F(\theta, \varphi, u)=\left\{a A \theta+b \kappa_{a}\left([\theta]^{4}-\varphi\right)-B r, \alpha A \varphi+\kappa_{a}\left(\varphi-[\theta]^{4}\right)-B u\right\} .
$$

Задача $(C P)$. Найти тройку $\{\theta, \varphi, u\} \in V \times V \times U$ такую, что

$$
J_{\lambda}(\theta, u) \equiv \frac{1}{2}\left\|\theta-\theta_{b}\right\|_{\Gamma}^{2}+\frac{\lambda}{2}\|u\|_{\Gamma}^{2} \rightarrow \inf , \quad F(\theta, \varphi, u)=0 \text {. }
$$

\section{$2.2 .2$ Разрешимость задачи $(C P)$}
Докажем предварительно однозначную разрешимость краевой задачи $(2.18),(2.19)$. Лемма 1. Пусть выполняются условия (і), (ii), $u \in U$. Тогда существует единственное слабое решение задачи $(2.18),(2.19)$ и при этом

$$
\begin{array}{r}
a\|\theta\|_{V} \leqslant\|r\|_{\Gamma}+\frac{C_{0} \mathrm{~K}_{a}}{\alpha}\|r+b u\|_{\Gamma}, \\
\alpha b\|\varphi\|_{V} \leqslant\|r\|_{\Gamma}+\left(\frac{C_{0} \mathrm{~K}_{a}}{\alpha}+1\right)\|r+b u\|_{\Gamma} .
\end{array}
$$

Доказательство. Если второе уравнение в (2.23) умножить на $b$ и сложить с первым, то получим равенства

$$
A(a \theta+\alpha b \varphi)=B(r+b u), a \theta+\alpha b \varphi=A^{-1} B(r+b u), \varphi=\frac{1}{\alpha b}\left(A^{-1} B(r+b u)-a \theta\right)
$$

Поэтому $\theta \in V$ является решением следующего уравнения:

$$
a A \theta+\frac{\mathrm{K}_{a}}{\alpha} \theta+b \mathrm{\kappa}_{a}[\theta]^{4}=g .
$$

Здесь

$$
g=B r+\frac{\mathrm{\kappa}_{a}}{\alpha} A^{-1} B(r+b u) \in V^{\prime} .
$$

Однозначная разрешимость уравнения (2.26) с монотонной нелинейностью хорошо известна (см. например [89]). Следовательно задача (2.23) однозначно разрешима.

Для получения оценок (2.25) умножим скалярно $(2.26)$ на $\theta \in V$ и отбросим неотрицательные слагаемые в левой части. Тогда

$$
a\|\theta\|_{V}^{2} \leqslant(g, \theta) \leqslant\|g\|_{V^{\prime}}\|\theta\|_{V}, \quad a\|\theta\|_{V} \leqslant\|g\|_{V^{\prime}} .
$$

Неравенства (2.25) позволяют оценить $\|g\|_{V^{\prime}}$ и $\|\varphi\|_{V}$,

$$
\|g\|_{V^{\prime}} \leqslant\|r\|_{\Gamma}+\frac{C_{0} \mathrm{~K}_{a}}{\alpha}\|r+b u\|_{\Gamma}, \quad\|\varphi\|_{V} \leqslant \frac{1}{\alpha b}\|r+b u\|_{\Gamma}+\frac{a}{\alpha b}\|\theta\|_{V} .
$$

В результате получаем оценки $(2.25)$.

Теорема 1. Пусть выполняттся условия (i),(ii). Тогда существует решение задачи $(C P)$.

Доказательство. Пусть $j_{\lambda}=\inf J_{\lambda}$ на множестве $u \in U, F(\theta, \varphi, u)=0$. Выберем минимизирующую последовательность $u_{m} \in U, \theta_{m} \in V, J_{\lambda}\left(\theta_{m}, u_{m}\right) \rightarrow$ $j_{\lambda}$,

$$
a A \theta_{m}+b \kappa_{a}\left([\theta]^{4}-\varphi_{m}\right)=B r, \quad \alpha A \varphi_{m}+\kappa_{a}\left(\varphi_{m}-[\theta]^{4}\right)=B u_{m} .
$$

Из ограниченности последовательности $u_{m}$ в пространстве $U$ следуют, на основании леммы 1 , оценки

$$
\left\|\theta_{m}\right\|_{V} \leqslant C,\|\varphi\|_{V} \leqslant C,\left\|\theta_{m}\right\|_{L^{6}(\Omega)} \leqslant C .
$$

Здесь через $C>0$ обозначены различные постоянные, не зависящие от $\mathrm{m}$. Переходя при необходимости к подпоследовательностям, заключаем, что существует тройка $\{\hat{u}, \hat{\theta}, \hat{\varphi}\} \in U \times V \times V$,

$$
u_{m} \rightarrow \hat{u} \text { слабо в } U, \theta_{m}, \varphi_{m} \rightarrow \hat{\theta}, \hat{\varphi} \text { слабо в } V, \text { сильно в } L^{4}(\Omega) \text {. }
$$

Заметим также, что $\forall v \in V$

$$
\mid\left(\left[\theta_{m}\right]^{4}-[\hat{\theta}]^{4}, v\right) \leqslant \leqslant 2\left\|\theta_{m}-\hat{\theta}\right\|_{L^{4}(\Omega)}\|v\|_{L^{4}(\Omega)}\left(\left\|\theta_{m}\right\|_{L_{6}(\Omega)}^{3}+\|\hat{\theta}\|_{L_{6}(\Omega)}^{3}\right) .
$$

Результаты о сходимости $(2.28),(2.29)$ позволяют перейти к пределу в (2.27). Поэтому

$$
a A \hat{\theta}+b \kappa_{a}\left([\hat{\theta}]^{4}-\hat{\varphi}=B r\right), \alpha A \hat{\varphi}+\kappa_{a}\left(\hat{\varphi}-[\hat{\theta}]^{4}\right)=B \hat{u}
$$

и при этом $j_{\lambda} \leqslant J_{\lambda}(\hat{\theta}, \hat{u}) \leqslant \liminf J_{\lambda}\left(\theta_{m}, u_{m}\right)=j_{\lambda}$. Следовательно тройка $\{\hat{\theta}, \hat{\varphi}, \hat{u}\}$ есть решение задачи $(C P)$.

\subsection{3 Условия оптимальности}
Для получения системы оптимальности достаточно использовать принцип Лагранжа для гладко-выпуклых экстремальных задач [90, 91]. Проверим справедливость ключевого условия, что образ производной оператора ограничений $F(y, u)$, где $y=\{\theta, \varphi\} \in V \times V$, совпадает с пространством $V^{\prime} \times V^{\prime}$. Именно это условие гарантирует невырожденность условий оптимальности. Напомним, что

$$
F(y, u)=\left\{a A \theta+b \kappa_{a}\left([\theta]^{4}-\varphi\right)-B r, \alpha A \varphi+\kappa_{a}\left(\varphi-[\theta]^{4}\right)-B u\right\} .
$$

Лемма 2. Пусть выполняются условия (і),(ii). ДЛл любой пары $\hat{y} \in$ $V \times V, \hat{u} \in U$ справедливо равенство

$$
\operatorname{Im} F_{y}^{\prime}(y, u)=V^{\prime} \times V^{\prime} .
$$

Доказательство. Достаточно проверить, что задача

$$
a A \xi+b \kappa_{a}\left(4|\hat{\theta}|^{3} \xi-\eta\right)=f_{1}, ; \quad \alpha A \eta+\kappa_{a}\left(\eta-4|\hat{\theta}|^{3} \xi\right)=f_{2}
$$

разрешима для всех $f_{1,2} \in V^{\prime}$. Данная задача равносильна системе

$$
a A \xi+\mathrm{K}_{a}\left(4 b|\theta|^{3}+\frac{a}{\alpha}\right) \xi=f_{1}+\frac{\mathrm{\kappa}_{a}}{\alpha} f_{3}, ; \quad \eta=\frac{1}{\alpha b}\left(f_{3}-a \xi\right) .
$$

Здесь $f_{3}=A^{-1}\left(f_{1}+b f_{2}\right) \in V$. Разрешимость первого уравнения указанной системы очевидным образом следует из леммы Лакса-Мильграма.

В соответствии с леммой 2 , лагранжиан задачи $(C P)$ имеет вид

$L\left(\theta, \varphi, u, p_{1}, p_{2}\right)=J_{\lambda}(\theta, u)+\left(a A \theta+b \kappa_{a}\left([\theta]^{4}-\varphi\right)-B r, p_{1}\right)+\left(\alpha A \varphi+\kappa_{a}\left(\varphi-[\theta]^{4}\right)-B u, p_{2}\right)$

Здесь $p=\left\{p_{1}, p_{2}\right\} \in V \times V$ - сопряженное состояние. Если $\{\hat{\theta}, \hat{\varphi}, \hat{u}\}-$ решение задачи $(C P)$, то в силу принципа Лагранжа [90, Теорема 1.5] справедливы вариационные равенства $\forall v \in V, w \in U$

$$
\begin{gathered}
\left(\hat{\theta}-\theta_{b}, v\right)_{\Gamma}+\left(a A v+4 b \kappa_{a}|\hat{\theta}|^{3} v, p_{1}\right)-\kappa_{a}\left(4|\hat{\theta}|^{3} v, p_{2}\right)=0, b \kappa_{a}\left(v, p_{1}\right)+\left(\alpha A v+\kappa_{a} v, p_{2}\right)=0, \\
\lambda(\hat{u}, w)_{\Gamma}-\left(B w, p_{2}\right)=0 .
\end{gathered}
$$

Таким образом, из условий (2.30),(2.31) получаем следующий результат

Теорема 2. Пусть выполняются условия (і), (ii). Если $\{\hat{\theta}, \hat{\varphi}, \hat{u}\}-$ решение задачи $(C P)$, то существует единственная пара $\left\{p_{1}, p_{2}\right\} \in V \times V$ такая, чmo

$$
\begin{gathered}
a A \hat{\theta}+b \kappa_{a}\left([\hat{\theta}]^{4}-\hat{\varphi}\right)=B r, \quad \alpha A \hat{\varphi}+\kappa_{a}\left(\hat{\varphi}-[\hat{\theta}]^{4}\right)=B \hat{u}, \\
a A p_{1}+4|\hat{\theta}|^{3} \kappa_{a}\left(b p_{1}-p_{2}\right)=B\left(\theta_{b}-\hat{\theta}\right), \alpha A p_{2}+\kappa_{a}\left(p_{2}-b p_{1}\right) \\
\text { и nри этом } \lambda \hat{u}=p_{2} .
\end{gathered}
$$

\subsection{4 Аппрокимация задачи с условиями типа Коши}
Рассмотрим краевую задачу для уравнений сложного теплообмена, в которой нет краевых условий на интенсивность излучения.

$$
-a \Delta \theta+b \kappa_{a}\left([\theta]^{4}-\varphi\right)=0, \quad-\alpha \Delta \varphi+\kappa_{a}\left(\varphi-[\theta]^{4}\right)=0, x \in \Omega .
$$

$$
\theta=\theta_{b}, \quad \partial_{n} \theta=q_{b} \text { на } \Gamma .
$$

Существование $\theta, \varphi \in H^{2}(\Omega)$, удовлетворяющих $(2.32),(2.33)$ для достаточно гладких $\theta_{b}, q_{b}$ и достаточные условия единственности решения установлены в [92]. Покажем, что решения задачи $(C P)$ при $\lambda \rightarrow+0$ аппроксимируют решение задачи $(2.32),(2.33)$.

Теорема 3. Пусть выполняются условия (i),(ii) и существует решение задачи (2.32),(2.33). Если $\left\{\theta_{\lambda}, \varphi_{\lambda}, u_{\lambda}\right\}$ - решение задачи $(C P)$ для $\lambda>0$, то существует последовательность $\lambda \rightarrow+0$ такая, что

$$
\theta_{\lambda} \rightarrow \theta_{*}, \varphi_{\lambda} \rightarrow \varphi_{*} \text { слабо в } V, \text { сильно в } H,
$$

где $\theta_{*}, \varphi_{*}$ - решение задачи $(2.32),(2.33)$.

Доказательство. Пусть $\theta, \varphi \in H^{2}(\Omega)$ - решение задачи $(2.32),(2.33), u=$ $\alpha\left(\partial_{n} \varphi+\varphi\right) \in U$. Тогда

$$
a A \theta+b \kappa_{a}\left([\theta]^{4}-\varphi\right)=B r, \quad \alpha A \varphi+\kappa_{a}\left(\varphi-[\theta]^{4}\right)=B u,
$$

где $r:=a\left(\theta_{b}+q_{b}\right)$. Поэтому, с учетом того, что $\left.\theta\right|_{\Gamma}=\theta_{b}$,

$$
J_{\lambda}\left(\theta_{\lambda}, u_{\lambda}\right)=\frac{1}{2}\left\|\theta_{\lambda}-\theta_{b}\right\|_{\Gamma}^{2}+\frac{\lambda}{2}\left\|u_{\lambda}\right\|_{\Gamma}^{2} \leqslant J_{\lambda}(\theta, u)=\frac{\lambda}{2}\|u\|_{\Gamma}^{2}
$$

Следовательно,

$$
\left\|u_{\lambda}\right\|_{\Gamma}^{2} \leqslant C,\left\|\theta_{\lambda}-\theta_{b}\right\|_{\Gamma}^{2} \rightarrow 0, \lambda \rightarrow+0 .
$$

Здесь и далее $C>0$ не зависит от $\lambda$. Из ограниченности последовательности $u_{\lambda}$ в пространстве $U$ следуют, на основании леммы 1 , оценки

$$
\left\|\theta_{\lambda}\right\|_{V} \leqslant C,\|\varphi\|_{\lambda} \leqslant C .
$$

Поэтому можно выбрать последовательность $\lambda \rightarrow+0$ такую, что

$$
u_{\lambda} \rightarrow u_{*} \text { слабо в } U, \theta_{\lambda}, \varphi_{\lambda} \rightarrow \theta_{*}, \varphi_{*} \text { слабо в } V \text {, сильно в } L^{4}(\Omega) \text {. }
$$

Результаты (2.34) позволяют перейти к пределу при $\lambda \rightarrow+0$ в уравнениях для $\theta_{\lambda}, \varphi_{\lambda}, u_{\lambda}$ и тогда

$$
a A \theta_{*}+b \kappa_{a}\left(\left[\theta_{*}\right]^{4}-\varphi_{*}\right)=B r, \quad \alpha A \varphi_{*}+\kappa_{a}\left(\varphi_{*}-\left[\theta_{*}\right]^{4}\right)=B u_{*} .
$$

При этом $\left.\theta_{*}\right|_{\Gamma}=\theta_{b}$. Из первого уравнения в $(2.35)$, с учетом, что $r=a\left(\theta_{b}+q_{b}\right)$, Выводим

$$
-a \Delta \theta_{*}+b \kappa_{a}\left(\left[\theta_{*}\right]^{4}-\varphi_{*}\right)=0 \text { п.в. в } \Omega, \quad \theta_{*}=\theta_{b}, \quad \partial_{n} \theta=q_{b} \text { п.в. на } Г .
$$

Из второго уравнения в $(2.35)$ следует, что $-\alpha \Delta \varphi+\kappa_{a}\left(\varphi-[\theta]^{4}\right)=0$ почти всюду в $\Omega$. Таким образом, пара $\theta_{*}, \varphi_{*}-$ решение задачи $(2.32),(2.33)$.

\section{Глава 3. Численные эксперименты}
\section{$3.1$ Численный алгоритм нахождения квазирешения}
Пусть функционал $J(\theta)$ удовлетворяет условиям, указанным в разд. 2.1.3. Для удобства введём переобозначение $\hat{J}(u):=J(\theta(u)), \hat{J}: L^{2}\left(\Gamma_{1}\right) \rightarrow \mathbb{R}$. Здесь $\theta(u)$ - температурное поле задачи (2.1)-(2.2) отвечающее управлению $u \in L^{2}\left(\Gamma_{1}\right)$. Согласно формуле (2.14) градиент функционала $\hat{J}(u)$ [93] имеет вид

$$
\hat{J}^{\prime}(u)=\left(\varphi(u)-\theta_{b}^{4}\right) p_{2},
$$

где $\varphi(u)$ есть интенсивность излучения, $p_{2}$ - соответствующая переменная сопряжённой системы.

Предлагаемый алгоритм решения выглядит следующим образом:

\begin{center}
\begin{tabular}{l}
\hline
Алгоритм градиентного спуска с проекцией \\
\hline
1: Выбираем значение градиентного шага $\lambda$, \\
2: Выбираем количество итераций $N$, \\
3: Выбираем произвольное $u_{0} \in U_{a d}$, \\
4: for $k \leftarrow 0,1,2, \ldots, N$ do : \\
5: $\quad$ Для полученного $u_{k}$ расчитываем состояние $y_{k}=\left\{\theta_{k}, \varphi_{k}\right\}$ из (2.6). \\
6: $\quad$ Расчитываем значение функционала качества $J\left(\theta_{k}\right)$ из $(2.5)$. \\
7: $\quad$ Расчитываем сопряжённое состояние $p_{k}=\left\{p_{1 k}, p_{2 k}\right\} \quad$ из уравнений \\
$\quad$ (2.12)-(2.13), где $\hat{\theta}:=\theta_{k}, \hat{u}=u_{k}$. \\
8: Пересчитываем управление $u_{k+1}=P_{a d}\left[u_{k}-\lambda\left(\varphi_{k}-\theta_{b}^{4}\right) p_{2 k}\right]$. \\
\hline
\end{tabular}
\end{center}

Оператор проекции $P_{a d}: U \rightarrow U_{a d}$ определён следующим образом

$$
P_{a d}[v]= \begin{cases}u_{1}, & \text { если } v \leqslant u_{1} \\ v, & \text { если } u_{1}<v<u_{2} \\ u_{2}, & \text { если } v \geqslant u_{2}\end{cases}
$$

Приведём далее примеры расчётов для двумерного случая. Положим $\Omega=$ $\{(x, y), 0 \leqslant x, y \leqslant 1\}, l=1$ см. Граница $\partial \Omega$ состоит из участков:

$$
\begin{aligned}
&\Gamma_{0}=\{x=\{0,1\}, y \in[0,1]\} \\
&\Gamma_{1}=\{x \in[0,1], y=0\}-\text { участок с неизвестными отражающими свойствами, } \\
&\Gamma_{2}=\{x \in[0,1], y=1\}-\text { участок наблюдения. }
\end{aligned}
$$

Будем также далее считать, что $a=0.006\left[\mathrm{~cm}^{2} / \mathrm{c}\right], b=0.025[\mathrm{~cm} / \mathrm{c}], \beta=$ $0.00005[\mathrm{~cm} / \mathrm{c}], \mathrm{\kappa}=1\left[\mathrm{~cm}^{-1}\right], \mathrm{\kappa}_{s}=0, A=0, \gamma=0.3$. Указанные параметры соответствуют стеклу [93]. Температуру на границе $\Omega$ положим равной $\theta_{b}=\left(x^{2}+y^{2}\right) / 3$

При указанных параметрах для первого эксперимента выберем следующее тестовое значение функции $u$ (рис. 3.1):

$$
u(x)= \begin{cases}0.01, & \text { если } x \leqslant 0.5 \\ 0.5, & \text { если } x>0.5\end{cases}
$$

и для второго эксперимента (рис. 3.1):

$$
u(x)=0.49 x+0.01 .
$$

Вычислим решение прямой задачи (2.1)-(2.2) для этих случаев. Полученное температурное поле на участке наблюдения $\Gamma_{2}$ выберем в качестве $\theta_{0}$. Далее, применяя предложенный алгоритм находим квазирешение обратной задачи (2.1)-(2.4). Эффективность алгоритма, а также значение $u_{0}$ в первом и втором случаях иллюстрируются рис. 3.1. На рис. $2.20$ показана динамика функционала качества по итерациям.

Замечание. В предложенных примерах потребовалось $2 * 10^{6}$ итераций для нахождения квазирешения $u$. В то же время температурное поле на участке наблюдения $\Gamma_{2}$ становится близким к $\theta_{0}$ уже на $10^{2}$ итерации. Также наблюдается существенное падение скорости уменьшения функционала качества с каждой итерацией после того, как среднее значение найденной функции контроля становится близко к тестовой функции.

\begin{center}
\includegraphics[max width=\textwidth]{2022_10_02_2aab7a2952871da15492g-37(2)}
\end{center}

Первый эксперимент

\begin{center}
\includegraphics[max width=\textwidth]{2022_10_02_2aab7a2952871da15492g-37(1)}
\end{center}

Второй эксперимент

Рисунок $3.1$ - Тестовая функция $u$, начальная $u_{0}$, найденная функция $u_{e n d}$.

\begin{center}
\includegraphics[max width=\textwidth]{2022_10_02_2aab7a2952871da15492g-37(3)}
\end{center}

Первый эксперимент

\begin{center}
\includegraphics[max width=\textwidth]{2022_10_02_2aab7a2952871da15492g-37}
\end{center}

Второй эксперимент

Рисунок $3.2$ - Динамика функции $\hat{J}(u)$ по итерациям.

\section{$3.2$ Численная реализация решения оптимизационной задачи}
Представим алгоритм решения задачи управления. Заметим, что в соответствии с (2.31) градиент функционала качества равен

$$
J_{\lambda}^{\prime}(u)=\lambda u-p_{2} .
$$

Здесь $u \in U$ - управление в граничном условии $(2.19), p_{2}-$ соответствующая компонента сопряженного состояния из системы (2.30)-(2.31).

Предлагаемый алгоритм решения задачи (CP) выглядит следующим образом: Алгоритм градиентного спуска

1: Выбираем значение градиентного шага $\varepsilon$,

2: Выбираем количество итераций $N$,

3: Выбираем произвольное $u_{0} \in U$,

4: for $k \leftarrow 0,1,2, \ldots, N$ do :

5: $\quad$ Для полученного $u_{k}$ расчитываем состояние $y_{k}=\left\{\theta_{k}, \varphi_{k}\right\}$ из (2.18).

6: $\quad$ Расчитываем значение функционала качества $J\left(\theta_{k}, u_{k}\right)$ из (2.20).

7: Расчитываем состояние $p_{k}=\left\{p_{1 k}, p_{2 k}\right\}$ из уравнений (2.30), где $\hat{\theta}:=$ $\theta_{k}, \hat{u}=u_{k}$.

8: $\quad$ Пересчитываем управление $u_{k+1}=u_{k}-\varepsilon\left(\lambda u_{k}-p_{2}\right)$

Значение параметра $\varepsilon$ выбирается эмпирически таким образом, чтобы значение $\varepsilon\left(\lambda u_{k}-p_{2}\right)$ являлась существенной поправкой для $u_{k+1}$. Количество итераций $N$ выбирается достаточным для выполнения условия $J\left(\theta_{k}, u_{k}\right)-$ $J\left(\theta_{k+1}, u_{k+1}\right)<\delta$, где $\delta>0$ задает точность расчетов.

Примеры, рассмотренные ниже, иллюстрируют эффективность предложенного алгоритма даже при малых значениях параметра регуляризации $\lambda<$ $10^{-12}$

Для численного решения прямой задачи с заданным управлением и сопряженной системы использовался солвер FEniCS [94, 95].

Приведем примеры расчетов для куба с ребром $l=1$ см, $\Omega=$ $(x, y, z), 0 \leqslant x, y, z \leqslant l$. Будем также далее считать, что $a=0.006\left[\mathrm{~cm}^{2} / \mathrm{c}\right], b=$ $0.025[\mathrm{~cm} / \mathrm{c}], \beta=0.00005[\mathrm{~cm} / \mathrm{c}], \mathrm{k}=1\left[\mathrm{~cm}^{-1}\right], \mathrm{\kappa}_{s}=0, A=0, \gamma=0.3$. Указанные параметры соответствуют стеклу [96]. Параметр регуляризации $\lambda=10^{-12}$.

Пример 1. Пусть граничные данные $r$ и $u$ в $(2.19)$ имеют вид:

$$
r=0.8 \cos (x)+0.1, \quad u=\hat{u}=y .
$$

Далее рассчитываем состояние $\theta$ и $\varphi$ как решение задачи (2.18)-(2.19) и в качестве $\theta_{b}$ выбираем граничные значение функции $\theta$ на Г. Применяя предложенный алгоритм с начальным приближением $u_{0}=0.1$ находим приближенное решение задачи СР. Квадрат разницы тестового и найденного решения, а также динамика функционала качества представлена на рисунке 3.3.

\begin{center}
\includegraphics[max width=\textwidth]{2022_10_02_2aab7a2952871da15492g-39}
\end{center}

a)

\begin{center}
\includegraphics[max width=\textwidth]{2022_10_02_2aab7a2952871da15492g-39(3)}
\end{center}

Рисунок $3.3$ - Пример 1.

\begin{center}
\includegraphics[max width=\textwidth]{2022_10_02_2aab7a2952871da15492g-39(1)}
\end{center}

б)

Пример 2. Для второго эксперимента вместо тестовой функции управления $u$ определим функции $\theta_{b}, q_{b}$ из (2.21) следующим образом

$$
\theta_{b}=0.1 z+0.3, q_{b}= \begin{cases}0.11, & \text { если } z=1, \\ 0, & \text { если } 0<z<1, \\ -0.15, & \text { если } z=0 .\end{cases}
$$

В данном примере оптимальное управление $u$ в качестве тестового не задается. На рисунке $3.4$ представлен результат работы алгоритма

\begin{center}
\includegraphics[max width=\textwidth]{2022_10_02_2aab7a2952871da15492g-39(4)}
\end{center}

Найденное оптимальное управление

\begin{center}
\includegraphics[max width=\textwidth]{2022_10_02_2aab7a2952871da15492g-39(2)}
\end{center}

Изменение функционала в

Рисунок $3.4$ - Пример $2.1$

Компоненты состояния, соотвествующие найденному управлению представлены на рисунке 3.5.

\begin{center}
\includegraphics[max width=\textwidth]{2022_10_02_2aab7a2952871da15492g-40}
\end{center}

Излучение $\varphi$

Рисунок $3.5$ - Пример $2.2$

\begin{center}
\includegraphics[max width=\textwidth]{2022_10_02_2aab7a2952871da15492g-40(1)}
\end{center}

Температура $\theta]$

\section{Заключение}
В диссертации доказано существование квазирешения задачи нахождения коэффициента отражения участка границы для стационарной модели, по дополнительной информации о температурном поле. Экспериментально поверена устойчивость получаемых решений методом градиентного спуска. Таким образом, получены важные с теоретической точки зрения результаты, которые могут быть полезны при дальнейшем использовании стационарных моделей сложного теплообмена и анализе обратных задач в рамках нестационарных моделей сложного теплообмена. Развитые методы исследования начально-краевых задач могут применяться для изучения различных моделей, описываемых нелинейными уравнениями со сходной структурой.

Разработанный комплекс программ для постановки численных экспериментов показал свою надёжность и может в дальнейшем быть использован как пример для решения подобных задач.

Разработан комплекс программ для проведения вычислительных экспериментов. Для презентации результатов расчётов, помимо самих солверов, был разработан программный комплекс для отрисовки полученных расчётов в трёхмерных областях.

Исследование нестационарных моделей сложного теплообмена и соответсвующих им обратных задач является крайне перспективной областью математического моделирования и в то же время достаточно сложной для теоретического анализа и реализации численных решений. Более широкий класс процессов может быть покрыт задачами на оптимальное управление многими переменными среды, что позволяет более точно находить решения для инженерных задач.

\section{Список литературы}
[1] Clever D. и Lang J. «Optimal control of radiative heat transfer in glass cooling with restrictions on the temperature gradient,» B: Optimal Control Appl. and Meth., 33(2) (2012) 157-175. (2012).

[2] О. Tse, R. Pinnau и N. Siedow. «Identification of temperature-dependent parameters in laser interstilial thtermo therapy». B: Mathematical Models and Methods in Applied Sciences $22.09$ (2012).

[3] «Амосов А. А. Глобальная разрешимость одной нелинейной нестационарной задачи с нелокальным краевым условием типа теплообмена излучением // Дифференц. уравнения. —— T. 41, № 1. - С. 93-104.» В: (2005).

[4] «Амосов А. А. Разрешимость стационарной задачи радиационно-кондуктивного теплообмена в системе серых тел // Вестник МЭИ. —— № 6. C. 72-93.»B: (2009).

[5] «Amosov A. A. Nonstationary nonlinear nonlocal problem of radiative conductive heat transfer in a system of opaque bodies with properties depending on the radiation frequency // J. Math. Sci. —\_ Vol. 165, no. 1. - P. 1-41.»B: (2010).

[6] «Amosov A. A. Stationary nonlinear nonlocal problem of radiative-conductive heat transfer in a system of opaque bodies with properties depending on the radiation frequency // J. Math. Sci.——Vol. 164, no. 3. — P. 309-344.»B: $(2010)$

[7] «Druet P.-E. Weak solutions to a stationary heat equation with nonlocal radiation boundary condition and right-hand side in $\operatorname{Lp}(\mathrm{p} \leqslant 1) / /$ Math. Methods Appl. Sci. —— Vol. 32, no. 2. — P. 135-166.»B: (2009).

[8] «Druet P.-E. Weak solutions to a time-dependent heat equation with nonlocal radiation boundary condition and arbitrary p-summable right-hand side // Appl. Math. —— Vol. 55, no. 2. — P. 111-149.» B: (2010).

[9] «Laitinen M., Tiihonen T. Conductive-radiative heat transfer in grey materials // Quart. Appl. Math. —— Vol. 59. — P. 737-768.»B: (2001). [10] «Metzger M. Existence for a time-dependent heat equation with non-local radiation terms // Math. Methods Appl. Sci. - Vol. 22, no. 13. - P. 1101-1119.» B: (1999).

[11] «Philip P. Analysis, optimal control, and simulation of conductive-radiative heat transfer// Ann. Acad. Rom. Sci. Ser. Math. Appl. —\_ Vol. 2. - P. 171-204.»B: (2010).

[12] «Tiihonen T. A nonlocal problem arising from heat radiation on non-convex surfaces // European J. Appl. Math. ——Vol. 8, no. 4.——P. 403-416.»B: (1997)

[13] «Tiihonen T. Stefan-Boltzmann radiation on non-convex surfaces// Math. Methods Appl. Sci. ——Vol. 20, no. 1. — P. 47-57.»B: (1997).

[14] «Existence and uniqueness of a steady state solution of a coupled radiativeconductive heat transfer problem for a non-grey anisotropically and participating medium / F. Asllanaj, G. Jeandel, J. R. Roche, D. Schmitt // Transport Theory Statist. Phys. ——Vol. 32, no. 1. — P. 1-35.»B: (2003).

[15] «Kelley C. T. Existence and uniqueness of solutions of nonlinear systems of conductive-radiative heat transfer equations// Transport Theory Statist. Phys. —— Vol. 25, no. 2. — P. 249-260.»B: (1996).

[16] «Ghattassi M., Roche J. R., Schmitt D. Existence and uniqueness of a transient state for the coupled radiative-conductive heat transfer problem // Comput. Math. Appl.—— Vol. 75, no. 11. — P. 3918-3928.» B: (2018).

[17] «Porzio M. M., L'opez-Pouso O. Application of accretive operators theory to evolutive combined conduction, convection and radiation// Rev. Mat. Iberoamericana. —— Vol. 20, no. 1. — P. 257-275.» B: (2004).

[18] «Thompson M., Segatto C., de Vilhena M. Existence theory for the solution of a stationary nonlinear conductive-radiative heat-transfer problem in three space dimensions // Transport Theory Statist. Phys. —— Vol. 33, no. 5-7. P. $563-576 . » \mathrm{~B}:(2004)$.

[19] «Гершуни Г. З., Жуковицкий Е. М. Конвективная устойчивость несжимаемой жидкости. - М. : Наука, .» В: (1972). [20] «Asllanaj F., Parent G., Jeandel G. Transient radiation and conduction heat transfer in a gray absorbing-emitting medium applied on two-dimensional complex-shaped domains // Numerical Heat Transfer, Part B. —— Vol. 52, no. 2. — P. 179-200.»B: (2007).

[21] «Banoczi J. M., Kelley C. T. A fast multilevel algorithm for the solution of nonlinear systems of conductive-radiative heat transfer equations in two space dimensions // SIAM J. Sci. Comput. ——Vol. 20, no. 4. — P. 1214-1228.» B: (1999).

[22] «Galerkin method for solving combined radiative and conductive heat transfer / M. Ghattassi, J. R. Roche, F. Asllanaj, M. Boutayeb // International Journal of Thermal Sciences. —— Vol. 102. — P. 122-136.»B: (2016).

[23] «Klein O., Philip P. Transient conductive-radiative heat transfer: Discrete existence and uniqueness for a finite volume scheme // Math. Models Methods Appl. Sci. —— Vol. 15, no. 2. — P. 227-258.»B: (2005).

[24] «Kovtanyuk A. E., Botkin N. D., Hoffmann K.-H. Numerical simulations of a coupled radiative-conductive heat transfer model using a modified Monte Carlo method // Int. J. Heat. Mass Transf. ——Vol. 55, no. 4.——. 649-654.» B: (2012).

[25] «Амосов А. А. Нестационарная задача сложного теплообмена в системе полупрозрачных тел с краевыми условиями диффузного отражения и преломления излучения // Современная математика. Фундаментальные направления. —— Т. 59. - С. 5-34.» В: (2016).

[26] «Амосов А. А. Стационарная задача сложного теплообмена в системе полупрозрачных тел с краевыми условиями диффузного отражения и преломления излучения $/ / Ж$. вычисл. матем. и матем. физ. —— Т. 57, № 3. — C. $510-535 . »$ B: (2017).

[27] «Amosov A. A. Unique solvability of a nonstationary problem of radiativeconductive heat exchange in a system of semitransparent bodies // Russ. J. Math. Phys. ―—Vol. 23, no. 3. — P. 309-334.» B: (2016).

[28] «Amosov A. A. Unique solvability of stationary radiative-conductive heat transfer problem in a system of semitransparent bodies // J. Math. Sci. Vol. 224, no. 5. — P. 618-646.»B: (2017). [29] «End T. On analytical results for the optimal control of the quasi-stationary radiative heat transfer system // Proc. Appl. Math. Mech. - Vol. 11, no. 1. — P. 793-794.»B: (2011).

[30] «End T. On optimization of the full radiative heat transfer system// Proc. Appl. Math. Mech. ―— Vol. 10, no. 1. — P. 533-534.» B: (2010).

[31] «Pereverzyev S. S., Pinnau R., Siedow N. Initial temperature reconstruction for nonlinear heat equation: application to a coupled radiative-conductive heat transfer problem // Inverse Probl. Sci. Eng. ——Vol. 16, no. 1. — P. 55-67.» B: (2008).

[32] «Birgelis K. Optimal control in models with conductive-radiative heat transfer // Math. Model. Anal. Vol. 8, no. 1. — P. 1-12.»B: (2003).

[33] «Meyer C., Philip P., Tr “oltzsch F. Optimal control of a semilinear PDE with nonlocal radiation interface conditions // SIAM J. Control Optim. - Vol. 45, no. 2. — P. 699-721.» B: (2006).

[34] «Meyer C., Yousept I. State-constrained optimal control of semilinear elliptic equations with nonlocal radiation interface conditions// SIAM J. Control Optim. —— Vol. 48, no. 2. — P. 734-755.»B: (2009).

[35] «Belmiloudi A., Mah'e F. On nonlinear inverse problems of heat transfer with radiation boundary conditions: application to dehydration of gypsum plasterboards exposed to fire // Adv. Numer. Anal. ―— Vol. .» B: (2014).

[36] «Kovtanyuk A. E., Chebotarev A. Y. An iterative method for solving a complex heat transfer problem// Appl. Math. Comput. - Vol. 219, no. 17. — P. 9356-9362.»B: (2013).

[37] «Numerical methods and optimal control for glass cooling processes / $\mathrm{G}$. Th"ommes, R. Pinnau, M. Sea ${ }^{*} 1 \mathrm{~d}$ et al. // Transport Theory Statist. Phys. ― Vol. 31, no. 4-6. — P. 513-529.»B: (2002).

[38] «Pinnau R., Sea ${ }^{*}$ 1d M. Simplified PN models and natural convectionradiation // Progress in Industrial Mathematics at ECMI- Springer, 2008. - P. 397-401.»B: (2006).

[39] «Siewert C., Thomas J. A computational method for solving a class of coupled conductive-radiative heat transfer problems // J. Quant. Spectrosc. Radiat. Transfer. —— Vol. 45, no. 5. — P. 273-281.»B: (1991). [40] «Analysis of a fractional-step scheme for the $\mathrm{P} 1$ radiative diffusion model/ T. Gallou eet, R. Herbin, A. Larcher, J.-C. Latch'e // Comput. Appl. Math. Vol. 35, no. 1. — P. 135-151.» B: (2016).

[41] «Elliptic formulation of the simplified spherical harmonics method in radiative heat transfer / M. F. Modest, J. Cai, W. Ge, E. Lee// Int. J. Heat Mass Tran. — Vol. 76. — P. 459-466.» B: (2014).

[42] «Frank M., Lang J., Sch“afer M. Adaptive finite element simulation of the time-dependent simplified PN equations // J. Sci. Comput. - Vol. 49, no. 3. - P. 332-350.»B: (2011).

[43] «Simplified PN approximations to the equations of radiative heat transfer and applications / E. W. Larsen, G. Th“ommes, A. Klar et al. // J. Comput. Phys. Vol. 183, no. 2. - P. 652-675.»B: (2002).

[44] «Time-dependent simplified $\mathrm{PN}$ approximation to the equations of radiative transfer / M. Frank, A. Klar, E. W. Larsen, S. Yasuda // J. Comput. Phys. — Vol. 226, no. 2. — P. 2289-2305.»B: (2007).

[45] «Addam M., Bouhamidi A., Seaid M. A frequency-domain approach for the P1 approximation of time-dependent radiative transfer// J. Sci. Comput. Vol. 62, no. 3. — P. 623-651.» B: (2015).

[46] «Asymptotic derivation and numerical investigation of time-dependent simplified PN equations / E. Olbrant, E. W. Larsen, M. Frank, B. Seibold // J. Comput. Phys. —— Vol. 238. — P. 315-336.» B: (2013).

[47] «Frank M., Klar A., Pinnau R. Optimal control of glass cooling using simplified PN theory // Transport Theory Statist. Phys. - Vol. 39, no. 2-4. - P. 282-311.» B: (2010).

[48] «Pinnau R. Analysis of optimal boundary control for radiative heat transfer modelled by SP1 system // Commun. Math. Sci. —— Vol. 5, no. 4. - P. 951-969.» B: (2007).

[49] «Pinnau R., Tse O. Optimal control of a simplified natural convectionradiation model // Commun. Math. Sci.—— Vol. 11, no. 3. — P. $679-707 . »$ B: (2013).

[50] «Ковтанюк А. Е. Стационарные модели переноса излучения и сложного теплообмена : дисс. д-ра физ.-мат. наук / А. Е. Ковтанюк. —.» В: (2014). [51] «Ковтанюк А. Е., Чеботарев А. Ю. Нелокальная однозначная разрешимость стационарной задачи сложного теплообмена // Ж. вычисл. матем. и матем. физ. —— Т. 56, № 5. - С. 816-823.» В: (2016).

[52] «Unique solvability of a steady-state complex heat transfer model / A. E. Kovtanyuk, A. Yu. Chebotarev, N. D. Botkin, K.-H. Hoffmann// Commun. Nonlinear Sci. Numer. Simul. —— Vol. 20, no. 3. — P. 776-784.»B: (2015).

[53] «Чеботарев А. Ю., Гренкин Г. В., Ковтанюк А. Е. Однозначная разрешимость субдифференциальной краевой задачи для уравнений сложного теплообмена // Дальневост. матем. журн. —— Т. 16, № 2. - С. 229-236.» B: (2016)

[54] «Astrakhantseva A. A., Chebotarev A. Y., Kovtanyuk A. E. Analysis of the radiative-conductive heat transfer equations with unknown intensity of heat sources // Progress In Electromagnetics Research Symposium - Spring (PIERS). —— P. 1359-1361.» B: (2017).

[55] «Inverse problem with finite overdetermination for steady-state equations of radiative heat exchange / A. Yu. Chebotarev, G. V. Grenkin, A. E. Kovtanyuk et al. // J. Math. Anal. Appl. ——Vol. 460, no. 2. — P. 737-744.» B: (2018).

[56] «Clever D., Lang J. Optimal control of radiative heat transfer in glass cooling with restrictions on the temperature gradient// Optimal Control Appl. Methods. —— Vol. 33, no. 2. — P. 157-175.» B: (2012).

[57] «Clever D., Lang J., Schr* oder D. Model hierarchy-based optimal control of radiative heat transfer // Int. J. Computational Science and Engineering. Vol. 9, no. 5/6. — P. 509-525.» B: (2014).

[58] «Lang J. Adaptive computation for boundary control of radiative heat transfer in glass // J. Comput. Appl. Math. ——Vol. 183, no. 2. — P. 312-326.» B: (2005)

[59] «Pinnau R., Schulze A. Newton's method for optimal temperature-tracking of glass cooling processes // Inverse Probl. Sci. Eng. —— Vol. 15, no. 4.—P. 303-323.»B: (2007).

$[60]$ «innau R., Th`̈ommes G. Optimal boundary control of glass cooling processes // Math. Methods Appl. Sci. ——Vol. 27, no. 11. - P. 12611281.»B: (2004). [61] «Optimal boundary control of a steady-state heat transfer model accounting for radiative effects / A. E. Kovtanyuk, A. Yu. Chebotarev, N. D. Botkin, K.-H. Hoffmann // J. Math. Anal. Appl.——Vol. 439, no. 2. — P. 678-689.» B: $(2016)$.

[62] «Astrakhantseva A. A., Chebotarev A. Y., Kovtanyuk A. E. Design of the boundary reflection properties to minimize the energy flows // Progress In Electromagnetics Research Symposium - Spring (PIERS). — P. 1332-1336.»B: (2017).

[63] «Strong optimal controls in a steady-state problem of complex heat transfer / A. Yu. Chebotarev, A. E. Kovtanyuk, N. D. Botkin, K.-H. Hoffmann// IFIP Conference on System Modeling and Optimization/ Springer. P. 209-219.»B: (2015).

[64] «Theoretical analysis of an optimal control problem of conductiveconvectiveradiative heat transfer / A. E. Kovtanyuk, A. Yu. Chebotarev, N. D. Botkin, K.-H. Hoffmann // J. Math. Anal. Appl. —— Vol. 412, no. 1. — P. 520-528.»B: (2014).

[65] «Месенев П. Р., Чеботарев А. Ю. Граничная обратная задача для уравнений сложного теплообмена // Дальневост. матем. журн. - Т. 18, № 1. — C. 75-84.» B: (2018).

[66] «А. Ю. Чеботарев, П. Р. Месенев, "Алгоритм решения краевой задачи радиационного теплообмена без условий для интенсивности излучения", Дальневост. матем. журн., 20:1, 114-122». В: (2020).

[67] Modest M.F. «Radiative Heat Transfer». B: Academic Press (2013).

[68] «Оцисик М.Н. Сложный теплообмен. М: Мир». В: (1976).

[69] «Kovtanyuk A.E., Chebotarev A.Yu., Botkin N.D., Hoffmann K.-H. The unique solvability of a complex 3D heat transfer problem// J. Math. Anal. Appl. V. 409. 2. P. 808-815.» B: (2014).

[70] «Тепломассообмен [Электронный ресурс]: курс лекций / М.С. Лобасова, К.А. Финников, Т.А. Миловидова и др. - Красноярск: ИПК СФУ,» в: $(2009)$

[71] «Мазо А.Б. Основы теории и методы расчета теплопередачи: учебное пособие. Казань: Казан. ун-т,» в: (2013). [72] «Ильин В.А., Позняк Э.Г. Основы математического анализа: В 2-х ч. Часть II: Учеб.: Для вузов. 4-е изд. М.: ФИЗМАТЛИТ,» в: (2002).

[73] «Frank M., Klar A., Pinnau R. Optimal control of glass cooling using simplified $P_{N}$ theory// Transport Theory Statist. Phys. V. 39. 2-4. P. 282-311.»B: $(2010)$

[74] «Marshak R. E. Note on the spherical harmonic method as applied to the Milne problem for a sphere// Phys. Rev. V. 71. 7. P. 443-446.» B: (1947).

[75] «Ковтанюк А.Е., Чеботарев А.Ю. Стационарная задача сложного теплообмена // Журн. вычисл. матем. и матем. физ. Т. 54. 4. С. 711-719.» В: (2014)

[76] «Duderstadt J.J., Hamilton L.J. Nuclear reactor analysis. Wiley,» в: (1976).

[77] «Duderstadt J.J., Martin W.R. Transport theory. Wiley,» в: (1979).

[78] «Wang L.V., Wu H. Biomedical optics. Wiley,» в: (2007).

[79] «Boas D.A. Diffuse photon probes of structural and dynamical properties of turbid media: theory and biomedical applications. A Ph.D. Dissertation in Physics. University of Pennsylvania,» B: (1996).

[80] «Fishkin J.B., Gratton E. Propagation of photon-density waves in strongly scattering media containing an absorbing semi-infinite plane bounded by a straight edge // J. Opt. Soc. Am. A. V. 10. 1. P. 127-140.» B: (1993).

[81] «Haskell R.C., Svaasand L.O., Tsay T.-T., Feng T.-C., McAdams M.S., Tromberg B.J. Boundary conditions for the diffusion equation in radiative transfer // J. Opt. Soc. Am. A. V. 11. 10. P. 2727-2741.» B: (1994).

[82] «Кейз К., Цвайфель П. Линейная теория переноса. М.: Мир,» в: (1972).

[83] Pinnau R. «Analysis of optimal boundary control for radiative heat transfer modeled by the $S P_{1}$-system». B: Comm. Math. Sci., 5(4) 951-969. (2007).

[84] Kovtanyuk A.E. и др. «Unique solvability of a steady-state complex heat transfer model,» в: Commun. Nonlinear Sci. Numer. Simulat. 20 (2015) $796-784 \cdot(2015)$

[85] Kovtanyuk Andrey E. и др. «Theoretical analysis of an optimal control problem of conductive convective radiative heat transfer». B: J. Math. Anal. Appl. 412 (2014) 520-528 (2013). [86] A.D. Ioffe и V.M. Tikhomirov. «Theory of Extremal Problems». B: NorthHolland, Amsterdam (1979).

[87] «Modest M.F. Radiative Heat Transfer. Academic Press.822 p.» B: (2003).

[88] «Kovtanyuk A.E., Chebotarev A.Yu., Botkin N.D. Unique solvability of a steady-state complex heat transfer model// Commun. Nonlinear Sci. Numer. Simul.V. 20. N 3. P. 776-784.»B: (2015).

[89] «S. Fuč ik, A. Kufner, Nonlinear differential equations, Elsevier, Amsterdam-Oxford-New York,» B: (1980).

[90] «Зверев В. Г., Гольдин В. Д., Назаренко В. Радиационно-кондуктивный теплоперенос в волокнистой термостойкой изоляции при тепловом воздействии // Теплофизика высоких температур. —— Т. 46, № 1. - С. 119-125.» B: $(2008)$

[91] «Калиткин Н. Н. Численные методы. - М. : Наука, .» В: (1978).

[92] «Колобов А.Г., Пак Т.В., Чеботарев А.Ю. Стационарная задача радиационного теплообмена с граничными условиями типа Коши// Журнал вычислительной математики и математической физики. , том 59 , № 7 , с. 1258-1263.» B: (2019).

[93] Chebotarev A. и др. «Boundary optimal control problem of complex heat transfer model». B: J. Math. Anal. Appl., 433:2, 1243-1260 (2016).

[94] «M. S. Alnaes, J. Blechta, J. Hake, A. Johansson, B. Kehlet, A. Logg, C. Richardson, J. Ring, M. E. Rognes, G. N. Wells The FEniCS Project Version 1.5 Archive of Numerical Software, vol. 3, ,» в: (2015).

[95] «A. Logg and G. N. Wells, DOLFIN: Automated Finite Element Computing ACM Transactions on Mathematical Software, vol. 37». B: (2010).

[96] «Grenkin G.V., Chebotarev A.Yu., Kovtanyuk A.E., Botkin N.D., Hoffmann K.-H. Boundary optimal control problem of complex heat transfer model $/ /$ J. Math. Anal. Appl.V. 433. № 2. P. 1243-1260.» B: (2016).

\section{Приложение А}
\section{Приложение}
\section{A.1 Расчёт квазирешения для обратной граничной задачи в пакете FreeFem $++$}
\begin{center}
\includegraphics[max width=\textwidth]{2022_10_02_2aab7a2952871da15492g-51}
\end{center}

\begin{center}
\includegraphics[max width=\textwidth]{2022_10_02_2aab7a2952871da15492g-52}
\end{center}

II Получumb значение управления на границе func real uwallfunc(real xx, real yy)

\section{A.2 Код программы генерации изображений на языке Python}
\begin{center}
\includegraphics[max width=\textwidth]{2022_10_02_2aab7a2952871da15492g-53}
\end{center}

\begin{center}
\includegraphics[max width=\textwidth]{2022_10_02_2aab7a2952871da15492g-54}
\end{center}


\end{document}