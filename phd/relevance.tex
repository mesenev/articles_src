%! Author = mesenev
%! Date = 23.07.2021

% Preamble
\documentclass[11pt]{article}

% Packages
\usepackage{amsmath}

% Document
\begin{document}


    Под сложным теплообменом понимают процесс переноса тепла, в котором участвуют
    несколько процессов переноса тепла -- радиационный, конвективный и кондуктивный.
    В данном процессе учёт радиационного процеса становится более существенным с
    увеличением температуры.
    Данные процессы моделируются системой из дифферинциального уравнения теплопроводности,
    а также интегро--дифференциального уравнения переноса излучения.
    Исследование обратных задач, поставленных на моделях сложного теплообмена представляет из себя
    решение задач поставленных с одним или несколькими неизвестными параметрами, или с целью получить
    определенные значения решения на границе.

    Физические процессы теплообмена, а также соответствующие математические
    модели крайне востребованы для исследований.
    Теоретические знания полученные в ходе работ ложатся в основу проектирования высокоэффективных энергетических
    установок, медицинского диагностического оборудования и так далее.
    Исследование краевых задач сложного теплообмена представляет актуальную тему.
    Представим некоторые из публикаций ниже.


    Анализ обратных задач на моделях сложного теплообмена представлен не так широко.
    Отметим работы ...

    Численное решение задач сложного теплообмена представлено ...


    Целью диссертационной работы является теоретический и численный анализ обратных задач теплопереноса,
    в который входит установление разрешимости подобных задач, исследование полученных решений, их свойств
    и особенностей, разработка алгоритмов по численному решению поставленных задач, а также
    разработка соответствующего программного обеспечения которое включает в себя имплементацию алгоритмов по решению и
    графической презентации полученных решений.

    В основе теоретического исследования обратных задач лежит общая теория условно-корректных задач совместно
    с функциональным анализом.
    Разработка вычислительных алгоритмов основывается на современных методах компьютерного моделирования с применением
    множества технологий -- виртуализации, параллельных вычислений, облачных хранилищ и так далее.

    Представим ниже основные положения данной работы, а также приведём рассмотренные задачи по главам.

    В главе один изучается ...
    В главе два рассматривается ...
    Глава три настоящей работы
    в Главе четыре представлен

    Диссертация заканчивается списком цитируемой литературы.
    В алфавитном порядке приводятся сначала публикации на русском языке, затем на английском.










\end{document}
