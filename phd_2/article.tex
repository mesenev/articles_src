\documentclass[10pt]{article}
\usepackage[utf8]{inputenc}
\usepackage[english,russian]{babel}
\usepackage[T2A]{fontenc}
\usepackage{graphics}
\usepackage{graphicx}
\usepackage{float}
\usepackage{subfig}
\usepackage{amsmath}
\usepackage{amsfonts}
\usepackage{algorithm}
\captionsetup[algorithm]{labelformat=empty}
\usepackage[noend]{algpseudocode}


\graphicspath{{./experiments/results/}}

\begin{document}
    \section{Эксперименты}

    В качестве модели для численных экспериментов будем использовать
    \begin{equation}
        \label{initial}
        \begin{aligned}
            - a \Delta \theta + b \kappa_a(\theta ^ 3 | \theta | - \varphi) = 0,  \\
            - \alpha \Delta \varphi + \kappa_a (\varphi - \theta ^3 | \theta |) = 0
        \end{aligned}
        \qquad \text{в } \Omega.
    \end{equation}

    Уравнения дополняются граничными условиями
    \begin{equation}
        \label{initial_boundary}
        \begin{aligned}
            \theta|_{\partial \Omega} = \theta_b, \\
            \partial_n \theta |_{\partial \Omega} = q_b
        \end{aligned}
        \qquad \text{на } \partial \Omega.
    \end{equation}
    Функции $q_b, \theta_b$ являются известными.

    Найдя из~\eqref{initial}--\eqref{initial_boundary} функции состояния
    $\theta, \varphi$ мы можем также найти $\gamma$ из стандартного
    краевого условия для интенсивности излучения:
    \begin{equation}
        \label{initial_boundary_phi}
        a \partial_n \varphi + \gamma (\varphi -\theta_b^4) = 0.
    \end{equation}

    Решение краевой задачи заключается в нахождении функций $\theta, \varphi$,
    удовлетворяющих~\eqref{initial}--\eqref{initial_boundary}.
    Численное решение данной задачи затруднено в связи с необходимостью решать
    дифференциальное уравнение четвёртого порядка.
    Поиск решения задачи~\eqref{initial}--\eqref{initial_boundary}
    заменён на решение задачи оптимального управления с граничными условиями
    \begin{equation}
        \label{initial_boundary_2}
        a \partial_n \theta + \beta (\theta -\theta_0) = 0, \; \alpha \partial_n \varphi = u,
    \end{equation}
    и функционалом качества
    \begin{equation}
        \label{quality}
        J(\theta, u) = \frac{1}{2} \int_{\partial \Omega}
        (\theta - \theta_0)^2 ds + \frac{\varepsilon}{2} \int_{\partial \Omega} u^2 ds.
    \end{equation}

    Функция $\beta$ является известной. $\theta_0$ выражается
    через функции $q_b, \theta_b$ как $\theta_0 = \frac{1}{\beta}q_b + \theta_b$.

    Задача оптимального управления заключается в нахождении функции $\hat u \in L^2(\partial \Omega)$ такой, что
    \[J(\theta(\hat u), \hat u) = inf J(\theta, u).\]
    Функция $\theta(\hat u)$ является температурой, соответствующей управлению
    $\hat u$.~\eqref{quality} на решениях состоит в нахождении функций
    $u(s), s \in \partial \Omega,\; \theta(x), \varphi(x), x \in \Omega$ удовлетворяющих
    уравнениям~\eqref{initial}--\eqref{initial_boundary}, а также дополнительному условию на границе области:

    \begin{equation}
        \label{additional}
        \theta|_{\partial \Omega} = \theta_0
    \end{equation}
    где $\theta_0$ -- известная функция.
    Сформулированная обратная задача~\eqref{initial}--\eqref{additional} сводится к экстремальной задаче,
    состоящей в минимизации функционала

    На решениях краевой задачи~\eqref{initial}--\eqref{initial_boundary}.
    Решение задачи~\eqref{initial}--\eqref{initial_boundary},~\eqref{quality}
    называется квазирешением задачи~\eqref{initial}--\eqref{additional}.

    Для удобства введём переобозначение
    $\hat{J}(u):=J(\theta(u),u), \hat{J}:L^2(\partial \Omega) \to \mathbb{R}$.
    Здесь $\theta(u)$ -- температурное поле задачи~\eqref{initial}--\eqref{initial_boundary}
    отвечающее управлению $u \in L^2(\partial \Omega)$.
    Согласно формуле~\eqref{adjoint_3} градиент функционала $\hat J(u)$ имеет вид:
    \begin{equation}
        \label{j_gradient}
        \hat J'(u) = p_2 - \varepsilon u,
    \end{equation}
    где $p_2$ -- соответствующая компонента сопряжённой системы.

    Предлагаемый алгоритм выглядит следующим образом:
    \begin{algorithm}[H]
        \caption{Алгоритм градиентного спуска с проекцией}
        \begin{algorithmic}[1]
            \State Выбираем значение градиентного шага $\lambda$,
            \State Выбираем количество итераций $N$,
            \State Выбираем произвольное $u_0 \in U_{ad}$,
            \For{$k \gets 0,1,2,...,N$}:
            \State Для полученного $u_k$ рассчитываем состояние
            $y_k = \{\theta_k, \varphi_k\}$ из  (\ref{weak_operational}).
            \State Рассчитываем значение функционала качества $J(\theta_k)$ из (\ref{quality}).
            \State Рассчитываем сопряжённое состояние $p_k=\{p_{1k},p_{2k}\}$ из уравнений
            ~\eqref{therorem_2_eq1}--\eqref{therorem_2_eq2}, где $ \hat{\theta} := \theta_k, \hat{u}=u_k$.
            \State Пересчитываем управление $u_{k+1} = u_k - \lambda (p_2 - \varepsilon u)$
            \EndFor
        \end{algorithmic}
    \end{algorithm}

    Значение параметра $\lambda$ выбирается согласованным со значением градиента $J'(u_k)$
    таким образом, чтобы значение $\lambda(p_{2k}-\varepsilon u)$ определяло значимую поправку для $u_k$.
    В экспериментах, приведённых ниже, значение параметра $\lambda = 20$.

    Количество итераций $N$ выбирается достаточным для выполнения условия
    $J(\theta_{N-1}) - J(\theta_N) < 10^{-15}$. Эксперименты показывают хорошее восстановление функции $u$ при $N > 10^4$.

    Приведём примеры расчётов для трёхмерного случая.
    Положим $\Omega =\{(x,y,z), 0\leq x, y, z \leq l\}, l=1 \text{см}$.

    Параметры системы положим: $a = 0.006[\text{см}^2/\text{c}]$,
    $b=0.025[\text{см}/\text{с}]$, $\beta = 1[\text{см}/\text{с}]$
    , $\kappa=1[\text{см}^{-1}]$, $\kappa_s = 0$, $A = 0$.
    Указанные параметры соответствуют стеклу \cite{grenkin_13}.
    Температуру на границе $\Omega$ положим равной $\theta_b = (x+y+z^2)/4+0.06$.
    При указанных параметрах выберем тестовое значение $u = -0.11(x + y +z^2)/2$ (рис. \ref{fig:original})

    \begin{figure}[H]
        \label{fig:origial}
        \subfloat[Вертикальные грани куба]{\includegraphics{original_belt}} \\
        \subfloat[Верхняя грань]{\includegraphics[width= 3in]{original_top}}
        \subfloat[Нижняя грань]{\includegraphics[width = 3in]{original_bottom}}
        \caption{Значение $q_\varphi$ на $\partial \Omega$}
    \end{figure}


\end{document}
