\documentclass[12pt]{article}
\usepackage[russian]{babel}
\usepackage{amsmath}
\usepackage{amsfonts,amssymb}
\usepackage{graphicx}
\usepackage{subfig}
\usepackage{float}
\textwidth 16.5cm \textheight 23cm \topmargin -1.5cm
%\oddsidemargin -0.5cm
%\usepackage{algorithm}% http://ctan.org/pkg/algorithms
%\captionsetup[algorithm]{labelformat=empty}

\newtheorem{theorem}{Теорема}
\newtheorem{lemma}{Лемма}
\newtheorem{corollary}{Следствие}
\newtheorem{definition}{Определение}
\newtheorem{remark}{Замечание}
\newtheorem{problem}{Задача}

\renewcommand{\div}{\mathop{\mathrm{div}}\nolimits}

\begin{document}
    %\Large
    УДК 517.95
    \begin{center}
    {\bf ОПТИМИЗАЦИОННЫЙ МЕТОД ДЛЯ ЗАДАЧИ РАДИАЦИОННОГО ТЕПЛООБМЕНА С ГРАНИЧНЫМИ УСЛОВИЯМИ ТИПА КОШИ}
        \footnote[{1}]{$^)$Работа выполнена при финансовой поддержке РФФИ (проект 20-01-00113)}$^)$
    \end{center}
    \begin{center}
    {\bf \copyright\  2019 г.\ \  П.Р. Месенев, А.Ю. Чеботарев $^{*}$}
        \\
        {\it (690041 Владивосток,ул.Радио,7,ИПМ ДВО РАН)\\
        e-mail:  $^{*}$cheb@iam.dvo.ru}\\
        {\small  Поступила в редакцию 04.02.2020 г.\\
        Принята к публикации 2020 г.}
    \end{center}

    \sloppy
    \begin{quote}
        \small
        Исследована стационарная задача оптимального управления для
        уравнений
        радиационно-кондуктивного теплообмена в трехмерной
        области в рамках $P_1$--приближения уравнения
        переноса излучения. Показано, что
        последовательность решений
        экстремальных задач
        сходится к решению краевой задачи с условиями типа Коши для температуры.
        Результаты теоретического анализа проиллюстрированы численными примерами. Библ. 30.
    \end{quote}
    {\bf Ключевые слова:} уравнения радиационного теплообмена, диффузионное
    приближение, задача оптимального управления, условия типа Коши.

    \begin{center}
        \textbf{1. ВВЕДЕНИЕ. ПОСТАНОВКА ЗАДАЧИ ОПТИМАЛЬНОГО УПРАВЛЕНИЯ}
    \end{center}

    Стационарный радиационный и диффузионный теплообмен в
    ограниченной области $\Omega\subset \mathbb{R}^3$ с границей
    $\Gamma=\partial\Omega$ моделируется в рамках $P_1$--приближения для уравнения
    переноса излучения следующей
    краевой задачей~\cite{Modest,Kovt}:
    \begin{equation}
        \label{eq1}
        - a\Delta\theta + b\kappa_a(|\theta|\theta^3- \varphi)=0,   \quad
        -\alpha \Delta \varphi + \kappa_a(\varphi-|\theta|\theta^3)=0,\; x\in\Omega.
    \end{equation}
    \begin{equation}
        \label{bc1} a(\partial_n\theta+\theta) = r,\;\; \alpha(\partial_n\varphi+\varphi) = u \text{  на  }\Gamma.
    \end{equation}
    Здесь $\theta$ -- нормализованная температура, $\varphi$ --
    нормализованная интенсивность излучения, усредненная по всем
    направлениям.
    Положительные физические параметры $a$, $b$, $\kappa_a$ и $\alpha$, описывающие
    свойства среды, определяются стандартным образом \cite{Kovt}.
    Функция $r(x),\, x\in\Gamma$ является заданной, а неизвестная функция $u(x),\, x\in\Gamma$
    играет роль управления. Через $\partial_n$ обозначаем производную в направлении
    внешней нормали $\mathbf n$.

    Экстремальная задача заключается в отыскании тройки $\{\theta_\lambda,\varphi_\lambda,u_\lambda\}$
    такой, что
    \begin{equation}
        \label{cost}
        J_\lambda(\theta, u) = \frac{1}{2}\int\limits_\Gamma (\theta - \theta_b)^2d\Gamma + \frac{\lambda}{2}\int\limits_\Gamma u^2d\Gamma \rightarrow\inf
    \end{equation}
    на решениях краевой задачи (\ref{eq1}),(\ref{bc1}).
    Функция $\theta_b(x),\, x\in\Gamma$  и параметр регуляризации $\lambda>0$ заданы.

    Задача оптимального управления (\ref{eq1})--(\ref{cost}), если
    $r:=a(\theta_b+q_b)$, где $q_b$ -- заданная на $\Gamma$ функция,
    является при малых $\lambda$ аппроксимацией краевой задачи для уравнений (\ref{eq1}), в которой
    неизвестны краевые условия для интенсивности излучения $\varphi$, а ставятся
    граничные условия для температуры и тепловых потоков на границе,
    \begin{equation}
        \label{bc2}
        \theta|_\Gamma = \theta_b,\;\;
        \partial_n\theta|_\Gamma = q_b.
    \end{equation}


    В статьях \cite{Pinnau07}--\cite{JMAA-19} представлен анализ
    краевых и обратных задач, а также задач управления
    для уравнений радиационного теплообмена
    в рамках $P_1$--приближения для уравнения
    переноса излучения.
    Различные краевые задачи, связанных с радиационным теплообменом
    изучены в~\cite{AmosA05}--\cite{Amosov18}.
    Нелокальная разрешимость
    нестационарной и стационарной краевых задач для уравнений сложного теплообмена
    без краевых условий на интенсивность излучения и
    с условиями (\ref{bc2}) для температуры доказана в
    \cite{CNSNS19},\cite{CMMP20}.


    Основные результаты работы состоят в получении априорных оценок
    решения задачи (\ref{eq1}),(\ref{bc1}), на основе которых
    доказана разрешимость задачи оптимального управления
    (\ref{eq1})--(\ref{cost}) и выведена система оптимальности.
    Показано, что
    последовательность $\{\theta_\lambda,\varphi_\lambda,u_\lambda\}$ решений
    экстремальной задачи (\ref{eq1})--(\ref{cost}) при $\lambda\to +0$
    сходится к решению краевой задачи (\ref{eq1}),(\ref{bc2}) с условиями типа Коши для температуры.
    Результаты теоретического анализа проиллюстрированы численными примерами.


    \begin{center}
        \textbf{2. ФОРМАЛИЗАЦИЯ ЗАДАЧИ УПРАВЛЕНИЯ}
    \end{center}

    В дальнейшем считаем, что $\Omega\subset \mathbb{R}^3$~--- ограниченная строго липшицева
    область, граница $\Gamma$ которой состоит из конечного числа
    гладких кусков.
    Через $L^p$, $1 \leq p \leq \infty$ обозначаем
    пространство Лебега, а через $H^s$ -- пространство Соболева
    $W^s_2$. Пусть $H = L^2(\Omega), \; V = H^1(\Omega)$, через $V'$ обозначаем
    пространство, сопряженное с пространством $V$. Пространство $H$
    отождествляем с пространством $H'$, так что $V \subset H = H'
    \subset V'$.
    Обозначим через $\|\cdot\|$ стандартную норму
    в $H$, а через
    $(f,v)$ -- значение функционала $f\in V'$ на элементе $v\in V$,
    совпадающее со скалярным произведением в $H$, если $f\in H$.
    Через $U$ обозначаем пространство $L^2(\Gamma)$ с нормой
    $\|u\|_\Gamma=\left(\int_\Gamma u^2d\Gamma\right)^{1/2}.$



    Будем предполагать, что

    $(i) \;\; a,b,\alpha,\kappa_a, \lambda
    ={\rm Const}> 0 ,$

    $(ii) \;\, \theta_b, \,q_b \in U,\;\; r=a(\theta_b+q_b).$


    Определим операторы $A\colon V \to V'$, $B\colon U \to V'$, используя
    следующие равенства, справедливые для любых $y,z \in V$, $w\in U$:
    \[
        (Ay,z) = (\nabla y, \nabla z) +
        \int\limits_{\Gamma}yz d\Gamma, \;\;\; (Bw, z)
        = \int\limits_{\Gamma}wz d\Gamma.
    \]
    Билинейная форма $(Ay,z)$ определяет скалярное произведение
    в пространстве $V$, а соответствующая норма $\|z\|_V=\sqrt{(Az,z)}$ эквивалентна
    стандартной норме $V$. Поэтому определен непрерывный обратный оператор
    $A^{-1}:\,V'\mapsto V.$ Отметим справедливость, для любых
    $v\in V$, $w\in U$, $g\in V'$,
    следующих неравенств
    \begin{equation}
        \label{E}
        \|v\|^2\leq C_0\|v\|^2_V,\; \|v\|_{V'}\leq C_0\|v\|_V,\; \|Bw\|_{V'}\leq \|w\|_\Gamma,\;
        \|A^{-1}g\|_{V}\leq \|g\|_{V'}.
    \end{equation}
    Здесь постоянная $C_0>0$ зависит только от области $\Omega.$




    Далее используем следующее обозначение
    $[h]^s := |h|^s \mathrm{sign}\, h$,
    $s > 0$, $h \in \mathbb R$ для монотонной степенной функции.


    {\bf Определение.} Пара $\theta, \varphi\in V$
    называется {\it слабым решением} задачи (\ref{eq1}),(\ref{bc1}), если
    \begin{equation}
        \label{w1}
        a A \theta + b \kappa_a ([\theta]^4 - \varphi ) = Br,\quad
        \alpha A \varphi + \kappa_a (\varphi - [\theta]^4)  = Bu.
    \end{equation}

    Для формулировки задачи оптимального управления определим оператор
    ограничений $F(\theta, \varphi, u) : V \times V \times U \rightarrow V' \times V'$,
    \[
        F(\theta, \varphi, u) = \{ aA\theta + b \kappa_a ( [\theta]^4- \varphi) - Br,\;
        \alpha A \varphi + \kappa_a (\varphi -[\theta]^4) - Bu\}.
    \]


    \textbf{Задача $(CP)$.} Найти тройку $\{\theta, \varphi, u \} \in V \times V \times U$
    такую, что
    \begin{equation}
        \label{CP}
        J_\lambda(\theta, u) \equiv \frac{1}{2}\|\theta -\theta_b\|^2_\Gamma
        + \frac{\lambda}{2}\|u\|^2_\Gamma \rightarrow \text{inf},\;\; F(\theta, \varphi, u)=0.
    \end{equation}





    \begin{center}
        \textbf{3. РАЗРЕШИМОСТЬ ЗАДАЧИ $(CP)$}
    \end{center}

    Докажем предварительно однозначную разрешимость краевой задачи (\ref{eq1}),(\ref{bc1}).

    \textbf{Лемма 1.}
    {\it
    Пусть выполняются условия} (i),(ii), $u\in U$. {\it Тогда
    существует единственное слабое решение задачи (\ref{eq1}),(\ref{bc1}) и при этом}
    \begin{equation}
        \label{E1}
        \begin{aligned}
            a\|\theta\|_V \leq \|r\|_\Gamma + \frac{C_0\kappa_a}{\alpha}\|r+bu\|_\Gamma, \\
            \alpha b \|\varphi\|_V \leq \|r\|_\Gamma +
            \left(\frac{C_0\kappa_a}{\alpha} + 1\right)\|r+bu\|_\Gamma.
        \end{aligned}
    \end{equation}

    {\bf Доказательство.}
    Если второе уравнение в (\ref{w1}) умножить на $b$ и сложить с первым, то получим равенства
    \begin{gather*}
        A \left( a \theta + \alpha b \varphi \right) = B(r + bu),\;
        a\theta + \alpha b \varphi = A^{-1}B(r + bu),\;
        \varphi = \frac{1}{\alpha b}(A^{-1}B(r +bu) - a\theta).
    \end{gather*}
    Поэтому $\theta \in V$ является решением следующего уравнения:
    \begin{equation}
        \label{lemma-1-1}
        a A \theta + \frac{\kappa_a}{\alpha} \theta + b\kappa_a [\theta]^4 = g.
    \end{equation}
    Здесь \[ g = Br + \frac{\kappa_a}{\alpha}A^{-1}B(r+bu) \in V'. \]
    Однозначная разрешимость уравнения~\eqref{lemma-1-1} с монотонной нелинейностью
    хорошо известна (см. например \cite{Kufner}). Следовательно задача (\ref{w1})
    однозначно разрешима.

    Для получения оценок (\ref{E1}) умножим скалярно~\eqref{lemma-1-1} на $\theta \in V$ и отбросим неотрицательные
    слагаемые в левой части. Тогда
    \[
        a \|\theta\|^2_V \leq (g, \theta) \leq \|g\|_{V'}\|\theta\|_V,
        \quad a\|\theta\|_V \leq \|g\|_{V'}.
    \]
    Неравенства (\ref{E1}) позволяют оценить $\|g\|_{V'}$ и $\|\varphi\|_V $,
    \[
        \|g\|_{V'} \leq \|r\|_\Gamma + \frac{C_0\kappa_a}{\alpha}\|r + bu\|_\Gamma, \quad
        \|\varphi\|_V \leq \frac{1}{\alpha b} \|r + bu\|_\Gamma + \frac{a}{\alpha b} \|\theta\|_V.
    \]
    В результате получаем оценки \eqref{E1}.

    \textbf{Теорема 1.}
    {\it
    Пусть выполняются условия} (i),(ii).
    {\it Тогда существует решение задачи $(CP).$
    }

    {\bf Доказательство.}
    Пусть $j_\lambda = \inf J_\lambda$ на множестве $u \in U$, $F(\theta, \varphi, u)=0.$
    Выберем минимизирующую последовательность
    $u_m \in U, \; \theta_m \in V$, $J_\lambda(\theta_m, u_m)
    \rightarrow j_\lambda,$
    \begin{equation}
        \label{MS}
        a A \theta_m +b \kappa_a([\theta]^4 - \varphi_m) = Br, \;\;
        \alpha A \varphi_m + \kappa_a (\varphi_m - [\theta]^4) = B u_m.
    \end{equation}
    Из ограниченности последовательности $u_m$ в пространстве $U$ следуют, на основании
    леммы 1, оценки
    \[
        \|\theta_m\|_V \leq C,\;\;
        \|\varphi\|_V \leq C,\;\;\|\theta_m\|_{L^6(\Omega)} \leq C.
    \]
    Здесь через $C>0$ обозначены различные постоянные, не зависящие от $m$.
    Переходя при необходимости к подпоследовательностям, заключаем, что
    существует тройка $\{ \hat{u}, \hat{\theta}, \hat{\varphi} \} \in U \times V \times V,$
    \begin{equation}
        \label{L}
        u_m \rightarrow \hat{u} \text{  слабо в } U, \;\;
        \theta_m, \varphi_m \rightarrow \hat{\theta}, \hat{\varphi} \text{
        слабо в } V, \text{
        сильно в } L^4(\Omega).
    \end{equation}
    Заметим также, что $\forall v \in V$
    \begin{equation}
        \label{L1}
        |( [\theta_m]^4 - [\hat{\theta}]^4, v) \leq
        \leq 2 \| \theta_m - \hat{\theta}\|_{L^4(\Omega)} \|v\|_{L^4(\Omega)}
        \left( \| \theta_m \|^3_{L_6(\Omega)} + \| \hat{\theta} \|^3_{L_6(\Omega)}\right).
    \end{equation}
    Результаты о сходимости \eqref{L},\eqref{L1} позволяют перейти
    к пределу в~\eqref{MS}. Поэтому
    \[
        a A \hat{\theta} + b \kappa_a ([\hat{\theta}]^4 - \hat{\varphi} = Br), \;
        \alpha A \hat{\varphi} + \kappa_a (\hat{\varphi} -[\hat{\theta}]^4) = B \hat{u},
    \]
    и при этом $j_\lambda \leq J_\lambda(\hat{\theta}, \hat{u}) \leq \liminf J_\lambda(\theta_m, u_m) =
    j_\lambda$.
    Следовательно тройка $\{\hat{\theta}, \hat{\varphi}, \hat{u} \}$ есть
    решение задачи $(CP).$





    \begin{center}
        \textbf{4. УСЛОВИЯ ОПТИМАЛЬНОСТИ}
    \end{center}

    Для получения системы оптимальности достаточно использовать
    принцип Лагранжа для гладко-выпуклых экстремальных задач \cite{10,11}.
    Проверим справедливость ключевого условия, что образ производной
    оператора ограничений $F(y, u)$, где $y=\{\theta,\varphi\}\in V\times V$,
    совпадает с пространством $V'\times V'.$ Именно это условие гарантирует
    невырожденность условий оптимальности.
    Напомним, что
    \[
        F(y, u) = \{ aA\theta + b \kappa_a ( [\theta]^4- \varphi) - Br,\;
        \alpha A \varphi + \kappa_a (\varphi -[\theta]^4) - Bu\}.
    \]


    \textbf{Лемма 2.}
    {\it
    Пусть выполняются условия} (i),(ii).
    {\it Для любой пары $\hat{y} \in V \times V, \hat{u} \in U$ справедливо равенство}
    \[
        \texttt{Im}F_y'(y, u) = V' \times V'.
    \]


    {\bf Доказательство.} Достаточно проверить, что задача
    \[
        aA \xi + b \kappa_a (4|\hat{\theta}|^3 \xi - \eta) = f_1, ;\ \;
        \alpha A \eta + \kappa_a (\eta - 4|\hat{\theta}|^3 \xi) = f_2
    \]
    разрешима для всех $f_{1,2}\in V'.$ Данная задача равносильна системе
    \[
        aA\xi + \kappa_a\left(4b|\theta|^3 + \frac{a}{\alpha}\right) \xi = f_1
        +\frac{\kappa_a}{\alpha}f_3, ;\ \;
        \eta =\frac{1}{\alpha b}( f_3-a\xi).
    \]
    Здесь $f_3=A^{-1}(f_1+bf_2)\in V.$ Разрешимость первого уравнения указанной системы очевидным образом следует из леммы Лакса-Мильграма.


    В соответствии с леммой 2, лагранжиан задачи $(CP)$ имеет вид
    \[
        L(\theta, \varphi, u, p_1, p_2) = J_\lambda(\theta, u)
        + (aA\theta + b\kappa_a([\theta]^4 - \varphi) - Br, p_1)
        + (\alpha A \varphi + \kappa_a(\varphi - [\theta]^4) - Bu, p_2).
    \]
    Здесь $p=\{p_1,p_2\}\in V\times V$ -- сопряженное состояние.
    Если $\{\hat{\theta}, \hat{\varphi}, \hat{u} \}$ -- решение задачи $(CP)$, то
    в силу принципа Лагранжа \cite[Теорема 1.5]{10} справедливы вариационные равенства
    $\forall v\in V,\, w\in U$
    \begin{equation}
        \label{OC1}
        (\hat{\theta} -\theta_b, v)_\Gamma + (aAv + 4 b\kappa_a |\hat{\theta}|^3v, p_1)
        - \kappa_a ( 4 |\hat{\theta}|^3v, p_2) = 0,\;
        b \kappa_a (v, p_1)+ (\alpha A v + \kappa_a v, p_2) = 0,
    \end{equation}
    \begin{equation}
        \label{OC2}
        \lambda(\hat{u},w)_\Gamma - (Bw, p_2) = 0.
    \end{equation}
    Таким образом, из условий \eqref{OC1},\eqref{OC2}
    получаем следующий результат

    \textbf{Теорема 2.}
    {\it Пусть выполняются условия} (i),(ii).
    {\it Если $\{\hat{\theta}, \hat{\varphi}, \hat{u}\}$ -- решение
    задачи $(CP)$, то существует единственная пара $\{p_1, p_2 \} \in V\times V$
    такая, что}
    \begin{gather*}
        aA\hat{\theta} + b \kappa_a([\hat{\theta}]^4 -\hat{\varphi}) = Br, \;\;
        \alpha A \hat{\varphi} + \kappa_a(\hat{\varphi} - [\hat{\theta}]^4) = B\hat{u},\\
        aAp_1 +4|\hat{\theta}|^3 \kappa_a(bp_1 - p_2) = B(\theta_b - \hat{\theta}), \;\;
        \alpha A p_2 + \kappa_a (p_2 - b p_1)\\
    \end{gather*}

    {\it и при этом} $\lambda\hat{u} = p_2.$




    \begin{center}
        \textbf{5. АППРОКСИМАЦИЯ ЗАДАЧИ С УСЛОВИЯМИ ТИПА КОШИ}
    \end{center}


    Рассмотрим краевую задачу для уравнений сложного теплообмена, в которой нет краевых условий на
    интенсивность излучения.
    \begin{equation}
        \label{eq11}  - a\Delta\theta + b\kappa_a([\theta]^4- \varphi)=0,\quad
        -\alpha \Delta \varphi +
        \kappa_a(\varphi-[\theta]^4)=0,\; x\in\Omega.
    \end{equation}
    \begin{equation}
        \label{bc11} \theta=\theta_b,\quad \partial_n\theta = q_b \text{  на  }\Gamma.
    \end{equation}
    Существование $\theta,\varphi\in H^2(\Omega)$, удовлетворяющих \eqref{eq11},\eqref{bc11}
    для достаточно гладких
    $\theta_b,\, q_b$ и достаточные условия единственности решения
    установлены в \cite{CMMP20}.
    Покажем, что решения задачи $(CP)$ при $\lambda\to+0$
    аппроксимируют решение задачи~\eqref{eq11},\eqref{bc11}.


    \textbf{Теорема 3.}
    {\it
    Пусть выполняются условия} (i),(ii) {\it и существует решение задачи \eqref{eq11},\eqref{bc11}.}
    {\it  Если $\{\theta_\lambda,\varphi_\lambda,u_\lambda\}$ -- решение
    задачи $(CP)$ для $\lambda>0$, то существует последовательность
    $\lambda\to +0$
    такая, что}
    \[
        \theta_\lambda\rightarrow\theta_*, \;\; \varphi_\lambda\rightarrow\varphi_*
        \text{ слабо в }V,\text{ сильно в }H,
    \]
    {\it где $\theta_*,\varphi_*$ -- решение задачи \eqref{eq11},\eqref{bc11}.}

    {\bf Доказательство.}
    Пусть $\theta,\varphi\in H^2(\Omega)$ -- решение задачи \eqref{eq11},\eqref{bc11},
    $u=\alpha(\partial_n\varphi+\varphi)\in U.$ Тогда
    \[
        a A \theta + b \kappa_a ([\theta]^4 - \varphi ) = Br,\quad
        \alpha A \varphi + \kappa_a (\varphi - [\theta]^4)  = Bu,
    \]
    где $r:=a(\theta_b+q_b).$ Поэтому, с учетом того, что $\theta|_\Gamma=\theta_b$,
    \[
        J_\lambda(\theta_\lambda, u_\lambda) = \frac{1}{2}\|\theta_\lambda -\theta_b\|^2_\Gamma
        + \frac{\lambda}{2}\|u_\lambda\|^2_\Gamma\leq J_\lambda(\theta, u)=\frac{\lambda}{2}\|u\|^2_\Gamma.
    \]
    Следовательно,
    \[
        \|u_\lambda\|^2_\Gamma\leq C,\;\; \|\theta_\lambda -\theta_b\|^2_\Gamma\to 0,\; \lambda\to +0.
    \]
    Здесь и далее $C>0$ не зависит от $\lambda.$
    Из ограниченности последовательности $u_\lambda$ в пространстве $U$ следуют, на основании
    леммы 1, оценки
    \[
        \|\theta_\lambda\|_V \leq C,\;\;
        \|\varphi\|_\lambda \leq C.
    \]
    Поэтому можно выбрать последовательность $\lambda\to+0$ такую, что
    \begin{equation}
        \label{LL}
        u_\lambda \rightarrow u_* \text{  слабо в } U, \;\;
        \theta_\lambda, \varphi_\lambda \rightarrow \theta_*,\varphi_* \text{
        слабо в } V, \text{
        сильно в } L^4(\Omega).
    \end{equation}
    Результаты \eqref{LL} позволяют перейти к пределу при $\lambda\to+0$
    в уравнениях для $\theta_\lambda,\varphi_\lambda,u_\lambda$ и тогда
    \begin{equation}
        \label{CC}
        a A \theta_* + b \kappa_a ([\theta_*]^4 - \varphi_* ) = Br,\quad
        \alpha A \varphi_* + \kappa_a (\varphi_* - [\theta_*]^4)  = Bu_*.
    \end{equation}
    При этом $\theta_*|_\Gamma=\theta_b.$
    Из первого уравнения в \eqref{CC}, с учетом, что $r=a(\theta_b+q_b)$,
    выводим
    \[
        - a\Delta\theta_* + b\kappa_a([\theta_*]^4- \varphi_*)=0 \text{ п.в. в }\Omega,
        \quad \theta_*=\theta_b,\quad \partial_n\theta = q_b \text{ п.в. на  }\Gamma.
    \]
    Из второго уравнения в \eqref{CC} следует, что $-\alpha \Delta \varphi +
    \kappa_a(\varphi-[\theta]^4)=0$ почти всюду в $\Omega.$ Таким образом,
    пара $\theta_*,\varphi_*$ -- решение задачи~\eqref{eq11},\eqref{bc11}.




    \begin{center}
        \textbf{6. ЧИСЛЕННЫЕ ЭКСПЕРИМЕНТЫ}
    \end{center}
    Пусть функционал $J_\lambda (\theta, u)$ соответсвует оговорённому ранее.
    Согласно формуле~\eqref{OC2} градиент функционала качества равен
    \[
        J'_\lambda (u) = \lambda u - p_2.
    \]

    Предлагаемый алгоритм решения задачи (CP) выглядит следующим образом:
    %    \begin{algorithm}[H]
    %        \caption{Алгоритм градиентного спуска}
    %        \begin{algorithmic}[1]
    %            \State Выбираем значение градиентного шага $\varepsilon$,
    %            \State Выбираем количество итераций $N$,
    %            \State Выбираем произвольное $u_0 \in U$,
    %            \For{$k \gets 0,1,2,\dots,N$}:
    %            \State Для полученного $u_k$ расчитываем состояние $y_k = \{\theta_k, \varphi_k\}$ из~\eqref{eq1}.
    %            \State Расчитываем значение функционала качества $J(\theta_k, u_k)$ из~\eqref{cost}.
    %            \State Расчитываем сопряжённое состояние $p_k=\{p_{1k},p_{2k}\}$ из уравнений~\eqref{OC1}, где $ \hat{\theta} := \theta_k, \hat{u}=u_k$.
    %            \State Пересчитываем управление $u_{k+1} = u_k - \varepsilon (\lambda u_k - p_2)$
    %            \EndFor
    %        \end{algorithmic}
    %    \end{algorithm}
    Значение параметра $\varepsilon$ выбирается эмпирически таким образом, чтобы значение
    $\varepsilon (\lambda u_k - p_2)$ являлась существенной поправкой для $u_{k+1}$.
    Количество итераий $N$ выбирается достаточным для выполнения условия
    $J(\theta_k, u_k) - J(\theta_{k+1}, u_{k+1}) < 10^{-15}$.
    Эксперименты показывают хорошее восстановление функции $u$ при $N > 10^{3}$.

    Приведём примеры расчётов для стеклянного куба с ребром 1~см,
    $\Omega = {(x, y, z), 0 \leq x,y,z \leq l}, l = 1 \text{см}$.
    Будем также далее считать, что $a = 0.006[\text{см}^2/\text{c}]$,
    $b=0.025[\text{см}/\text{с}]$, $\beta = 0.00005[\text{см}/\text{с}]$,
    $\kappa=1[\text{см}^{-1}]$, $\kappa_s = 0$, $A = 0$, $\gamma = 0.3$.
    Указанные параметры соответствуют стеклу~\cite{grenkin_13}.

    Для первого теста определим параметры $r$ и $u$ как:
    $u = u^* = y$.

    $\theta_b$ положим равным $\theta |_\Gamma$ из решения прямой задачи (1)-(2).
    Положим начальное значение $u_0 = 0.1$.
    Далее, применяя предложенный алгоритм находим приближенное решение обратной задачи (1)–(4).
    Тестовое и найденное значение $u$ представлено на рис. 1
    Эффективность алгоритма, а также значение $u_0$ в первом и втором случаях иллюстрируются рис. 1.

    \begin{figure}[H]
        \centering
        \subfloat[Тестовое значение]
        {
        \label{fig1:exp1}
        \includegraphics[width=.51\linewidth]{img/test_control_1.eps}
        }
        \subfloat[Найденное значение]
        {
        \label{fig1:exp2}
        \includegraphics[width=.51\linewidth]{img/final_control_1.eps}
        }
        \caption{Тестовая функция $u$, начальная $u_0$, найденная функция $u_{end}.$}
        \label{img_control}
    \end{figure}

    На рисунке 2 показана динамика функционала качества по итерациям.

    \begin{figure}[H]
        \centering
        \subfloat[Первый эксперимент]
        {
        \label{fig2:exp1}
        \includegraphics[width=.51\linewidth]{img/quality_1.png}
        }
        \subfloat[Второй эксперимент]
        {
        \label{fig2:exp2}
        \includegraphics[width=.51\linewidth]{img/quality_1.png}
        }
        \caption{Динамика функции $\hat{J}(\theta, u)$ по итерациям.}
        \label{img_cost}

        Для второго теста определим $\partial_n \theta$ и $\theta_b$ из (4) следующим образом
        $\partial_n \theta = ..., \; \theta_b = ...$
        В данном случае мы не знаем решение $u$.


    \end{figure}



    \begin{thebibliography}{999}

        \bibitem{Modest}
        Modest~M.F. Radiative Heat Transfer. Academic Press. 2003. 822 p.

        \bibitem{Kovt}
        Kovtanyuk A.E., Chebotarev A.Yu., Botkin N.D. Unique solvability of a steady-state complex heat
        transfer model // Commun. Nonlinear Sci. Numer. Simul. 2015. V.~20. N 3. P. 776--784.

        \bibitem{Pinnau07}
        R.~Pinnau. Analysis of Optimal Boundary Control for Radiative Heat
        Transfer Modelled by the SP$_1$-System~// Comm. Math. Sci. 2007. V.~5.
        N~4. P.~951--969.

        \bibitem{Tse12}
        Tse O., Pinnau R., Siedow N. Identification of temperature
        dependent parameters in laser--interstitial thermo therapy~//
        Math. Models Methods Appl. Sci. 2012. V.~22. N~9. P. 1--29.

        \bibitem{Kovt13}
        Kovtanyuk~A.E., Chebotarev~A.Yu. An iterative method for solving a
        complex heat transfer problem~// Appl. Math. Comput. 2013. V.~219.
        P.~9356--9362.

        \bibitem{Cheb14-1}
        Kovtanyuk~A.E., Chebotarev~A.Yu., Botkin~N.D., Hoffmann~K.-H. The
        unique solvability of a complex 3D heat transfer problem~// J.
        Math. Anal. Appl. 2014. V.~409. N~2. P.~808-815.

        \bibitem{JVM-14}
        Ковтанюк А.Е., Чеботарев А.Ю. Стационарная задача сложного теплообмена // Ж. вычисл. матем. и матем. физ. 2014. Т. 54. \textnumero~4. С. 711--719.

        \bibitem{Kovt14-2}
        Ковтанюк А.Е., Чеботарев А.Ю. Стационарная задача свободной конвекции с радиационным теплообменом // Дифференц. ур-ния. 2014. Т. 50. \textnumero~12. С. 1590--1597.

        \bibitem{Kovt14-3}
        Kovtanyuk A.E., Chebotarev A.Yu., Botkin N.D., Hoffmann K.-H. Theoretical analysis of an optimal control problem of conductive-convective-radiative heat transfer // J. Math. Anal.
        Appl. 2014. V. 412. \textnumero~1. P. 520--528.

        \bibitem{Grenkin1}
        Гренкин Г.В., Чеботарев А.Ю. Нестационарная задача сложного теплообмена // Ж. вычисл. матем. и матем. физ. 2014. Т. 54. \textnumero~11. С. 1806--1816.

        \bibitem{Grenkin3}
        Гренкин Г.В., Чеботарев А.Ю. Неоднородная нестационарная задача сложного теплообмена // Сиб. электрон. матем. изв. 2015. Т. 12. С. 562--576.

        \bibitem{Grenkin4}
        Гренкин Г.В., Чеботарев А.Ю. Нестационарная задача свободной конвекции с радиационным теплообменом // Ж. вычисл. матем. и матем. физ. 2016. Т. 56. \textnumero~2. С. 275--282.

        \bibitem{Grenkin5}
        Grenkin G.V., Chebotarev A.Yu., Kovtanyuk A.E., Botkin N.D., Hoffmann K.-H. Boundary optimal control problem of complex heat transfer model // J. Math. Anal. Appl. 2016. V. 433. \textnumero~2. P. 1243--1260.

        \bibitem{JMAA-16}
        Kovtanyuk A.E., Chebotarev A.Yu., Botkin N.D., Hoffmann K.-H. Optimal boundary control of a steady-state heat transfer model accounting for radiative effects // J. Math. Anal. Appl.
        2016. V. 439. \textnumero~2. P. 678--689.

        \bibitem{AMC-16}
        Chebotarev A.Yu., Kovtanyuk A.E., Grenkin G.V., Botkin N.D., Hoffmann K.-H.
        Nondegeneracy of optimality conditions in control problems for a radiative-conductive heat transfer model // Appl. Math. Comput. 2016. V. 289. P. 371--380.

        \bibitem{JVM-16}
        Ковтанюк А.Е., Чеботарев А.Ю. Нелокальная однозначная разрешимость стационарной задачи сложного теплообмена // Ж. вычисл. матем. и матем. физ. 2016. Т. 56. \textnumero~5. С. 816--823.

        \bibitem{ESAIM}
        Chebotarev A.Yu., Grenkin G.V., Kovtanyuk A.E. Inhomogeneous steady-state problem of complex heat transfer // ESAIM Math. Model. Numer. Anal. 2017. V. 51. \textnumero~6. P. 2511--2519.

        \bibitem{JMAA-18}
        Chebotarev A.Yu., Grenkin G.V., Kovtanyuk A.E., Botkin N.D., Hoffmann K.-H. Inverse problem with finite overdetermination for steady-state equations of radiative heat exchange // J. Math. Anal. Appl. 2018. V. 460. \textnumero~2. P. 737--744.

        \bibitem{JMAA-19}
        Chebotarev A.Yu., Pinnau R. An inverse problem for a quasi-static approximate model of radiative heat transfer // J. Math. Anal. Appl. 2019. V. 472. \textnumero~1. P. 737--744.


        \bibitem{AmosA05}
        Амосов~А.А. Глобальная разрешимость одной нелинейной
        нестационарной задачи с нелокальным краевым условием типа
        теплообмена излучением~// Дифференц. ур-ния. Т.~41. N~1.
        2005. С.~93--104.

        \bibitem{AmosA10}
        Amosov~A.A. Stationary nonlinear nonlocal problem of
        radiative-conductive heat transfer in a system of opaque bodies
        with properties depending on the radiation frequency~// J.
        Math. Sc. 2010. V.~164. N~3. P.~309--344.

        \bibitem{Amosov16}
        Amosov A. Unique Solvability of a Nonstationary Problem of Radiative - Conductive
        Heat Exchange in a System of Semitransparent Bodies // Russian J. of Math.
        Phys. 2016. V. 23, N~3. P.~309-334.

        \bibitem{Amosov17}
        Амосов А.А. Стационарная задача сложного теплообмена в системе полупрозрачных тел с краевыми условиями диффузного отражения и преломления излучения // Ж. вычисл. матем. и матем. физ. 2017. Т. 57. \textnumero~3. С. 510--535.

        \bibitem{Amosov17-1}
        Amosov A.A. Unique Solvability of Stationary Radiative - Conductive Heat Transfer
        Problem in a System of Semitransparent Bodies // J. of Math. Sc.(United
        States). 2017. V. 224. N~5. P.~618-646.

        \bibitem{Amosov18}
        Amosov A.A. Nonstationary problem of complex heat transfer in a system of
        semitransparent bodies with boundary-value conditions of diffuse reflection and refraction
        of radiation // J. of Math. Sc.(United
        States). 2018. V. 233. N~6. P.~777 -806.

        \bibitem{CNSNS19}
        Chebotarev A.Yu., Kovtanyuk A.E., Botkin N.D. Problem of radiation heat exchange with boundary conditions of the Cauchy type // Commun. Nonlinear Sci. Numer. Simul. 75 (2019) 262-269.

        \bibitem{CMMP20}
        Колобов А.Г., Пак Т.В., Чеботарев А.Ю. Стационарная задача радиационного теплообмена с граничными условиями типа Коши // Журнал вычислительной математики и математической физики. 2019, том 59, № 7, с. 1258-1263.


        \bibitem{Kufner} S.~Fu\v{c}ik, A.~Kufner, Nonlinear differential equations,
        Elsevier, Amsterdam--Oxford--New York, 1980.

        \bibitem{10} A.V.~Fursikov, Optimal Control of Distributed
        Systems. Theory and Applications, American Math. Soc., 2000.

        \bibitem{11} A.D.~Ioffe, V.M.~Tikhomirov, Theory of Extremal
        Problems, North-Holland, Amsterdam, 1979.

    \end{thebibliography}


\end{document}
