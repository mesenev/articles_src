\documentclass[12pt]{article}
\usepackage[russian]{babel}
\usepackage{amsmath}
\usepackage{amsfonts,amssymb}
\usepackage{mathtools}
\usepackage{graphicx}
\usepackage{xcolor}
\selectcolormodel{gray}
\usepackage{subfig}
\textwidth 16.5cm \textheight 23cm \topmargin -1.5cm
\usepackage{algorithm}% http://ctan.org/pkg/algorithms
\captionsetup[algorithm]{labelformat=empty}
\usepackage[noend]{algpseudocode}
\usepackage{hyperref}

\newtheorem{theorem}{Теорема}
\newtheorem{lemma}{Лемма}
\newtheorem{corollary}{Следствие}
\newtheorem{definition}{Определение}
\newtheorem{remark}{Замечание}
\newtheorem{problem}{Задача}

\renewcommand{\div}{\mathop{\mathrm{div}}\nolimts}

\begin{document}
    %\Large
    УДК 517.95
    \begin{center}
    {\bf анализ оптимизационного метода решения задачи сложного теплообмена с граничными условиями типа коши}
        \footnote[{1}]{$^)$Работа выполнена при финансовой поддержке РФФИ (проект 20-01-00113)}$^)$
    \end{center}
    \begin{center}
    {\bf \copyright\  2020 г.\ \  П.Р. Месенев, А.Ю. Чеботарев $^{*}$}
        \\
        {\it (690041 Владивосток,ул.Радио,7,ИПМ ДВО РАН)\\
        e-mail:  $^{*}$cheb@iam.dvo.ru}\\
        {\small  Поступила в редакцию 30.10.2020 г.\\
        Принята к публикации }
    \end{center}

    \sloppy
    \begin{quote}
        \small
        Предложен оптимизационный метод решения краевой задачи с условиями типа Коши
        для уравнений радиационно-кондуктивного теплообмена в рамках $P_1$--приближения
        уравнения переноса излучения. Выполнен теоретический анализ соответствующей задачи
        граничного оптимального управления.
        Показано, что последовательность решений экстремальных задач
        сходится к решению краевой задачи с условиями типа Коши для температуры.
        Результаты теоретического анализа проиллюстрированы численными примерами.
        Библ.
        32.
    \end{quote}
    {\bf Ключевые слова:} уравнения радиационно-кондуктивного теплообмена, диффузионное
    приближение, задача оптимального управления, условия типа Коши.

    \begin{center}
        \textbf{1. ВВЕДЕНИЕ}
    \end{center}

    \begin{center}
        \textbf{2. ФОРМАЛИЗАЦИЯ ЗАДАЧИ УПРАВЛЕНИЯ}
    \end{center}

    \begin{center}
        \textbf{3. РАЗРЕШИМОСТЬ ЗАДАЧИ $(CP)$}
    \end{center}




    \begin{center}
        \textbf{5. }
    \end{center}



    \begin{thebibliography}{999}

        \bibitem{Pinnau07}
        Pinnau R. Analysis of optimal boundary control for radiative heat transfer modeled by $SP_1$-system //
        Commun. Math. Sci. 2007. V. 5. \textnumero~4. P. 951--969.

        \bibitem{AMC-13}
        Kovtanyuk A.E., Chebotarev A.Yu. An iterative method for solving a complex heat transfer problem // Appl. Math. Comput. 2013. V. 219. \textnumero~17. P. 9356--9362.
        \bibitem{Kovt14-1}
        Kovtanyuk A.E., Chebotarev A.Yu., Botkin N.D., Hoffmann K.-H. The unique solvability of a complex 3D heat transfer problem // J. Math. Anal. Appl. 2014. V. 409. \textnumero~2. P. 808--815.
        \bibitem{JVM-14}
        Ковтанюк А.Е., Чеботарев А.Ю. Стационарная задача сложного теплообмена // Ж. вычисл. матем. и матем. физ. 2014. Т. 54. \textnumero~4. С. 711--719.
        \bibitem{Kovt14-2}
        Ковтанюк А.Е., Чеботарев А.Ю. Стационарная задача свободной конвекции с радиационным теплообменом // Дифференц. ур-ния. 2014. Т. 50. \textnumero~12. С. 1590--1597.
        \bibitem{Kovt14-3}
        Kovtanyuk A.E., Chebotarev A.Yu., Botkin N.D., Hoffmann K.-H. Theoretical analysis of an optimal control problem of conductive-convective-radiative heat transfer // J. Math. Anal.
        Appl. 2014. V. 412. \textnumero~1. P. 520--528.
        \bibitem{Grenkin1}
        Гренкин Г.В., Чеботарев А.Ю. Нестационарная задача сложного теплообмена // Ж. вычисл. матем. и матем. физ. 2014. Т. 54. \textnumero~11. С. 1806--1816.
        \bibitem{CNSNS-15}
        Kovtanyuk A.E., Chebotarev A.Yu., Botkin N.D., Hoffmann K.-H. Unique solvability of a steady-state complex heat transfer model // Commun. Nonlinear Sci. Numer. Simul. 2015. V. 20. \textnumero~3. P. 776--784.

        \bibitem{Grenkin3}
        Гренкин Г.В., Чеботарев А.Ю. Неоднородная нестационарная задача сложного теплообмена // Сиб. электрон. матем. изв. 2015. Т. 12. С. 562--576.
        \bibitem{Grenkin4}
        Гренкин Г.В., Чеботарев А.Ю. Нестационарная задача свободной конвекции с радиационным теплообменом // Ж. вычисл. матем. и матем. физ. 2016. Т. 56. \textnumero~2. С. 275--282.
        \bibitem{Grenkin5}
        Grenkin G.V., Chebotarev A.Yu., Kovtanyuk A.E., Botkin N.D., Hoffmann K.-H. Boundary optimal control problem of complex heat transfer model // J. Math. Anal. Appl. 2016. V. 433. \textnumero~2. P. 1243--1260.



        \bibitem{JMAA-16}
        Kovtanyuk A.E., Chebotarev A.Yu., Botkin N.D., Hoffmann K.-H. Optimal boundary control of a steady-state heat transfer model accounting for radiative effects // J. Math. Anal. Appl.
        2016. V. 439. \textnumero~2. P. 678--689.
        \bibitem{AMC-16}
        Chebotarev A.Yu., Kovtanyuk A.E., Grenkin G.V., Botkin N.D., Hoffmann K.-H\@.
        Nondegeneracy of optimality conditions in control problems for a radiative-conductive heat transfer model // Appl. Math. Comput. 2016. V. 289. P. 371--380.
        \bibitem{JVM-16}
        Ковтанюк А.Е., Чеботарев А.Ю. Нелокальная однозначная разрешимость стационарной задачи сложного теплообмена // Ж. вычисл. матем. и матем. физ. 2016. Т. 56. \textnumero~5. С. 816--823.
        \bibitem{ESAIM}
        Chebotarev A.Yu., Grenkin G.V., Kovtanyuk A.E. Inhomogeneous steady-state problem of complex heat transfer // ESAIM Math. Model. Numer. Anal. 2017. V. 51. \textnumero~6. P. 2511--2519.

        \bibitem{CNSNS18}
        Chebotarev, A.Y., Grenkin, G.V., Kovtanyuk, A.E., Botkin, N.D., Hoffmann, K.-H. Diffusion approximation of the radiative-conductive heat transfer model with Fresnel matching conditions// Communications in Nonlinear Science and Numerical Simulation. 57 (2018) 290-298.

        \bibitem{JMAA-18}
        Chebotarev A.Yu., Grenkin G.V., Kovtanyuk A.E., Botkin N.D., Hoffmann K.-H. Inverse problem with finite overdetermination for steady-state equations of radiative heat exchange // J. Math. Anal. Appl. 2018. V. 460. \textnumero~2. P. 737--744.
        \bibitem{JMAA-19}
        Chebotarev A.Yu., Pinnau R. An inverse problem for a quasi-static approximate model of radiative heat transfer // J. Math. Anal. Appl. 2019. V. 472. \textnumero~1. P. 314--327.
        \bibitem{JVM-19-INV} Гренкин Г.В., Чеботарев А.Ю. Обратная задача для уравнений сложного теплообмена // Журнал вычислительной математики и математической физики. 2019, том 59, N 8, с. 1420-1430.

        \bibitem{CNSNS19}
        Chebotarev A.Y., Kovtanyuk A.E., Botkin N.D. Problem of radiation heat exchange with boundary conditions of the Cauchy type // Communications in Nonlinear Science and Numerical Simulation. 75 (2019) 262-269.

        \bibitem{CMMP20}
        Колобов А.Г., Пак Т.В., Чеботарев А.Ю.
        Стационарная задача радиационного теплообмена с граничными условиями типа Коши
        // Журнал вычислительной математики и математической физики.
        2019, том 59, N 7, с. 1258--1263.

        \bibitem{Amosov05}
        Амосов~А.А. Глобальная разрешимость одной нелинейной
        нестационарной задачи с нелокальным краевым условием типа
        теплообмена излучением~// Дифференц. ур-ния. Т.~41. N~1.
        2005. С.~93--104.

        \bibitem{Amosov10}
        Amosov~A.A. Stationary nonlinear nonlocal problem of
        radiative-conductive heat transfer in a system of opaque bodies
        with properties depending on the radiation frequency~// J\@.
        Math. Sc. 2010. V.~164. N~3. P.~309--344.

        \bibitem{Amosov16}
        Amosov A. Unique Solvability of a Nonstationary Problem of Radiative - Conductive
        Heat Exchange in a System of Semitransparent Bodies // Russian J. of Math.
        Phys. 2016. V. 23, N~3. P.~309-334.

        \bibitem{Amosov17}
        Amosov A.A. Unique Solvability of Stationary Radiative - Conductive Heat Transfer
        Problem in a System of Semitransparent Bodies // J. of Math. Sc.(United
        States). 2017. V. 224. N~5. P.~618-646.

        \bibitem{Amosov20-1}
        Amosov, A.A. Asymptotic Behavior of a Solution to the Radiative Transfer Equation in a Multilayered Medium with Diffuse Reflection and Refraction Conditions // J Math Sci 244, 541-575 (2020).

        \bibitem{Amosov20}
        Amosov A.A., Krymov N.E. On a Nonstandard Boundary Value Problem Arising in Homogenization of Complex Heat Transfer Problems // J. of Math. Sc.(United
        States). 2020. V. 244. P.~357 - 377.

        \bibitem{Kufner} S.~Fu\v{c}ik, A.~Kufner, Nonlinear differential equations,
        Elsevier, Amsterdam--Oxford--New York, 1980.

        \bibitem{10} A.V.~Fursikov, Optimal Control of Distributed
        Systems. Theory and Applications, American Math. Soc., 2000.

        \bibitem{11} A.D.~Ioffe, V.M.~Tikhomirov, Theory of Extremal
        Problems, North-Holland, Amsterdam, 1979.

        \bibitem{fenics} M. S. Alnaes, J. Blechta, J. Hake, A. Johansson,
        B. Kehlet, A. Logg, C. Richardson, J. Ring, M. E. Rognes, G. N. Wells
        The FEniCS Project Version 1.5
        Archive of Numerical Software, vol. 3, 2015,

        \bibitem{dolfin} A. Logg and G. N. Wells, DOLFIN: Automated Finite Element Computing
        ACM Transactions on Mathematical Software, vol. 37, 2010

        \bibitem{mesenev-github} \url{https://github.com/mesenev/articles_src}


    \end{thebibliography}
\end{document}
