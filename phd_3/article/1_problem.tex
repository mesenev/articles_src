\section{Введение}\label{sec:intro}
Стационарный радиационный и диффузионный теплообмен в ограниченной области
$\Omega \subset \mathbb{R}^3$ с границей $\Gamma = \partial \Omega$ моделируется в рамках  $P_1$--приближения для уравнения
переноса излучения следующей системой эллиптических уравнений~\cite{Pinnau07, AMC-13, Kovt14-1}:
\begin{equation}
    \label{eq:model}
    - a \Delta\theta + b\kappa_a(|\theta|\theta^3- \varphi)=0,   \quad
    - \alpha \Delta \varphi + \kappa_a(\varphi-|\theta|\theta^3)=0,\; x\in\Omega.
\end{equation}
Здесь $\theta$ -- нормализованная температура, $\varphi$ -- нормализованная интенсивность излучения,
усредненная по всем направлениям.
Положительные физические параметры $a$, $b$, $\kappa_a$ и $\alpha$, описывающие
свойства среды, определяются стандартным образом~\cite{Kovt14-1}.
Подробный теоретический и численный анализ различных постановок краевых и обратных задач,
а также задач управления для уравнений радиационного теплообмена
в рамках $P_1$--приближения для уравнения переноса излучения представлен в~\cite{Pinnau07}--\cite{CMMP20}.
Отметим также серьезный анализ интересных краевых задач, связанных с радиационным теплообменом,
представленный в~\cite{Amosov05}--\cite{Amosov20}.

Пусть $\Gamma \coloneqq \partial \Omega =\overline{\Gamma}_1 \cup \overline{\Gamma}_2$,
где части границы не имеют пересечений, $\Gamma_1 \cap \Gamma_2 =  \emptyset$.
На участках $\Gamma_1, \Gamma_2$ заданы краевые условия

Определим следующие функции
\begin{equation}
    \label{eq:boundary}
    \begin{aligned}
        \Gamma_1 :\; &a \partial_n \theta = q_b, \\
        &\alpha\partial_n\varphi + \gamma (\varphi - \theta_b ^4 ) = 0, \\
        \Gamma_2 :\; & a \partial_n \theta = q_b.
    \end{aligned}
\end{equation}
Краевое условие для интенсивности излучения на $\Gamma_2$ неизвестны, но ставится дополнительное
граничное условие для температуры на $\Gamma_1$.
%\[
%    r \coloneqq \alpha \partial_n \varphi \text{ на }\Gamma_2, \\
%    \tilde{r} \coloneqq \alpha \partial_n \varphi \text{ на }\Gamma_1.
%\]

Обратная задача состоит в отыскании тройки функций $\{r, \theta, \varphi\}$
удовлетворяющих условиям~\eqref{eq:model}--\eqref{eq:boundary}, а также дополнительному условию
на участке границы $\Gamma_1$:
\begin{equation}
    \label{eq:boundary-extra}
    \theta = \theta_b \text{ на } \Gamma_1.
\end{equation}

%Обратная задача~\eqref{eq:model}--\eqref{eq:boundary-extra} сводится к экстремальной задаче,
%состоящей в минимизации функционала
%\begin{equation}
%    \label{eq:cost}
%    J(\theta) = \frac{1}{2}\int\limits_\Gamma (\theta - \theta_b)^{2} d\Gamma \rightarrow \inf
%\end{equation}
%на решениях краевой задачи~\eqref{eq:model}--\eqref{eq:boundary}.

Если определить новую неизвестную функцию $\psi = a\theta + \alpha b \varphi$ то
задачу~\eqref{eq:model}--\eqref{eq:boundary-extra} можно переписать в виде
\begin{equation}
    \label{eq:model-psi}
    - a \Delta \theta + g (\theta) = \frac{\kappa}{\alpha}\psi, \quad
    \Delta \psi = 0, ;\ x \in \Omega.
\end{equation}

Краевые условия для температуры имеют вид
\begin{equation}
    \label{eq:boundary-psi-1}
    \begin{aligned}
        a \partial_n \theta + s \theta = q_b + s \theta_b \text{ на } \Gamma_1 \\
        a \partial_b \theta = q_b \text{ на } \Gamma_2.
    \end{aligned}
\end{equation}
Здесь $s > 0$ размерный параметр, $g(\theta) = b \kappa|\theta|\theta^3 + \frac{a\kappa}{\alpha}\theta$.
Соответствующие условия для $\psi$:
\begin{equation}
    \label{eq:boundary-psi-2}
    \begin{aligned}
        \Gamma_1 &: \; \alpha \partial_n \psi + \gamma \psi
        = \alpha b \gamma \theta_b^4
        + \alpha q_b + a \gamma \theta_b = r, \\
        \Gamma_2 &: \; \alpha \partial_n \psi = u.
    \end{aligned}
\end{equation}
Функция $u$ неизвестна и играет роль управления.

В таком случае полученные далее дифференциальные уравнения являются независимыми и
могут считаться последовательно,
что отличается от классического подхода и может иметь высокий исследовательский потенциал.

Алгоритм решения граничной обратной задачи состоит в рассмотрении задачи оптимального управления,
и заключается в минимизации функционала
\begin{equation}
    \label{eq:quality}
    J_\lambda = \frac{1}{2}\int\limits_{\Gamma_1} (\theta - \theta_b)^2 d\Gamma
    + \frac{\lambda}{2} \int\limits_{\Gamma_2} u^2 d\Gamma \rightarrow \inf
\end{equation}
на решениях~\eqref{eq:model-psi}--\eqref{eq:boundary-psi-2}.


Статья организована следующим образом.
В разд \. 2 вводятся необходимые пространства и операторы,
а также формальная постановка задачи оптимального управления.
Априорные оценки решения задачи~\eqref{eq:model-psi}--\eqref{eq:boundary-psi-2}, на основе которых доказана разрешимость
указанной краевой задачи и задачи оптимального управления (1),(4),(5), получены в разд\. 3.
В разд \. 4 представлен алгоритм решения задачи управления, работа которого
проиллюстрирована численными примерами.

