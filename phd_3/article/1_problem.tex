Модель теплопереноса в $\Omega$ с границей $\Gamma = \partial \Omega$:
\begin{equation}
    \label{model}
    - a \Delta\theta + b\kappa_a(|\theta|\theta^3- \varphi)=0,   \quad
    - \alpha \Delta \varphi + \kappa_a(\varphi-|\theta|\theta^3)=0,\; x\in\Omega.
\end{equation}
Для постановки задачи разобьём границу на две области:
$\Gamma := \partial \Omega =\overline{\Gamma}_0 \cup \overline{\Gamma}_1$,
где части границы $\Gamma_0, \Gamma_1$ не имеют пересечений.
Определим следующие функции
\begin{equation}
    \label{boundary}
    \begin{aligned}
        \Gamma &: \alpha(\partial_n\varphi+\varphi) = u, \\
        \Gamma_0 &: \partial_n\Gamma = q_b, \\
        \Gamma_1 &: a(\partial_n\theta+\theta) = r.
    \end{aligned}
\end{equation}

Обратная задача состоит в нахождении функций $r \in  \Gamma_1, \{\theta, \varphi\} \in \Omega $
удовлетворяющих условиям~\eqref{model}--\eqref{boundary}, а также дополнительному условию
на участке границы $\Gamma_0$:
\begin{equation}
    \label{boundary-extra}
    \Gamma = \theta_b \text{ на } \Gamma_0.
\end{equation}
Сформулированная обратная задача~\eqref{model}--\eqref{boundary-extra} сводится к экстремальной задаче,
состоящей в минимизации функционала
\begin{equation}
    \label{cost}
    J(\theta) = \frac{1}{2}\int\limits_\Gamma (\theta - \theta_b)^{2} d\Gamma_0 \rightarrow \inf
\end{equation}
на решениях краевой задачи~\eqref{model}--\eqref{boundary}.

Основные результаты работы состоят в получении красивых картинок для научного руководителя ¯\\_(ツ)_/¯.