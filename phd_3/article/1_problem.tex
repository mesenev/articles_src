Модель теплопереноса в $\Omega$ с границей $\Gamma = \partial \Omega$:
\begin{equation}
    \label{model}
    - a \Delta\theta + b\kappa_a(|\theta|\theta^3- \varphi)=0,   \quad
    - \alpha \Delta \varphi + \kappa_a(\varphi-|\theta|\theta^3)=0,\; x\in\Omega.
\end{equation}
Для постановки задачи разобьём границу на две области:
$\Gamma := \partial \Omega =\overline{\Gamma}_1 \cup \overline{\Gamma}_2$,
где части границы не имеют пересечений.
Определим следующие функции
\begin{equation}
    \label{boundary}
    \begin{aligned}
        \Gamma_1 :\; &\partial_n \theta = q_b,\; \theta = \theta_b, \\
        &\alpha\partial_n\varphi + \gamma (\varphi - \theta_b ^4 ) = 0, \\
        \Gamma_2 :\; & \partial_n \theta = q_b.
    \end{aligned}
\end{equation}
Для постановки задачи введём следующее обозначение:
\[
    r:= \alpha \partial_n \varphi \text{ на }\Gamma_2
\]
Обратная задача состоит в отыскании тройки функций $r \in  \Gamma_1, \{\theta, \varphi\} \in \Omega $
удовлетворяющих условиям~\eqref{model}--\eqref{boundary}, а также дополнительному условию
на участке границы $\Gamma_2$:
\begin{equation}
    \label{boundary-extra}
    \theta = \theta_b \text{ на } \Gamma_2.
\end{equation}
Сформулированная обратная задача~\eqref{model}--\eqref{boundary-extra} сводится к экстремальной задаче,
состоящей в минимизации функционала
\begin{equation}
    \label{cost}
    J(\theta) = \frac{1}{2}\int\limits_\Gamma (\theta - \theta_b)^{2} d\Gamma \rightarrow \inf
\end{equation}
на решениях краевой задачи~\eqref{model}--\eqref{boundary}.


Другим подходом может быть сведение исходной задачи к поиску некоторой
переменной $\psi = a\theta + \alpha b \varphi$:
\begin{equation}
    \label{eq:equation}
    -a \Delta \theta + g (\theta) = \frac{\kappa}{\alpha}\psi, \quad
    \Delta \psi = 0.
\end{equation}
С граничными условиями для $\theta$ из~\eqref{boundary}, дополненными следующими условиями для
$\psi$:
\begin{equation}
    \label{eq:boundary-2}
    \begin{aligned}
        \Gamma_1 &: \; \alpha \partial_n \psi + \gamma \psi = \alpha b \gamma \theta_b^4 + \alpha a \theta_b + \gamma a q_b = r, \\
        \Gamma_2 &: \; \partial_n \psi = u.
    \end{aligned}
\end{equation}
Здесь $g(\theta) = b \kappa|\theta|\theta^3 + \frac{a\kappa}{\alpha}\theta$.
Функция $u$, в свою очередь неизвеста и играет роль управления.