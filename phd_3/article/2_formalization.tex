\section{Формализация задачи управления}\label{sec:formalization}
Будем далее считать, что $\Omega \subset R^3$ - ограниченная строго липшицева область,
граница $\Gamma \equiv \partial \Omega$ состоит из конечного числа гладких кусков.
Будем предполагать, что для области $\Omega$ справедливы свойства 1, 2 из~\cite{}.
Через $L^p, \; 1 \leq p \leq \infty$ обозначим пространство Лебега, через $H^s$ - пространство Соболева $W^s_2$.
$H^s_0(\Omega)$ есть замыкание $C^\infty_0(\Omega)$ по норме пространства $H^s(\Omega)$.
Далее через обозначим скалярное произведение и норму в $L^2(\Omega)$
\[
    (f, g) = \int_\Omega f(x)g(x)dx, \; \| f \| = (f, f).
\]
Пусть $V = H^2_0(\Omega)$.
Условия на область $\Omega$ позволяют выбрать скалярное произведение в V в виде
$[u, v] = (\nabla u, \nabla v), \| f \|_V = \| \nabla f \|$.
Через $U$ обозначаем пространство $L^2(\Gamma)$ с нормой
$\|u\|_\Gamma=\left(\int_\Gamma u^2 d\Gamma\right)^{1/2}.$

Будем предполагать, что

$(i) \;\; a,b,\alpha,\kappa_a, \lambda =\textrm{Const} > 0 ,$

$(ii) \;\, \theta_b, \,g \, u \in U,\;\; r=a(\theta_b+q_b).$


Определим операторы $A_1\colon V \to V'$, $A_2\colon U \to V'$, используя
следующие равенства, справедливые для любых $y,z \in V$, $w\in U$:
\[
    (A_1 y,z) = a(\nabla y, \nabla z) + \int_{\Gamma_1} yz d\Gamma, \,
    (A_2 y, z) = \alpha (\nabla y, \nabla z) + \int_{\Gamma_1} \gamma yz d\Gamma.
\]
    {\it Слабым решением} задачи~\eqref{model}-\eqref{boundary} будем называть
пару $\theta, \psi \in V$ если
\begin{equation}
    \label{weak}
    A_1\theta  + g(\theta) = \frac{\kappa}{\alpha} \psi + f_1, \quad
    A_2\psi = f_2 + B_2 u.
\end{equation}

Здесь
$(f_1, v) = \int_{\Gamma_1} (a q_b + \theta_b)d\Gamma + \int_{\Gamma_2} a q_b d\Gamma$,\\
$(f_2, v) = \int_{\Gamma_1} r v d\Gamma$,\\
$(B_1 h,v) = \int_{\Gamma_1} h v d\Gamma$,\\
$(B_2 u, v) = \int_{\Gamma_2} u v d\Gamma$,\\
$g(\theta) = b\kappa_a \theta^3 |\theta| + \frac{a\kappa_a}{\alpha}\theta.$


