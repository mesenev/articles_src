\section{Формализация задачи управления}\label{sec:formalization}
Будем далее считать, что $\Omega \subset \mathbb{R}^3$ - ограниченная липшицева область.
Через $L^p, \; 1 \leq p \leq \infty$ обозначим пространство Лебега, через $H^s$ - пространство Соболева $W^s_2$.
Далее обозначим скалярное произведение и норму в $L^2(\Omega)$
\[
    (f, g) = \int_\Omega f(x)g(x)dx, \; \| f \| = (f, f).
\]
Пусть $V = H^1(\Omega)$.
Через $U$ обозначим пространство $L^2(\Gamma)$ с нормой
$\|u\|_{U} = \left(\int_{\Gamma_2} u^2 d\Gamma\right)^{1/2}.$

Будем предполагать, что \\
$(i) \;\; a, b,\alpha,\kappa_a, s, \lambda = \textrm{Const} > 0,$ \\
$(ii) \;\, \theta_b, q_b \in U$.

Определим операторы $A_1\colon V \to V'$, $A_2\colon V \to V'$, используя
следующие равенства, справедливые для любых $y,z \in V$, $w\in U$:
\[
    (A_1 y,z) = a(\nabla y, \nabla z) + s\int_{\Gamma_1} yz d\Gamma, \,
    (A_2 y, z) = \alpha (\nabla y, \nabla z) + \gamma \int_{\Gamma_1} yz d\Gamma.
\]
\textit{Слабым решением} задачи~\eqref{eq:model-psi},\eqref{eq:boundary-psi-2},\eqref{eq:boundary-psi-3}
будем называть пару $\theta, \psi \in V$ если
\begin{equation}
    \label{eq:weak}
    A_1\theta  + g(\theta) = \frac{\kappa}{\alpha} \psi + f_1, \quad
    A_2\psi = f_2 + B_2 u.
\end{equation}

Здесь
$(f_1, v) = \int_{\Gamma_1} (a q_b + s \theta_b) v d\Gamma + \int_{\Gamma_2} a q_b v d\Gamma
= B_1(a q_b + s \theta_b) + B_2(a q_b)$,\\
$(f_2, v) = \int_{\Gamma_1} r v d\Gamma = B_1 r$,\\
$(B_1 h,v) = \int_{\Gamma_1} h v d\Gamma$,\\
$(B_2 u, v) = \int_{\Gamma_2} u v d\Gamma$.

