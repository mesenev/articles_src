Лагранжиан задачи выглядит следующим образом:
\begin{equation}
    \label{eq:lagrange}
    L = J_\lambda + (A_1 \theta+ g(\theta) - \frac{\kappa}{\alpha}\psi -f_1, p_1)
    + (A_2 \psi - f_2 - B_2 u, p_2).
\end{equation}
\begin{equation}
    \label{eq:lagrange_t}
    <L'_\theta, v> = (\theta - \theta_b, v)_{\Gamma_1}
    + (A_1 v + g'(\theta)v, p_1) = 0.\, \forall v \in V.
\end{equation}
Здесь
\[
    g'(\theta) = 4 b \kappa |\theta |^ 3 + \frac{a \kappa}{\alpha}  > 0.
\]
Таким образом получаем, что
\begin{equation}
    \label{eq:a1}
    A_1 p_1 + g'(\theta) p_1 = - B_1(\theta - \theta_b).
\end{equation}

\begin{equation}
    \label{eq:lagrange_p}
    <L'_\psi, v> = - ( \frac{\kappa}{\alpha} v, p_1) + (A_2v, p_2) = 0.
\end{equation}
\begin{equation}
    \label{eq:a2}
    A_2 p_2 - \frac{\kappa}{\alpha} p_1 = 0.
\end{equation}

\begin{equation}
    \label{eq:lagrange_u}
    <L'_u, w> = \lambda(u, w)_{\Gamma_2} - (B_2w, p_2) = 0.
\end{equation}

Из последнего уравнения естественным образом следует
\begin{equation}
    \label{eq:a3}
    \lambda u = p_2 \text{ на } \Gamma_2.
\end{equation}

